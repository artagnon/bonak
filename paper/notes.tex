

\documentclass[10pt]{art}

\usepackage{minted, booktabs, setspace}
\usepackage{geometry}

% Fancy headers
\pagestyle{fancy}
\fancyhf{}
\fancyhead[R]{\footnotesize\textcolor{gray80}{{\thepage / \pageref{LastPage}}}}
\renewcommand{\sectionmark}[1]{\markboth{}{\thesection.\ #1}}
\fancyhead[L]{\footnotesize\textcolor{gray80}{{\MakeUppercase{\rightmark}}}}

% Section and section styles
\renewcommand{\thesection}{\Roman{section}}
\titleformat{\section}{\centering\scshape\Large\color{raspeberry}}{\thesection}{0.7em}{}
\titleformat{\section}{\strongenv\large\color{gray80}}{\thesection}{0.7em}{}

% Pack enumerate items and bibitems
\setlist{nolistsep}
\setlength{\bibitemsep}{.2\baselineskip plus .05\baselineskip minus .05\baselineskip}

% Code listings
\usemintedstyle{solarized-dark}
\setminted{escapeinside=~~}
\setmintedinline{escapeinside=~~}

% Fonts
\setmainfont{Avenir Next}
\setmonofont{Source Code Pro}
\defaultfontfeatures{Ligatures=TeX, Scale=MatchUppercase}

\title{Bonak: Notes on the formalization}
\author{Ramkumar Ramachandra}
\date{}

\begin{document}
\thispagestyle{empty}
\maketitle

The core of our formalization effort consists of inequalities in \emph{Peano arithmetic}. We start with one inequality primitive, namely $\leq$, and build up all our inequalities on the basis of that inequality. Two different proofs of \mintinline{coq}{p <= q} must be the same: we chose to use \mintinline{coq}{SProp} for the proof irrelevance.

\section{Notations in \texorpdfstring{\leq}{<=}}
On the outset, we defined various \emph{raising}, \emph{lowering}, and \emph{composition} operations on $\leq$:

\mint{coq}/~↑~ := (Hnm : n <= m) : n <= S m/
\mint{coq}/~⇑~ := (Hnm : n <= m) : S n <= S m/
\mint{coq}/~↓~ := (Hnm : S n <= m) : n <= m/
\mint{coq}/~⇓~ := (Hnm : S n <= S m) : n <= m/
\mint{coq}/~↕~ := (Hnm : n <= m) (Hmp : m <= p) : n <= p/

\section{An SProp primer}
\mintinline{coq}{SProp} is a replacement for standard propositions, or \mintinline{coq}{Prop}, that provides proof irrelevance.

There are three inhabitants of \mintinline{coq}{SProp}: \mintinline{coq}{sUnit} corresponding to $\top$, \mintinline{coq}{sEmpty} corresponding to $\bot$, and \mintinline{coq}{sProposition} corresponding to a definitionally proof-irrelevant term. The way \mintinline{coq}{SProp} implements definitional proof-irrelevance is a simple engineering detail: there is hard-coding in Coq to render two inhabitants of \mintinline{coq}{sProposition} trivially inter-convertible.

\begin{listing}[H]
  \begin{minted}{coq}
    Theorem SPropIrr (P : SProp) (x y : P) : x = y.
    Proof.
      by reflexivity.
    Abort. (* Type-check fails at Qed.
            * (=) : ∀ A, A -> A -> Prop, but we want to return an SProp. *)
  \end{minted}
\end{listing}

We can, however, define an \mintinline{coq}{=S} that returns an \mintinline{coq}{SProp}:

\begin{listing}[H]
  \begin{minted}{coq}
    Inductive eqsprop {A : SProp} (x : A) : A -> Prop :=
      eqsprop_refl : eqsprop x x.
    Infix "=S" := eqsprop (at level 70) : type_scope.
  \end{minted}
\end{listing}

\section{\texorpdfstring{\leq}{<=} in SProp}
We tried three different approaches to defining $\leq$ in \mintinline{coq}{SProp}.

\begin{enumerate}
  \item[(a)] A recursive definition.
    \begin{listing}[H]
      \begin{minted}{coq}
      Inductive sFalse : SProp :=.
      Inductive sTrue : SProp := sI.

      Fixpoint le m n : SProp :=
        match m with
        | 0 => sTrue
        | S m =>
          match n with
          | 0 => sFalse
          | S n => le m n
          end
        end.
     \end{minted}
    \end{listing}
  \item[(b)] An inductive definition.
    \begin{listing}[H]
      \begin{minted}{coq}
      Inductive le' (n : nat) : nat -> SProp :=
      | le_refl : n <= n
      | le_S_up : forall {m : nat}, n <= m -> n <= S m
      where "n <= m" := (le' n m) : nat_scope.

    Definition le := le'.
    \end{minted}
    \end{listing}
  \item[(c)] A refinement of the inductive definition; a nested implication.
    \begin{listing}[H]
      \begin{minted}{coq}
      Definition le n m := forall p, p <= n -> p <= m.
      \end{minted}
    \end{listing}
\end{enumerate}

(a) was unworkable right from the start, and we tried approach (b) for a while, before running into unification problems. We then settled on approach \c, because it made compositions of inequalities almost trivial to prove. Compare the following snippets defining transitivity of $\leq$:

\begin{listing}[H]
  \begin{minted}{coq}
    Theorem le_trans {p q n} : p <= q -> q <= n -> p <= n.
    (* ... a non-trivial proof ... *)
    Defined.
  \end{minted}
  \caption{\mintinline{coq}{le_trans} when $\leq$ is defined using an inductive approach}
\end{listing}

\begin{listing}[H]
  \begin{minted}{coq}
    Definition le_trans {n m p} (Hnm : n <= m) (Hmp : m <= p) : n <= p :=
      fun q (Hqn : le' q n) => Hmp _ (Hnm _ Hqn).
  \end{minted}
  \caption{\mintinline{coq}{le_trans} when $\leq$ is defined using a nested implication}
\end{listing}

Implications are naturally easier to work with, so the nested implication in \mintinline{coq}{SProp} was an easy choice. Many of the associtivity and commutativity proofs are then a simple \mintinline{coq}{reflexivity}:

\begin{listing}[H]
  \begin{minted}{coq}
  Theorem le_trans_comm4 {n m p} (Hmn : S n <= S m) (Hmp : S m <= S p) :
  ~⇓~ (Hmn ~↕~ Hmp) =S (~⇓~ Hmn) ~↕~ (~⇓~ Hmp).
    reflexivity.
  Defined.
  \end{minted}
\end{listing}

\section{The Fibonacci trick}
We'd initially defined \mintinline{coq}{Cubical} in a monolithic manner, and attempted to use \emph{well-founded induction} to build it up in steps. Induction is the process of successively building up a type by reasoning on the result of the previous iteration, and well-founded induction remembers all the previous levels. A simple induction is clearly insufficient for our purposes, and as it turned out, well-founded induction was a non-starter too.

Our initial \mintinline{coq}{Record} looked like:

\begin{listing}[H]
  \begin{minted}{coq}
    Record Cubical {n : nat} :=
    {
    csp {n'} (Hn' : n' <= n) : Type@{l'} ;
    hd {n'} {Hn' : S n' <= n} : csp Hn' -> csp (⇓ Hn') ;
    box {n' p} {Hn' : n' <= n} (Hp : p <= n') : csp Hn' -> Type@{l} ;
    tl {n'} {Hn' : S n' <= n} (D : csp Hn') :
      box (le_refl n') (hd D) -> Type@{l} ;
    (* ... *)
    }.
  \end{minted}
\end{listing}

To build it up in steps, we defined a \mintinline{coq}{Fixpoint} as follows, but ran into unification problems:

\begin{listing}[H]
  \begin{minted}[linenos]{coq}
    Theorem le_dec {n m} : n <= m -> {n = m} + {n <= pred m} + {m = O}.
    Admitted.
    Theorem lower_both {m n} : S m <= S n -> m <= n.
    Admitted.

    (* ... *)

    Fixpoint cubical {n : nat} : Cubical :=
    match n with
    | O => {|
      csp _ _ := unit;
      hd _ Hn' _ := ltac:(apply (le_discr Hn'));
      box _ _ _ _ _ := unit;
      tl _ Hn' _ _ := ltac:(apply (le_discr Hn'));
      (* ... *)
      |}
    | S n => let cn := cubical (n := n) in

    (* ... *)

    let hdn {n'} (Hn' : S n' <= S n) :=
    match le_dec (lower_both Hn') return cspn Hn' (* S n' <= S n *) ->
    cspn (⇓ Hn') (* n' <= S n *) with
    | inleft Hcmp =>
    match Hcmp with
    | left _ (* n' = n *) => fun D => D.1 (* n' = S n *)
    | right _ (* n' <= pred n *) => fun D => cn.(hd) D (* n' <= n *)
    end
    \end{minted}
  \caption{Type unification on the term \mintinline{coq}{D} fails}
\end{listing}

We then settled on remembering the previous two levels of induction, a middle-ground between plain induction and well-founded recursion, which we refer to as the Fibonacci trick. Our final \mintinline{coq}{Record} for the box looks like:

\begin{listing}[H]
  \begin{minted}{coq}
    Record PartialBoxPrev (n : nat) (csp : Type@{l'}) := {
      box' {p} (Hp : p.+1 <= n) : csp -> Type@{l} ;
      box'' {p} (Hp : p.+2 <= n) : csp -> Type@{l} ;
      subbox' {p q} {Hp : p.+2 <= q.+2} (Hq : q.+2 <= n) (ε : side) {D : csp} :
        box' (↓ (Hp ↕ Hq)) D -> box'' (Hp ↕ Hq) D;
    }.

    Record PartialBox (n p : nat) (csp : Type@{l'})
    (PB : PartialBoxBase n csp) := {
      box (Hp : p <= n) : csp -> Type@{l} ;
      subbox {q} {Hp : p.+1 <= q.+1} (Hq : q.+1 <= n) (ε : side) {D : csp} :
      box (↓ (Hp ↕ Hq)) D -> PB.(box') (Hp ↕ Hq) D;
      cohbox {q r} {Hpr : p.+2 <= r.+2} {Hr : r.+2 <= q.+2} {Hq : q.+2 <= n}
      {ε : side} {ω : side} {D: csp} (d : box (↓ (⇓ Hpr ↕ (↓ (Hr ↕ Hq)))) D) :
      PB.(subbox') (Hp := Hpr ↕ Hr) Hq ε (subbox (Hp := ⇓ Hpr) (↓ (Hr ↕ Hq)) ω d) =
      (PB.(subbox') (Hp := Hpr) (Hr ↕ Hq) ω (subbox (Hp := ↓ (Hpr ↕ Hr)) Hq ε d));
    }.
  \end{minted}
\end{listing}

To build up the box, we'd simply have to induction on \mintinline{coq}{p}:

\begin{listing}[H]
  \begin{minted}{coq}
    Definition mkcsp {n : nat} {C : Cubical n} : Type@{l'} :=
      { D : C.(csp) & C.(Box).(box) (le_refl n) D -> Type@{l} }.

    Definition mkBoxPrev {n} {C : Cubical n} :
      PartialBoxBase n.+1 mkcsp := {|
      box' {p} {Hp : p.+1 <= n.+1} {D : mkcsp} := C.(Box).(box) (⇓ Hp) D.1 ;
      box'' {p} {Hp : p.+2 <= n.+1} {D : mkcsp} := C.(PB).(box') (⇓ Hp) D.1 ;
      subbox' {p q} {Hp : p.+2 <= q.+2} {Hq : q.+2 <= n.+1} {ε} {D : mkcsp} {d} :=
        C.(Box).(subbox) (Hp := ⇓ Hp) (⇓ Hq) ε d ;
    |}.

    Definition mkBox {n} {C : Cubical n} p : PartialBox n.+1 p mkcsp mkPB.
      induction p.
      + (* ... *)
  \end{minted}
\end{listing}

\section{UIP}
We use UIP in the final coherence condition in \mintinline{coq}{cohbox} and \mintinline{coq}{cohcube}. If we weren't to use UIP in \mintinline{coq}{cohbox}, then we'd have something similar to:

\begin{listing}[H]
  \begin{minted}{coq}
    cohbox : subbox ( subbox ... ) = subbox ( subbox ... )
    coh2box : cohbox ... = cohbox ...
    ...
    cohNbox : cohPredNbox ... = cohPredNbox ...
  \end{minted}
\end{listing}

In attempting to write the proof without UIP, we'd have a relation between \mintinline{coq}{cohNbox} and \mintinline{coq}{cohPredNbox}, which, we expect to be tricky to prove.

\end{document}

\documentclass{article}
\usepackage[a4paper, margin=2em]{geometry}

\usepackage{hyperref, amssymb, stmaryrd, url, xcolor, amsmath, graphicx, yfonts, tikz, enumitem}
\usetikzlibrary{cd, arrows}
\setlist[enumerate]{left=0pt,nosep}

\newcommand{\defeq}{\triangleq}
\newcommand{\U}{\ensuremath{\mathsf{{HSet}}}}
\newcommand{\myframe}{\mathit{frame}}
\newcommand{\myframetype}{\mathit{frametype}}
\newcommand{\painting}{\mathit{painting}}
\newcommand{\paintingtype}{\mathit{paintingtype}}
\newcommand{\restrframe}{\mathit{restrframe}}
\newcommand{\restrframetype}{\mathit{restrframetype}}
\newcommand{\restrpainting}{\mathit{restrpainting}}
\newcommand{\cohframe}{\mathit{cohframe}}
\newcommand{\cohpainting}{\mathit{cohpainting}}

\newcommand{\udensdash}[1]{%
    \tikz[baseline=(todotted.base)]{
        \node[inner sep=1pt,outer sep=0pt] (todotted) {$#1$};
        \draw[densely dashed] (todotted.south west) -- (todotted.south east);
    }%
}%

\begin{document}

\section{The recursive structure of iterated parametricity}
Iterated parametricity (in indexed form) produces a family of type
families of the following form:
\begin{equation*}
  \begin{array}{llr}
    X_0 & :              & \U                                                                        \\
    X_1 & :              & X_0 \times X_0 \rightarrow  \U                                            \\
    X_2 & : \Pi a b c d. & X_1(a,b) \times X_1 (c,d) \times X_1(a,c) \times X_1 (b,d) \rightarrow \U \\
    \ldots
  \end{array}
\end{equation*}

Such family can be recursively described by the equations given on Table ?.

The well-foundedness of the construction can be justified by the observation that:
\begin{enumerate}
  \item Defining $\myframe^{n,p}$ for $p \leq n$ requires the definition
        of $\painting^{n-1,p-1}$, thus also $\myframe^{n-1,p-1}$ by typing
        dependency, as well as $\restrframe^{n,p-1}$, where
        $\restrframe^{n,p-1}$ in turn depends in $\myframe^{n,p-1}$ for its
        typing.

        Moreover, $\myframe^{n,p}$ depends on $\myframe^{n,p-1}$, so these
        dependencies are between the blocks of $\myframe^{n,k}$ for all
        $k\leq p$, as well as $\myframe^{n-1,k}$, $\painting^{n-1,k}$ and
        $\restrframe^{n,k}$ for all $k\leq p-1$, which we respectively
        abbreviate as $\myframe^{n,[0,...,p]}$,
        $\myframe^{n-1,[0,...,p-1]}$, $\painting^{n-1,[0,...,p-1]}$ and
        $\restrframe^{n,[0,...,p-1]}$.

        Also, $\restrframe^{n,p-1}$ depends on $n$.

        Schematically, using dotted arrows for dependencies in
        the type and plain arrows for dependencies in the definition, this
        gives the following dependencies for $p\leq n$ fixed:
        \begin{center}
          \begin{tikzcd}
            \myframe^{n,[0,...,p]} \arrow[d] \arrow[ddr]\\
            \painting^{n-1,[0,...,p-1]} \arrow[d, dotted] \\
            \myframe^{n-1,[0,...,p-1]} &
            \restrframe^{n,[0,...,p-1]} \arrow[l, dotted] \arrow[uul, dotted, "\mbox{$[0,...,p-1]$}"', near start, shift right=2mm, hook']\\
          \end{tikzcd}
        \end{center}

        In particular, the mutual dependency on one side of
        $\myframe^{n,[0,...,p-1]}$ within the type of
        $\restrframe^{n,[0,...,p-1]}$ and on the other side of
        $\restrframe^{n,[0,...,p-1]}$ within the definition of
        $\myframe^{n,[0,...,p]}$ requires to mutually define the type of
        $\restrframe^{n,[0,...,p-1]}$ together with $\myframe^{n,p}$ (or
        $\myframe^{n,[0,...,p]}$) as a function from
        $\restrframe^{n,[0,...,p-1]}$.

        To talk about sequences of $\myframe$, $\painting$ or $\restrframe$,
        we need to specify their types. The type of $\restrframe$ will be
        defined mutually with $\myframe$ of the same level, but the type of
        $\myframe$, $\painting$ can be defined in advance, the second one
        being however dependent on the first one. In general, components of
        such sequences will have a type dependent on the previous components
        of the sequence, as it is the case for $\restrframe$. So, the type of
        a sequence of length $p$ will be expressed as a $n$-iterated
        $\Sigma$-type. In the case of $\myframe$ and $\painting$, there will
        be however no dependency, so an ordinary product can be used. If
        $\myframe^{n,[0,...,p]}$ is such a sequence of length $p+1$, we write
        $\myframe^{n,[0,...,p-1]}$ for its immediate subsequence of length $p$
        and $\myframe^{n,p}$ for its last component. We can now define the
        types of sequences of $\myframe$ and $\painting$ as follows:

        $$
          \begin{array}{llcl}
            \myframetype^{n,[0,...,p]}  &                                                     & :      & Type_{l+1}                                                               \\
            \myframetype^{n,[]}         &                                                     & \defeq & unit                                                                     \\
            \myframetype^{n,[0,...,p]}  &                                                     & \defeq & \myframetype^{n,[0,...,p-1]} \times Type_l                               \\
            \\
            \paintingtype^{n,[0,...,p]} & (\myframe^{n,[0,...,p]}:\myframetype^{n,[0,...,p]}) & :      & Type_{l+1}                                                               \\
            \paintingtype^{n,[]}        & \star                                               & \defeq & unit                                                                     \\
            \paintingtype^{n,[0,...,p]} & (\myframe^{n,[0,...,p]})                            & \defeq & \paintingtype^{n,[0,...,p-1]} \times (\myframe^{n,p} \rightarrow Type_l) \\
          \end{array}
        $$

        Formally, this can be described as:
        $$
          \begin{array}{llcl}
            \myframe^{n,[0,...,p]}                                     &
            \multicolumn{3}{l}{
              \left(\begin{array}{l}
                        \myframe^{n-1,[0,...,p-1]}:\myframetype^{n-1,[0,...,p-1]}                               \\
                        \painting^{n-1,[0,...,p-1]}:\paintingtype^{n-1,[0,...,p-1]}(\myframe^{n-1,[0,...,p-1]}) \\
                        \restrframe^{n,[0,...,p-1]}:\restrframetype^{n,[0,...,p-1]}                             \\
                      \end{array}\right) : \myframetype^{n-1,[0,...,p]}
            }                                                                                         \\
            \myframe^{n,[0]}                                           &
            \left(\begin{array}{l}
                      \star \\
                      \star \\
                      \star \\
                    \end{array}\right)                                     & \defeq & (\star,unit)      \\
            \myframe^{n,[0,...,p+1]}                                   &
            \left(\begin{array}{l}
                      \myframe^{n-1,[0,...,p]}  \\
                      \painting^{n-1,[0,...,p]} \\
                      \restrframe^{n,[0,...,p]} \\
                    \end{array}\right) & \defeq &
            \left(\begin{array}{l}
                      \myframe^{n,[0,...,p]}
                      \left(\begin{array}{l}
                          \myframe^{n-1,[0,...,p-1]}  \\
                          \painting^{n-1,[0,...,p-1]} \\
                          \restrframe^{n,[0,...,p-1]} \\
                        \end{array}\right)     , \\
                      \Sigma d:\myframe^{n,[0,...,p]}
                      \left(\begin{array}{l}
                          \myframe^{n-1,[0,...,p-1]}  \\
                          \painting^{n-1,[0,...,p-1]} \\
                          \restrframe^{n,[0,...,p-1]} \\
                        \end{array}\right).\, \\
                      \quad \Pi\omega.\,\painting^{n-1,p}(\restrframe^{n,p}(d))     \\
                    \end{array}\right)                    \\
            \restrframetype^{n,[0,...,p-1]}                            &
            \multicolumn{3}{l}{
              \left(\begin{array}{l}
                        \myframe^{n-1,[0,...,p-1]}:\myframetype^{n-1,[0,...,p-1]}                               \\
                        \painting^{n-1,[0,...,p-1]}:\paintingtype^{n-1,[0,...,p-1]}(\myframe^{n-1,[0,...,p-1]}) \\
                      \end{array}\right) : Type_{l}
            }                                                                                         \\
            \restrframetype^{n,[]}                                     &
            \left(\begin{array}{l}
                      \star \\
                      \star \\
                    \end{array}\right)                                     & \defeq &
            unit                                                                                      \\
            \restrframetype^{n,[0,...,p]}                              &
            \left(\begin{array}{l}
                      \myframe^{n-1,[0,...,p]}  \\
                      \painting^{n-1,[0,...,p]} \\
                    \end{array}\right) & \defeq &
            \begin{array}{l}
              \Sigma \restrframe^{n,[0,...,p-1]}                            \\
              \quad:\restrframetype^{n,[0,...,p-1]}
              \left(\begin{array}{l}
                        \myframe^{n-1,[0,...,p-1]}  \\
                        \painting^{n-1,[0,...,p-1]} \\
                      \end{array}\right)
              .\,                                                           \\
              \Pi q\leq n.\,\Pi \omega.\,\myframe^{n,[0,...,p]}
              \left(\begin{array}{l}
                        \myframe^{n-1,[0,...,p-1]}  \\
                        \painting^{n-1,[0,...,p-1]} \\
                        \restrframe^{n,[0,...,p-1]} \\
                      \end{array}\right) \\
            \end{array} \\
          \end{array}
        $$

        TODO: discuss $p\leq q \leq n$ vs $0 \leq q \leq n-p$.

        TODO: remark that we abuse terminology for frame, since, informally, a
        frame is an inhabitant of the type frame, while formally, it is the
        type of (an informal) frame which we call frame. (See same abuse of
        terminology when talking about a list, and using the term list to
        refer to the type of a list.)

        TODO: when to talk about the ``prefix'' $X_0, ..., X_{n-1}$?

        %% \item By induction on $p$, we deduce this for all $p\leq n$:
        %% \begin{center}
        %% \begin{tikzcd}
        %% frame^{n,[0,...,p...n]} \arrow[d] \arrow[ddr]\\
        %% painting^{n-1,[0,...,p-1...n-1]} \arrow[d, dotted] \\
        %% frame^{n-1,[0,...,p-1...n-1]} &
        %%    \restrframe^{n,[0,...,p-1...n-1]} \arrow[l, dotted] \arrow[uul, dotted, shift right=2mm]\\
        %% \end{tikzcd}
        %% \end{center}

        %% \item The same dependencies for $\myframe^{n-1,p}$ for $p \leq n-1$ gives:
        %% \begin{center}
        %% \begin{tikzcd}
        %% frame^{n-1,p} \arrow[d] \arrow[ddr, shift right=1mm]\\
        %% painting^{n-2,k<p} \arrow[d, dotted] \\
        %% frame^{n-2,k<p} & \restrframe^{n-1,k<p} \arrow[l, dotted] \arrow[uul, dotted, shift right=1mm]\\
        %% \end{tikzcd}
        %% \end{center}

        %% Similarly, $\restrframe^{n-1,p}$ and $\myframe^{n-1,p}$ as a function from
        %% all $\restrframe^{n-1,k}$ for $k < p$ have to be mutually defined.

  \item Defining $\painting^{n,p}$, in addition to the type dependency
        in $\myframe^{n,p}$, requires the definition of
        $\painting^{n-1,p-1}$, thus also $\myframe^{n-1,p-1}$ by typing
        dependency, as well as $\restrframe^{n,p-1}$. Since
        $\painting^{n,p}$ also depends on $\painting^{n,p+1}$ when $p<n$,
        these dependencies are actually between $\painting^{n,[p,...,n]}$,
        $\painting^{n-1,[p,...,n-1]}$, $\myframe^{n,[p,...,n]}$,
        $\myframe^{n-1,[p,...,n-1]}$ and $\restrframe^{n,[p,...,n]}$. Also,
        for the case $p=n$, this also requires the assumption of an
        inhabitant $X_{n}$ of $\nu$Set$^{=n}$. Finally,
        $\myframe^{n,[p,...,n]}$, $\myframe^{n-1,[p,...,n-1]}$ and
        $\restrframe^{n,[p,...,n-1]}$ respectively depend on the earlier
        values $\myframe^{n,[0,...,p-1]}$, $\myframe^{n-1,[0,...,p-1]}$ and
        $\restrframe^{n,[0,...,p-1]}$.  Below, we use dashed underline and
        dashed arrows for dependencies in subconstructions, here in the
        construction of $\myframe$ at level $n-1$ and $\restrframetype$ at
        level $n$:
        \begin{center}
          \begin{tikzcd}
            & & \painting^{n,[p,...,n]} \arrow[dd, bend right=70] \arrow[dddr] \arrow[d, dotted]\\
            \udensdash{\myframe}^{n,[0,...,p-1]} \arrow[d, dashed] \arrow[ddr, dashed] & &
            \udensdash{\myframe}^{n,[p,...,n]} \arrow[ll, dotted] \arrow[d, dashed] \arrow[ddr, dashed, shift right=2mm] \\
            \painting^{n-1,[0,...,p-1]} \arrow[d, dotted] & &
            \painting^{n-1,[p,...,n-1]} \arrow[ll, dotted] \arrow[d, dotted] \\
            \myframe^{n-1,[0,...,p-1]} &
            \udensdash{\restrframe}^{n,[0,...,p-1]} \arrow[l, dotted] \arrow[uul, dotted, shift right=2mm, hook'] &
            \myframe^{n-1,[p,...,n-1]} \arrow[ll, dotted, bend left=15] &
            \udensdash{\restrframe}^{n,[p,...,n-1]},X_n \arrow[ll, dotted, bend left=15] \arrow[l, dotted] \arrow[uul, dotted, hook']\\
          \end{tikzcd}
        \end{center}

  \item Defining $\restrframe^{n,p}$ for $p\leq n-1$, in addition to the
        mutual type dependency in $\myframe^{n,p}$ and the dependency in
        $\myframe^{n-1,p}$, requires $\restrpainting^{n-1,p-1}$, thus also
        $\restrframe^{n-1,p-1}$ and $\myframe^{n-1,p-1}$ mutually, as well
        as $\myframe^{n-2,p-2}$, $painting^{n-2,p-1}$ and
        $\painting^{n-2,p-1}$ by typing dependency, as well as
        $\cohframe^{n,p-1}$. Since $\restrframe^{n,p}$ also depends on
        $\restrframe^{n,p-1}$, these dependencies are between the blocks
        $\myframe^{n,[0..p]}$, $\myframe^{n-1,[0..p]}$,
        $\myframe^{n-2,[0..p]}$, $\painting^{n-1,[0..p-1]}$ and
        $\restrframe^{n,[0..p-1]}$, where cohframes$^{n,k}$ in turn depends
        in $\restrframe^{n,k}$ for its typing. Below, we use again dashed
        underline and dashed arrows for dependencies in subconstructions,
        here in the construction of $\myframe$ at level $n-1$ and $n-2$ and $\restrframetype$ at level $n$ and $n-1$ :

        \begin{center}
          \begin{tikzcd}
            \udensdash{\myframe}^{n,[0..p]} \arrow[d, dashed] \arrow[ddr, dashed] \\
            \painting^{n-1,[0..p-1]} \arrow[d, dotted]\\
            \udensdash{\myframe}^{n-1,[0..p-1]} \arrow[d, dashed] \arrow[ddr, dashed] &
            \restrframe^{n,[0..p-1]} \arrow[d] \arrow[ddr] \arrow[l, dotted] \arrow[uul, dotted, shift right=2mm, hook']\\
            \painting^{n-2,[0..p-2]} \arrow[d, dotted] &
            \restrpainting^{n-1,[0..p-2]} \arrow[d, dotted] \arrow[l, dotted] \arrow[uul, dotted, hook']\\
            \myframe^{n-2,[0..p-2]} &
            \restrframe^{n-1,[0..p-2]} \arrow[l, dotted] \arrow[uul, dotted, shift right=2mm, hook'] &
            \cohframe^{n,[0..p-2]} \arrow[l, dotted] \arrow[uul, dotted, shift right=2mm, hook']\\
          \end{tikzcd}
        \end{center}

        In particular, the mutual dependency of $\restrframe^{n,[0,...,p-1]}$
        in the type of $\cohframe^{n,[0,...,p-2]}$ and of
        $\cohframe^{n,[0,..,p-2]}$ in the definition of
        $\restrframe^{n,[0,...,p-1]}$ requires to mutually define the type of
        $\cohframe^{n,[0,...,p-2]}$ and $\restrframe^{n,p}$
        ($\restrframe^{n,[0,...,p-1]}$) as a function from
        $\cohframe^{n,[0,...,p-2]}$.

  \item Defining $\restrpainting^{n,p}$ for $p\leq n-1$, in addition to
        the type dependency in $\painting^{n,p}$, $\painting^{n-1,p}$, $\restrframe^{n,p}$, $\myframe^{n,p}$ and
        $\myframe^{n-1,p}$, requires the definition of $\restrpainting^{n-1,k}$,
        thus also $\restrframe^{n-1,k}$ and $\myframe^{n-2,p}$ by typing
        dependency, as well as $\cohframe^{n,k}$ for $p \geq k$ and the
        assumption of an inhabitant $X_{n-1}$ of nuSet$^{=n-1}$:
        \begin{center}
          \begin{tikzcd}
            & & painting^{n,p} \arrow[d, dotted] \\
            & & frame^{n,p} \\
            & & painting^{n-1,p} \arrow[d, dotted] & \restrpainting^{n,p} \arrow[ddlll] \arrow[dddll] \arrow[d, dotted] \arrow[uul, dotted] \arrow[ul, dotted] \arrow[l, dotted] \\
            & & frame^{n-1,p} & \restrframe^{n,p} \arrow[uul, dotted] \arrow[l, dotted]\\
            restrpainting^{n-1,[p,...,n]} \arrow[d, dotted] \\
            restrframe^{n-1,[p,...,n]} & \cohframe^{n,[p,...,n]},X_{n-1} \arrow[l, dotted]\\
          \end{tikzcd}
        \end{center}

  \item Defining $\cohframe^{n+1,p}$ for $p\leq n-1$, in addition to the
        type dependency in $\restrframe^{n+1,p}$, $\restrframe^{n,p}$,
        $\myframe^{n+1,p}$, $\myframe^{n,p}$ and $\myframe^{n-1,p}$, requires defining
        $\cohpainting^{n,k}$, thus also $\cohframe^{n,k}$, $\restrpainting^{n,k}$, $\restrpainting^{n-1,k}$, $\painting^{n,k}$, $\painting^{n-1,k}$ and $\painting^{n-2,k}$ by typing
        dependency, for $k < p$. By reasoning in sets, no higher-dimensional
        coherence frame is needed:
        \begin{center}
          \begin{tikzcd}
            \myframe^{n+1,p} \\
            \painting^{n,p} \arrow[d, dotted] \\
            \myframe^{n,p} & \restrframe^{n+1,p} \arrow[l, dotted] \arrow[uul, dotted, hook'] \\
            \painting^{n-1,p} \arrow[d, dotted] &
            \restrpainting^{n-1,p} \arrow[d, dotted] \arrow[l, dotted] \arrow[uul, dotted, hook'] \\
            \myframe^{n-1,p} &
            \restrframe^{n,p} \arrow[l, dotted] \arrow[uul, dotted, hook'] &
            \cohframe^{n+1,p} \arrow[d] \arrow[l, dotted] \arrow[uul, dotted, hook'] \\
            \painting^{n-2,p} \arrow[d, dotted] &
            \restrpainting^{n-1,p} \arrow[d, dotted] \arrow[l, dotted] \arrow[uul, dotted, hook'] &
            \cohpainting^{n,k<p} \arrow[d, dotted] \arrow[l, dotted] \arrow[uul, dotted, hook']\\
            \myframe^{n-2,p} &
            \restrframe^{n-1,p} \arrow[l, dotted] \arrow[uul, dotted, hook'] &
            \cohframe^{n,k<p} \arrow[l, dotted] \arrow[uul, dotted, hook']\\
          \end{tikzcd}
        \end{center}

  \item Defining $\cohpainting^{n+1,p}$ for $p\leq n-1$, in addition to
        the type dependency in $\restrpainting^{n+1,p}$,
        $\restrpainting^{n,p}$, $\painting^{n+1,p}$, $\painting^{n,p}$,
        $\painting^{n-1,p}$, $\restrframe^{n+1,p}$, $\restrframe^{n,p}$,
        $\myframe^{n+1,p}$, $\myframe^{n,p}$ and $\myframe^{n-1,p}$ requires the
        definition of $\cohpainting^{n,k}$, thus also $\cohframe^{n,k}$ by
        typing dependency, for also $p \geq k$. By reasoning in sets, no
        higher-dimensional coherence frame is needed:
        \begin{center}
          \begin{tikzcd}
            & & painting^{n+1,p} \arrow[d, dotted] \\
            & & frame^{n+1,p} \\
            & & painting^{n,p} \arrow[d, dotted] & \restrpainting^{n+1,p} \arrow[d, dotted] \arrow[uul, dotted] \arrow[ul, dotted] \arrow[l, dotted] \\
            & & frame^{n,p} & \restrframe^{n+1,p} \arrow[uul, dotted] \arrow[l, dotted] \\
            & & painting^{n-1,p} \arrow[d, dotted] & \restrpainting^{n,p} \arrow[d, dotted] \arrow[uul, dotted] \arrow[ul, dotted] \arrow[l, dotted] & \cohpainting^{n+1,p} \arrow[ddllll] \arrow[l, dotted] \arrow[uul, dotted] \arrow[d, dotted]\\
            & & frame^{n-1,p} & \restrframe^{n,p} \arrow[l, dotted] & \cohframe^{n+1,p} \arrow[l, dotted] \arrow[uul, dotted]\\
            cohpainting^{n,k\geq p} \arrow[d, dotted] \\
            cohframe^{n,k\geq p} \\
          \end{tikzcd}
        \end{center}
\end{enumerate}

Looking at the dependencies in the reverse order, it is thus enough:
\begin{enumerate}
  \item To assume $\myframe^{n-2,k}$ and $\painting^{n-2,k}$ for all $k< p$
        in order to mutually define, for all $p\leq n$, the type of
        $\restrframe^{n-1,p}$ and $\myframe^{n-1,p}$ as a function from all
        $\restrframe^{n-1,k}$ for $k< p$. Using red for assumptions, this
        can be pictured as:
        \begin{center}
          \begin{tikzcd}
            \myframe^{n-1,p}(\textcolor{red}{restrframe^{n-1,k<p}}) \arrow[d] \\
            \textcolor{red}{painting^{n-2,k<p}} \arrow[d, dotted] \\
            \textcolor{red}{frame^{n-2,k<p}} \\
          \end{tikzcd}
        \end{center}

  \item To further assume $\restrframe^{n-1,k}$ for all $k<p$ to get
        $\myframe^{n-1,p}$ for all $p\leq n-1$:
        \begin{center}
          \begin{tikzcd}
            \myframe^{n-1,p} \arrow[d] \arrow[ddr]\\
            \textcolor{red}{painting^{n-2,k<p}} \arrow[d, dotted] \\
            \textcolor{red}{frame^{n-2,k<p}} & \textcolor{red}{restrframe^{n-1,k<p}} \arrow[l, dotted] \arrow[uul, dotted, shift right=2mm]\\
          \end{tikzcd}
        \end{center}

  \item To rely on $\myframe^{n-2,k}$, $\painting^{n-2,k}$ and
        $\restrframe^{n-1,k}$, jointly with $\myframe^{n-1,k}$, and now for all
        $k \geq p$, as well as an inhabitant $X_{n-2}$ of $\nu$Set$^{=n-2}$
        to define $\painting^{n-1,p}$ for all $p \leq n-1$:
        \begin{center}
          \begin{tikzcd}
            & & & painting^{n-1,p} \arrow[ddlll] \arrow[dddll] \arrow[d, dotted] \\
            \myframe^{n-1,k\geq p} \arrow[d] \arrow[ddr] & & & frame^{n-1,p} \arrow[d] \arrow[ddr]\\
            \textcolor{red}{painting^{n-2,k\geq p}} \arrow[d, dotted] & & & \textcolor{red}{painting^{n-2,k<p}} \arrow[d, dotted] \\
            \textcolor{red}{frame^{n-2,k\geq p}} & \textcolor{red}{restrframe^{n-1,k\geq p},X_{n-2}} \arrow[l, dotted] & & \textcolor{red}{frame^{n-2,k<p}} & \textcolor{red}{restrframe^{n-1,k<p}} \arrow[l, dotted] \arrow[uul, dotted, shift right=2mm]\\
          \end{tikzcd}
        \end{center}

  \item To rely on $\myframe^{n-1,k}$ and $\painting^{n-1,k}$ for all $k <
          p$ to now construct the $\myframe^{n,p}$ for $p \leq n-1$ as a function
        from all $\restrframe^{n,k}$ for $k\leq p$:

        \begin{tikzcd}
          & & & frame^{n,p}(\textcolor{red}{restrframe^{n,k<p}}) \arrow[d] \\
          & & & painting^{n-1,p} \arrow[ddlll] \arrow[dddll] \arrow[d, dotted] \\
          & & & frame^{n-1,p} \arrow[d] \\
          \textcolor{red}{painting^{n-2,k\geq p}} \arrow[d, dotted] & & & \textcolor{red}{painting^{n-2,k<p}} \arrow[d, dotted] \\
          \textcolor{red}{frame^{n-2,k\geq p}} & \textcolor{red}{restrframe^{n-1,k\geq p},X_{n-2}} \arrow[l, dotted] & & \textcolor{red}{frame^{n-2,k<p}} & \textcolor{red}{restrframe^{n-1,k<p}} \arrow[l, dotted] \arrow[uul, dotted, shift right=2mm]\\
        \end{tikzcd}

  \item To additionally assume $\restrpainting^{n-1,k}$ for $k < p$ to
        mutually define for $p \leq n-1$ the type of $\cohframe^{n,p}$ and
        $\restrframe^{n,p}$ as a function from all $\cohframe^{n,k}$ for
        $k\leq p$, as well as, consequently, $\myframe^{n,p}$ for all $p\leq n$

        \begin{tikzcd}
          & & & frame^{n,p} \arrow[d] \arrow[ddr] \\
          & & & painting^{n-1,p} \arrow[ddlll] \arrow[dddll] \arrow[d, dotted] \\
          & & & frame^{n-1,p} \arrow[d] \arrow[ddr] & \restrframe^{n,k<p}(\textcolor{red}{cohframe^{n,k<p}}) \arrow[d] \arrow[uul, dotted, shift right=2mm] \arrow[l, dotted]\\
          \textcolor{red}{painting^{n-2,k\geq p}} \arrow[d, dotted] & & & \textcolor{red}{painting^{n-2,k<p}} \arrow[d, dotted] & \textcolor{red}{restrpainting^{n-1,k<p}} \arrow[d, dotted]\\
          \textcolor{red}{frame^{n-2,k\geq p}} & \textcolor{red}{restrframe^{n-1,k\geq p},X_{n-2}} \arrow[l, dotted] & & \textcolor{red}{frame^{n-2,k<p}} & \textcolor{red}{restrframe^{n-1,k<p}} \arrow[l, dotted] \arrow[uul, dotted, shift right=2mm]\\
        \end{tikzcd}

  \item To additionally assume $\cohframe^{n,k}$ for $k \leq p$ to get
        all $\restrframe^{n,k}$ for all $k \leq p$:

        \begin{tiny}
          \begin{tikzcd}
            & & & frame^{n,p} \arrow[d] \arrow[ddr]\\
            & & & painting^{n-1,p} \arrow[ddlll] \arrow[dddll] \arrow[d, dotted] \\
            & & & frame^{n-1,p} \arrow[d] \arrow[ddr] & \restrframe^{n,k<p} \arrow[d] \arrow[ddr] \arrow[uul, dotted, shift right=2mm] \arrow[l, dotted]\\
            \textcolor{red}{painting^{n-2,k\geq p}} \arrow[d, dotted] & & & \textcolor{red}{painting^{n-2,k<p}} \arrow[d, dotted] & \textcolor{red}{restrpainting^{n-1,k<p}} \arrow[l, dotted] \arrow[d, dotted]\\
            \textcolor{red}{frame^{n-2,k\geq p}} & \textcolor{red}{restrframe^{n-1,k\geq p},X_{n-2}} \arrow[l, dotted] & & \textcolor{red}{frame^{n-2,k<p}} & \textcolor{red}{restrframe^{n-1,k<p}} \arrow[l, dotted] \arrow[uul, dotted, shift right=2mm] & \textcolor{red}{cohframe^{n,k<p}} \arrow[l, dotted] \arrow[uul, dotted, shift right=2mm]\\
          \end{tikzcd}
        \end{tiny}

  \item To rely on $\myframe^{n-2,k\geq p}$, $\painting^{n-2,k\geq p}$ and
        $\restrframe^{n-1,k\geq p}$ to build $\myframe^{n-1,k\geq p}$ and
        $\painting^{n-1,k\geq p}$ for $p \leq n$:

        \begin{tiny}
          \begin{tikzcd}
            & & & frame^{n,p} \arrow[d] \arrow[ddr]\\
            \painting^{n-1,k\geq p} \arrow[d, dotted] \arrow[dddr] \arrow[dd, bend right=70] & & & painting^{n-1,p} \arrow[ddlll] \arrow[dddll] \arrow[d, dotted] \\
            \myframe^{n-1,k\geq p} \arrow[d] \arrow[ddr] & & & frame^{n-1,p} \arrow[d] \arrow[ddr] & \restrframe^{n,k<p} \arrow[d] \arrow[ddr] \arrow[uul, dotted, shift right=2mm] \arrow[l, dotted]\\
            \textcolor{red}{painting^{n-2,k\geq p}} \arrow[d, dotted] & & & \textcolor{red}{painting^{n-2,k<p}} \arrow[d, dotted] & \textcolor{red}{restrpainting^{n-1,k<p}} \arrow[l, dotted] \arrow[d, dotted]\\
            \textcolor{red}{frame^{n-2,k\geq p}} & \textcolor{red}{restrframe^{n-1,k\geq p},X_{n-2}} \arrow[l, dotted] & & \textcolor{red}{frame^{n-2,k<p}} & \textcolor{red}{restrframe^{n-1,k<p}} \arrow[l, dotted] \arrow[uul, dotted, shift right=2mm] & \textcolor{red}{cohframe^{n,k<p}} \arrow[l, dotted] \arrow[uul, dotted, shift right=2mm]\\
          \end{tikzcd}
        \end{tiny}

        Note that this can be done in parallel with the steps building $\myframe^{n,p}$.

  \item To assume $\restrpainting^{n-1,k}$ and $\cohframe^{n,k}$ for $k >
          p$ to build $\restrframe^{n,k\geq p}$:

        \begin{tiny}
          \begin{tikzcd}
            & & & frame^{n,p} \arrow[d] \arrow[ddr]\\
            \painting^{n-1,k\geq p} \arrow[d, dotted] \arrow[dddr] \arrow[dd, bend right=70] & & & painting^{n-1,p} \arrow[ddlll] \arrow[dddll] \arrow[d, dotted] \\
            \myframe^{n-1,k\geq p} \arrow[d] \arrow[ddr] & \restrframe^{n,k\geq p} \arrow[d] \arrow[ddr] \arrow[l, dotted] & & frame^{n-1,p} \arrow[d] \arrow[ddr] & \restrframe^{n,k<p} \arrow[d] \arrow[ddr] \arrow[uul, dotted, shift right=2mm] \arrow[l, dotted]\\
            \textcolor{red}{painting^{n-2,k\geq p}} \arrow[d, dotted] & \textcolor{red}{restrpainting^{n-1,k\geq p}} \arrow[l, dotted] \arrow[d, dotted] & &
            \textcolor{red}{painting^{n-2,k<p}} \arrow[d, dotted] &
            \textcolor{red}{restrpainting^{n-1,k<p}} \arrow[l, dotted] \arrow[d, dotted]\\
            \textcolor{red}{frame^{n-2,k\geq p}} & \textcolor{red}{restrframe^{n-1,k\geq p},X_{n-2}} \arrow[l, dotted] & \textcolor{red}{cohframe^{n,k\geq p},X_{n-1}} \arrow[l, dotted] &
            \textcolor{red}{frame^{n-2,k<p}} &
            \textcolor{red}{restrframe^{n-1,k<p}} \arrow[l, dotted] \arrow[uul, dotted, shift right=2mm] &
            \textcolor{red}{cohframe^{n,k<p}} \arrow[l, dotted] \arrow[uul, dotted, shift right=2mm] \\
          \end{tikzcd}
        \end{tiny}

  \item To rely on $\painting^{n-1,k\geq p}$ and $\restrframe^{n,p\leq n-1}$ to build $\painting^{n,p}$ for $p \leq n$:

        \begin{tiny}
          \begin{tikzcd}
            & & & painting^{n,p} \arrow[ddlll] \arrow[dddr] \arrow[d, dotted]\\
            & & & frame^{n,p} \arrow[d] \arrow[ddr]\\
            \painting^{n-1,k\geq p} \arrow[d, dotted] \arrow[dddr] \arrow[dd, bend right=70] & & & painting^{n-1,p} \arrow[ddlll] \arrow[dddll] \arrow[d, dotted] \\
            \myframe^{n-1,k\geq p} \arrow[d] \arrow[ddr] & \restrframe^{n,k\geq p} \arrow[d] \arrow[ddr] \arrow[l, dotted] & & frame^{n-1,p} \arrow[d] \arrow[ddr] & \restrframe^{n,k<p} \arrow[d] \arrow[ddr] \arrow[uul, dotted, shift right=2mm] \arrow[l, dotted]\\
            \textcolor{red}{painting^{n-2,k\geq p}} \arrow[d, dotted] & \textcolor{red}{restrpainting^{n-1,k\geq p}} \arrow[l, dotted] \arrow[d, dotted] & &
            \textcolor{red}{painting^{n-2,k<p}} \arrow[d, dotted] &
            \textcolor{red}{restrpainting^{n-1,k<p}} \arrow[l, dotted] \arrow[d, dotted]\\
            \textcolor{red}{frame^{n-2,k\geq p}} & \textcolor{red}{restrframe^{n-1,k\geq p},X_{n-2}} \arrow[l, dotted] & \textcolor{red}{cohframe^{n,k\geq p},X_{n-1}} \arrow[l, dotted] &
            \textcolor{red}{frame^{n-2,k<p}} &
            \textcolor{red}{restrframe^{n-1,k<p}} \arrow[l, dotted] \arrow[uul, dotted, shift right=2mm] &
            \textcolor{red}{cohframe^{n,k<p}} \arrow[l, dotted] \arrow[uul, dotted, shift right=2mm] \\
          \end{tikzcd}
        \end{tiny}

  \item To rely on $\restrpainting^{n-1,k}$ and $\cohframe^{n,k}$ for $k >
          p$ to also get $\restrpainting^{n,p}$ for all $p \leq n-1$:

        \begin{tiny}
          \begin{tikzcd}
            & & & painting^{n,p} \arrow[ddlll] \arrow[dddr] \arrow[d, dotted]\\
            & & & frame^{n,p} \arrow[d] \arrow[ddr]\\
            \painting^{n-1,k\geq p} \arrow[d, dotted] \arrow[dddr] \arrow[dd, bend right=70]  & & & painting^{n-1,p} \arrow[ddlll] \arrow[dddll] \arrow[d, dotted] & \restrpainting^{n,k<p} \arrow[d, dotted] \arrow[ddlll] \arrow[dddll] \arrow[l, dotted] \arrow[uul, dotted]\\
            \myframe^{n-1,k\geq p} \arrow[d] \arrow[ddr] & \restrframe^{n,k\geq p} \arrow[d] \arrow[ddr] \arrow[l, dotted] & &
            frame^{n-1,p} \arrow[d] \arrow[ddr] & \restrframe^{n,k<p} \arrow[d] \arrow[ddr] \arrow[uul, dotted, shift right=2mm] \arrow[l, dotted]\\
            \textcolor{red}{painting^{n-2,k\geq p}} \arrow[d, dotted] & \textcolor{red}{restrpainting^{n-1,k\geq p}} \arrow[d, dotted] & &
            \textcolor{red}{painting^{n-2,k<p}} \arrow[d, dotted] &
            \textcolor{red}{restrpainting^{n-1,k<p}} \arrow[l, dotted] \arrow[d, dotted]\\
            \textcolor{red}{frame^{n-2,k\geq p}} & \textcolor{red}{restrframe^{n-1,k\geq p},X_{n-2}} \arrow[l, dotted] & \textcolor{red}{cohframe^{n,k\geq p},X_{n-1}} \arrow[l, dotted] &
            \textcolor{red}{frame^{n-2,k<p}} &
            \textcolor{red}{restrframe^{n-1,k<p}} \arrow[l, dotted] \arrow[uul, dotted, shift right=2mm] &
            \textcolor{red}{cohframe^{n,k<p}} \arrow[l, dotted] \arrow[uul, dotted, shift right=2mm] \\
          \end{tikzcd}
        \end{tiny}

  \item To rely on $\myframe^{n,k}$ and $\painting^{n,k}$ for all $k < p$ to
        now construct $\myframe^{n+1,p}$ for $p \leq n+1$ as a function from
        all $\restrframe^{n+1,k}$ for $k \leq p$:

        \begin{tiny}
          \begin{tikzcd}
            & & & frame^{n+1,p}(\textcolor{red}{restrframe^{n+1,k<p}}) \arrow[d]\\
            & & & painting^{n,p} \arrow[ddlll] \arrow[dddr] \arrow[d, dotted]\\
            & & & frame^{n,p} \arrow[d] \arrow[ddr]\\
            \painting^{n-1,k\geq p} \arrow[d, dotted] \arrow[dddr] \arrow[dd, bend right=70]  & & & painting^{n-1,p} \arrow[ddlll] \arrow[dddll] \arrow[d, dotted] & \restrpainting^{n,k<p} \arrow[d, dotted] \arrow[ddlll] \arrow[dddll] \arrow[l, dotted] \arrow[uul, dotted]\\
            \myframe^{n-1,k\geq p} \arrow[d] \arrow[ddr] & \restrframe^{n,k\geq p} \arrow[d] \arrow[ddr] \arrow[l, dotted] & &
            frame^{n-1,p} \arrow[d] \arrow[ddr] & \restrframe^{n,k<p} \arrow[d] \arrow[ddr] \arrow[uul, dotted, shift right=2mm] \arrow[l, dotted]\\
            \textcolor{red}{painting^{n-2,k\geq p}} \arrow[d, dotted] & \textcolor{red}{restrpainting^{n-1,k\geq p}} \arrow[d, dotted] & &
            \textcolor{red}{painting^{n-2,k<p}} \arrow[d, dotted] &
            \textcolor{red}{restrpainting^{n-1,k<p}} \arrow[l, dotted] \arrow[d, dotted]\\
            \textcolor{red}{frame^{n-2,k\geq p}} & \textcolor{red}{restrframe^{n-1,k\geq p},X_{n-2}} \arrow[l, dotted] & \textcolor{red}{cohframe^{n,k\geq p},X_{n-1}} \arrow[l, dotted] &
            \textcolor{red}{frame^{n-2,k<p}} &
            \textcolor{red}{restrframe^{n-1,k<p}} \arrow[l, dotted] \arrow[uul, dotted, shift right=2mm] &
            \textcolor{red}{cohframe^{n,k<p}} \arrow[l, dotted] \arrow[uul, dotted, shift right=2mm] \\
          \end{tikzcd}
        \end{tiny}

  \item To rely on $\restrframe^{n,k}$ and $\restrpainting^{n,k}$ for all
        $k < p$ to now construct $\restrframe^{n+1,p}$ for $p \leq n$ as a
        function from all $\cohframe^{n+1,k}$ for $k \leq p$. This also
        allows to instantiate the dependency of $\myframe^{n+1,p}$ on
        $\restrframe^{n+1,p}$:

        \begin{tiny}
          \begin{tikzcd}
            & & & frame^{n+1,p} \arrow[d] \arrow[ddr] \\
            & & & painting^{n,p} \arrow[ddlll] \arrow[dddr] \arrow[d, dotted]\\
            & & & frame^{n,p} \arrow[d] \arrow[ddr] & \restrframe^{n+1,k<p}(\textcolor{red}{cohframe^{n+1,k<p}}) \arrow[d] \arrow[l, dotted] \arrow[uul, dotted, shift right=2mm] \\
            \painting^{n-1,k\geq p} \arrow[d, dotted] \arrow[dddr] \arrow[dd, bend right=70] & & &
            painting^{n-1,p} \arrow[ddlll] \arrow[dddll] \arrow[d, dotted] & \restrpainting^{n,k<p} \arrow[d, dotted] \arrow[ddlll] \arrow[dddll] \arrow[l, dotted] \arrow[uul, dotted]\\
            \myframe^{n-1,k\geq p} \arrow[d] \arrow[ddr] & \restrframe^{n,k\geq p} \arrow[d] \arrow[ddr] \arrow[l, dotted] & &
            frame^{n-1,p} \arrow[d] \arrow[ddr] & \restrframe^{n,k<p} \arrow[d] \arrow[ddr] \arrow[uul, dotted, shift right=2mm] \arrow[l, dotted]\\
            \textcolor{red}{painting^{n-2,k\geq p}} \arrow[d, dotted] & \textcolor{red}{restrpainting^{n-1,k\geq p}} \arrow[d, dotted] & &
            \textcolor{red}{painting^{n-2,k<p}} \arrow[d, dotted] &
            \textcolor{red}{restrpainting^{n-1,k<p}} \arrow[l, dotted] \arrow[d, dotted]\\
            \textcolor{red}{frame^{n-2,k\geq p}} & \textcolor{red}{restrframe^{n-1,k\geq p},X_{n-2}} \arrow[l, dotted] & \textcolor{red}{cohframe^{n,k\geq p},X_{n-1}} \arrow[l, dotted] &
            \textcolor{red}{frame^{n-2,k<p}} &
            \textcolor{red}{restrframe^{n-1,k<p}} \arrow[l, dotted] \arrow[uul, dotted, shift right=2mm] &
            \textcolor{red}{cohframe^{n,k<p}} \arrow[l, dotted] \arrow[uul, dotted, shift right=2mm] \\
          \end{tikzcd}
        \end{tiny}

  \item To additionally assume $\cohpainting^{n,k}$ for $k < p$ to get
        $\cohframe^{n+1,p}$ for all $p \leq n-1$. This also allows to
        instantiate the dependency of $\restrframe^{n+1,p}$ on
        $\cohframe^{n+1,p}$:

        \begin{tiny}
          \begin{tikzcd}
            & & & frame^{n+1,p} \arrow[d] \arrow[ddr] \\
            & & & painting^{n,p} \arrow[ddlll] \arrow[dddr] \arrow[d, dotted]\\
            & & & frame^{n,p} \arrow[d] \arrow[ddr] & \restrframe^{n+1,k<p} \arrow[d] \arrow[ddr] \arrow[l, dotted] \arrow[uul, dotted, shift right=2mm] \\
            \painting^{n-1,k\geq p} \arrow[d, dotted] \arrow[dddr] \arrow[dd, bend right=70] & & &
            painting^{n-1,p} \arrow[ddlll] \arrow[dddll] \arrow[d, dotted] & \restrpainting^{n,k<p} \arrow[d, dotted] \arrow[ddlll] \arrow[dddll] \arrow[l, dotted] \arrow[uul, dotted]\\
            \myframe^{n-1,k\geq p} \arrow[d] \arrow[ddr] & \restrframe^{n,k\geq p} \arrow[d] \arrow[ddr] \arrow[l, dotted] & &
            frame^{n-1,p} \arrow[d] \arrow[ddr] & \restrframe^{n,k<p} \arrow[d] \arrow[ddr] \arrow[l, dotted] \arrow[uul, dotted, shift right=2mm] &
            \cohframe^{n+1,p} \arrow[d] \arrow[uul, dotted, shift right=2mm] \arrow[l, dotted]\\
            \textcolor{red}{painting^{n-2,k\geq p}} \arrow[d, dotted] & \textcolor{red}{restrpainting^{n-1,k\geq p}} \arrow[d, dotted] & &
            \textcolor{red}{painting^{n-2,k<p}} \arrow[d, dotted] &
            \textcolor{red}{restrpainting^{n-1,k<p}} \arrow[l, dotted] \arrow[d, dotted] &
            \textcolor{red}{cohpainting^{n,k<p}} \arrow[l, dotted] \arrow[d, dotted] \\
            \textcolor{red}{frame^{n-2,k\geq p}} & \textcolor{red}{restrframe^{n-1,k\geq p},X_{n-2}} \arrow[l, dotted] & \textcolor{red}{cohframe^{n,k\geq p},X_{n-1}} \arrow[l, dotted] &
            \textcolor{red}{frame^{n-2,k<p}} &
            \textcolor{red}{restrframe^{n-1,k<p}} \arrow[l, dotted] \arrow[uul, dotted, shift right=2mm] &
            \textcolor{red}{cohframe^{n,k<p}} \arrow[l, dotted] \arrow[uul, dotted, shift right=2mm] \\
          \end{tikzcd}
        \end{tiny}

  \item To assume $\cohpainting^{n+1,k}$ for $k \geq p$ to build
        $\cohframe^{n+1,k\geq p}$:

        \begin{tiny}
          \begin{tikzcd}
            & & & frame^{n+1,p} \arrow[d] \arrow[ddr] \\
            & & & painting^{n,p} \arrow[ddlll] \arrow[dddr] \arrow[d, dotted]\\
            & & & frame^{n,p} \arrow[d] \arrow[ddr] & \restrframe^{n+1,k<p} \arrow[d] \arrow[ddr] \arrow[l, dotted] \arrow[uul, dotted, shift right=2mm] \\
            \painting^{n-1,k\geq p} \arrow[d, dotted] \arrow[dddr] \arrow[dd, bend right=70] & & &
            painting^{n-1,p} \arrow[ddlll] \arrow[dddll] \arrow[d, dotted] & \restrpainting^{n,k<p} \arrow[d, dotted] \arrow[ddlll] \arrow[dddll] \arrow[l, dotted] \arrow[uul, dotted]\\
            \myframe^{n-1,k\geq p} \arrow[d] \arrow[ddr] & \restrframe^{n,k\geq p} \arrow[d] \arrow[ddr] \arrow[l, dotted] & &
            frame^{n-1,p} \arrow[d] \arrow[ddr] & \restrframe^{n,k<p} \arrow[d] \arrow[ddr] \arrow[l, dotted] \arrow[uul, dotted, shift right=2mm] &
            \cohframe^{n+1,p} \arrow[d] \arrow[uul, dotted, shift right=2mm] \arrow[l, dotted]\\
            \textcolor{red}{painting^{n-2,k\geq p}} \arrow[d, dotted] & \textcolor{red}{restrpainting^{n-1,k\geq p}} \arrow[d, dotted] & &
            \textcolor{red}{painting^{n-2,k<p}} \arrow[d, dotted] &
            \textcolor{red}{restrpainting^{n-1,k<p}} \arrow[l, dotted] \arrow[d, dotted] &
            \textcolor{red}{cohpainting^{n,k<p}} \arrow[l, dotted] \arrow[d, dotted] \\
            \textcolor{red}{frame^{n-2,k\geq p}} & \textcolor{red}{restrframe^{n-1,k\geq p},X_{n-2}} \arrow[l, dotted] & \textcolor{red}{cohframe^{n,k\geq p},X_{n-1}} \arrow[l, dotted] &
            \textcolor{red}{frame^{n-2,k<p}} &
            \textcolor{red}{restrframe^{n-1,k<p}} \arrow[l, dotted] \arrow[uul, dotted, shift right=2mm] &
            \textcolor{red}{cohframe^{n,k<p}} \arrow[l, dotted] \arrow[uul, dotted, shift right=2mm] \\
          \end{tikzcd}
        \end{tiny}

  \item To rely on $\restrpainting^{n,k}$ and $\cohframe^{n+1,k}$ for $k \geq
          p$ to also get $\restrpainting^{n+1,p}$ for all $p \leq n$:

        \begin{tiny}
          \begin{tikzcd}
            & & & frame^{n+1,p} \arrow[d] \arrow[ddr] \\
            & & & painting^{n,p} \arrow[ddlll] \arrow[dddr] \arrow[d, dotted]\\
            & & & frame^{n,p} \arrow[d] \arrow[ddr] & \restrframe^{n+1,k<p} \arrow[d] \arrow[ddr] \arrow[l, dotted] \arrow[uul, dotted, shift right=2mm] \\
            \painting^{n-1,k\geq p} \arrow[d, dotted] \arrow[dddr] \arrow[dd, bend right=70] & & &
            painting^{n-1,p} \arrow[ddlll] \arrow[dddll] \arrow[d, dotted] & \restrpainting^{n,k<p} \arrow[d, dotted] \arrow[ddlll] \arrow[dddll] \arrow[l, dotted] \arrow[uul, dotted]\\
            \myframe^{n-1,k\geq p} \arrow[d] \arrow[ddr] & \restrframe^{n,k\geq p} \arrow[d] \arrow[ddr] \arrow[l, dotted] & &
            frame^{n-1,p} \arrow[d] \arrow[ddr] & \restrframe^{n,k<p} \arrow[d] \arrow[ddr] \arrow[l, dotted] \arrow[uul, dotted, shift right=2mm] &
            \cohframe^{n+1,p} \arrow[d] \arrow[uul, dotted, shift right=2mm] \arrow[l, dotted]\\
            \textcolor{red}{painting^{n-2,k\geq p}} \arrow[d, dotted] &
            \textcolor{red}{restrpainting^{n-1,k\geq p}} \arrow[l, dotted] \arrow[d, dotted] &
            \textcolor{red}{cohpainting^{n,k\geq p}} \arrow[l, dotted] \arrow[d, dotted] &
            \textcolor{red}{painting^{n-2,k<p}} \arrow[d, dotted] &
            \textcolor{red}{restrpainting^{n-1,k<p}} \arrow[l, dotted] \arrow[d, dotted] &
            \textcolor{red}{cohpainting^{n,k<p}} \arrow[l, dotted] \arrow[d, dotted] \\
            \textcolor{red}{frame^{n-2,k\geq p}} & \textcolor{red}{restrframe^{n-1,k\geq p},X_{n-2}} \arrow[l, dotted] & \textcolor{red}{cohframe^{n,k\geq p},X_{n-1}} \arrow[l, dotted] &
            \textcolor{red}{frame^{n-2,k<p}} &
            \textcolor{red}{restrframe^{n-1,k<p}} \arrow[l, dotted] \arrow[uul, dotted, shift right=2mm] &
            \textcolor{red}{cohframe^{n,k<p}} \arrow[l, dotted] \arrow[uul, dotted, shift right=2mm] \\
          \end{tikzcd}
        \end{tiny}

  \item To finally rely on $\cohpainting^{n,k}$ for $k \geq p$
        to also get $\cohpainting^{n+1,p}$ for $p \leq n-1$:

        \begin{tiny}
          \begin{tikzcd}
            & & & frame^{n+1,p} \arrow[d] \arrow[ddr] \\
            & & & painting^{n,p} \arrow[ddlll] \arrow[dddr] \arrow[d, dotted]\\
            & & & frame^{n,p} \arrow[d] \arrow[ddr] & \restrframe^{n+1,k<p} \arrow[d] \arrow[ddr] \arrow[l, dotted] \arrow[uul, dotted, shift right=2mm] \\
            \painting^{n-1,k\geq p} \arrow[d, dotted] \arrow[dddr] \arrow[dd, bend right=70] & & &
            painting^{n-1,p} \arrow[ddlll] \arrow[dddll] \arrow[d, dotted] & \restrpainting^{n,k<p} \arrow[d, dotted] \arrow[ddlll] \arrow[dddll] \arrow[l, dotted] &
            \cohpainting^{n+1,p} \arrow[ddlll] \arrow[d, dotted] \arrow[l, dotted] \\
            \myframe^{n-1,k\geq p} \arrow[d] \arrow[ddr] & \restrframe^{n,k\geq p} \arrow[d] \arrow[ddr] \arrow[l, dotted] & &
            frame^{n-1,p} \arrow[d] \arrow[ddr] & \restrframe^{n,k<p} \arrow[d] \arrow[ddr] \arrow[l, dotted] \arrow[uul, dotted, shift right=2mm] &
            \cohframe^{n+1,p} \arrow[d] \arrow[l, dotted] \arrow[uul, dotted, shift right=2mm]\\
            \textcolor{red}{painting^{n-2,k\geq p}} \arrow[d, dotted] &
            \textcolor{red}{restrpainting^{n-1,k\geq p}} \arrow[l, dotted] \arrow[d, dotted] &
            \textcolor{red}{cohpainting^{n,k\geq p}} \arrow[l, dotted] \arrow[d, dotted] &
            \textcolor{red}{painting^{n-2,k<p}} \arrow[d, dotted] &
            \textcolor{red}{restrpainting^{n-1,k<p}} \arrow[l, dotted] \arrow[d, dotted] &
            \textcolor{red}{cohpainting^{n,k<p}} \arrow[l, dotted] \arrow[d, dotted] \\
            \textcolor{red}{frame^{n-2,k\geq p}} &
            \textcolor{red}{restrframe^{n-1,k\geq p},X_{n-2}} \arrow[l, dotted] &
            \textcolor{red}{cohframe^{n,k\geq p},X_{n-1}} \arrow[l, dotted] &
            \textcolor{red}{frame^{n-2,k<p}} &
            \textcolor{red}{restrframe^{n-1,k<p}} \arrow[l, dotted] \arrow[uul, dotted, shift right=2mm] &
            \textcolor{red}{cohframe^{n,k<p}} \arrow[l, dotted] \arrow[uul, dotted, shift right=2mm] \\
          \end{tikzcd}
        \end{tiny}
\end{enumerate}

Thus, shortly, assuming $\myframe^{n-2,p}$, $\painting^{n-2,p}$,
$\restrframe^{n-1,p}$, $\restrpainting^{n-1,p}$, $\cohframe^{n,p}$ and
$\cohpainting^{n,p}$ for all $p \leq n-2$ allows to define
$\myframe^{n-1,p}$, $\painting^{n-1,p}$, $\restrframe^{n,p}$,
$\restrpainting^{n,p}$, $\cohframe^{n+1,p}$ and $\cohpainting^{n+1,p}$
for all $p \leq n-1$, so the construction can be recursively
continued.

\end{document}

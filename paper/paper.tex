\documentclass{lipics-v2021}

\usepackage{amsmath, amssymb, mathrsfs, tikz, tikz-cd, mathpazo, anyfontsize, xargs, environ, tabularx, booktabs, makecell, colortbl, float, art.cls/colorpal, art.cls/ct, art.cls/sset, art.cls/lim, art.cls/joinargs}
\usepackage[utf8]{inputenc}
\usepackage[OT1]{fontenc}
\usepackage[prefix=bonak]{art.cls/xkeymask}

% Use the patterns library to draw the cubes figure
\usetikzlibrary{patterns}

% Magic with xkeyval to go over the 9-argument limit of LaTeX
\makeatletter
\define@cmdkey[bonak]{X}{D}{(#1)}

\define@cmdkey[bonak]{frame}{D}{(#1)}
\define@cmdkey[bonak]{layer}{D}{(#1)}
\define@cmdkey[bonak]{layer}{d}{(#1)}
\define@cmdkey[bonak]{painting}{D}{(#1)}
\define@cmdkey[bonak]{painting}{E}{(#1)}
\define@cmdkey[bonak]{painting}{d}{(#1)}
\define@cmdkey[bonak]{fullframe}{D}{(#1)}

\define@cmdkey[bonak]{restrframe}{D}{(#1)}
\define@cmdkey[bonak]{restrframe}{d}{(#1)}
\define@cmdkey[bonak]{restrlayer}{D}{(#1)}
\define@cmdkey[bonak]{restrlayer}{d}{(#1)}
\define@cmdkey[bonak]{restrlayer}{l}{(#1)}
\define@cmdkey[bonak]{restrpainting}{D}{(#1)}
\define@cmdkey[bonak]{restrpainting}{E}{(#1)}
\define@cmdkey[bonak]{restrpainting}{d}{(#1)}
\define@cmdkey[bonak]{restrpainting}{c}{(#1)}

\define@cmdkey[bonak]{cohframe}{D}{(#1)}
\define@cmdkey[bonak]{cohframe}{d}{(#1)}
\define@cmdkey[bonak]{cohlayer}{D}{(#1)}
\define@cmdkey[bonak]{cohlayer}{d}{(#1)}
\define@cmdkey[bonak]{cohlayer}{l}{(#1)}
\define@cmdkey[bonak]{cohpainting}{D}{(#1)}
\define@cmdkey[bonak]{cohpainting}{E}{(#1)}
\define@cmdkey[bonak]{cohpainting}{d}{(#1)}
\define@cmdkey[bonak]{cohpainting}{c}{(#1)}
\makeatother

% Truncated sets, otherwise referred to as X: the general macro
\newcommandx{\X}[3][1,2,3]{
  \ensuremath{{\color{carolina}{\nu\textsf{Set}}}_{#1}^{#2}}
  \setkeys*[bonak]{X}{#3}
}

\newcommandx{\Xrefl}[3][1,2,3]{
  \ensuremath{{\color{carolina}{\nu\textsf{reflSet}}}_{#1}^{#2}}
  \setkeys*[bonak]{X}{#3}
}

% Frame, layer, and painting: the general macro
\newcommandx{\prim}[6][2,3,4,5,6]{
  \ensuremath{\mathsf{\color{indian-yellow}{#1}}_{#2}^{
    \joinargs[#3][#4][#5]}}
  \setkeys*[bonak]{#1}{#6}
}

% Restriction for frame, layer, and painting: the general macro
\newcommandx{\restr}[8][2,3,4,5,6,7,8]{
  \ensuremath{\mathsf{\color{russian-green}{restr}}_{
    \joinargs[\mathsf{\color{indian-yellow}{#1}}][#2][#3][#4]}^{\joinargs[#5][#6][#7]}}
  \setkeys*[bonak]{restr#1}{#8}
}

\newcommandx{\depsmacro}[5][4,5]{
  \ensuremath{\mathsf{\color{strawberry}{#1}}_{
    \joinargs[\mathsf{\color{#3}{#2}}]}^{\joinargs[#4][#5]}}
}

% Coherence conditions for frame, layer, and painting: the general macro

% Hack: we'd like \coh to take 10 arguments, with the 10th being passed to
% xkeyval, but this is not possible. The only possible compromise is to join
% #7 and #8 by hand in usage of \coh: it's now #7 or '#6,#7' in \cohfoo commands
\newcommandx{\coh}[9][2,3,4,5,6,7,8,9]{
  \ensuremath{\mathsf{\color{chestnut}{coh}}_{
    \joinargs[\mathsf{\color{indian-yellow}{#1}}][#2][#3][#4][#5][#6]}^{\joinargs[#7][#8]}}
  \setkeys*[bonak]{coh#1}{#9}
}

\newcommandx{\cohtwo}[9][2,3,4,5,6,7,8,9]{
  \ensuremath{\mathsf{\color{chestnut}{coh2}}_{
    \joinargs[\mathsf{\color{indian-yellow}{#1}}][#2][#3][#4][#5][#6]}^{\joinargs[#7][#8]}}
  \setkeys*[bonak]{coh#1}{#9}
}

% Simplicial and Cubical sets
\newcommand{\DeltaHat}{\ensuremath{\hat{\boldsymbol{\Delta}}}}
\newcommand{\DeltaPlus}{\ensuremath{\boldsymbol{\Delta}_+}}
\newcommand{\Cube}{\ensuremath{\boldsymbol{{\square}}}}
\newcommand{\CSet}{\ensuremath{\mathsf{Set}_{\boldsymbol{\square}}}}

% Universe, referred to as U
\newcommandx{\HSet}[1][1=]{\ensuremath{\mathsf{\color{spanish-blue}{HSet}}_{#1}}}
\newcommand{\HProp}{\ensuremath{\mathsf{\color{spanish-blue}{HProp}}}}
\newcommandx{\HGpd}[1][1=]{\ensuremath{\mathsf{\color{spanish-blue}{HGpd}}_{#1}}}
\newcommand{\Type}{\ensuremath{\mathsf{\color{spanish-blue}{Type}}}}
\newcommand{\SProp}{\ensuremath{\mathsf{\color{spanish-blue}{SProp}}}}

% The unit and bool types
\newcommand{\unittype}{\ensuremath{\mathsf{unit}}}
\newcommand{\booltype}{\ensuremath{\mathsf{bool}}}
\newcommand{\unitpoint}{\ensuremath{\ast}}

% Definitional and extensional equality
\newcommand{\defeq}{\ensuremath{\triangleq}}
\newcommand{\refl}{\ensuremath{\mathsf{refl}}}

% Some abbreviations
\newcommand{\tl}{\ensuremath{\mathsf{tl}}}
\newcommand{\hd}{\ensuremath{\mathsf{hd}}}
\newcommand{\imp}{\rightarrow}
\newcommand{\overright}[1]{\overrightarrow{#1}}
\newcommand{\ap}{\mathsf{ap}\;}
\renewcommand{\D}{D}
\newcommand{\hdD}{D.1}
\newcommand{\tlD}{D.2}
\renewcommand{\d}{d}
\renewcommand{\E}{E}
\renewcommand{\l}{l}
\renewcommand{\c}{c}
\newcommand{\pair}[2]{#1, #2}
\newcommand{\Dom}{\textsf{Dom}}
\newcommand{\UIP}{\textsf{UIP}}

\newcommand{\udensdash}[1]{%
    \tikz[baseline=(todotted.base)]{
        \node[inner sep=1pt,outer sep=0pt] (todotted) {$#1$};
        \draw[densely dashed] (todotted.south west) -- (todotted.south east);
    }%
}%

% Adjust font size
\newcommand{\smfontsize}{\fontsize{8}{11}\selectfont}

% These are renewed in tables, but let's define them now for in-document
% typesetting
\newcommandx{\framep}[2][1,2]{\prim{frame}[][#1][#2][][]}
\newcommandx{\layer}[2][1,2]{\prim{layer}[][#1][#2][][]}
\newcommandx{\painting}[2][1,2]{\prim{painting}[][#1][#2][][]}
\newcommandx{\restrf}[4][1,2,3,4]{\restr{frame}[][#3][#4][#1][#2][][]}
\newcommandx{\restrl}[4][1,2,3,4]{\restr{layer}[][#3][#4][#1][#2][][]}
\newcommandx{\restrp}[4][1,2,3,4]{\restr{painting}[][#3][#4][#1][#2][][]}
\newcommandx{\cohf}{\coh{frame}[][][][][][][][]}
\newcommandx{\cohl}{\coh{layer}[][][][][][][][]}
\newcommandx{\cohp}{\coh{painting}[][][][][][][][]}
\newcommandx{\fullframe}[1][1]{\prim{fullframe}[][#1][][][]}

\newcommandx{\frametype}[2][1,2]{\prim{FRAME}[][#1][#2][][]}
\newcommandx{\layertype}[2][1,2]{\prim{LAYER}[][#1][#2][][]}
\newcommandx{\paintingtype}[2][1,2]{\prim{PAINTING}[][#1][#2][][]}
\newcommandx{\restrftype}[2][1,2]{\restr{FRAME}[][][][#1][#2][][]}
\newcommandx{\restrltype}[2][1,2]{\restr{LAYER}[][][][#1][#2][][]}
\newcommandx{\restrptype}[2][1,2]{\restr{PAINTING}[][][][#1][#2][][]}
\newcommandx{\cohftype}[2][1,2]{\coh{FRAME}[][][][][][#1][#2][]}
\newcommandx{\cohltype}[2][1,2]{\coh{LAYER}[][][][][][#1][#2][]}
\newcommandx{\cohptype}[2][1,2]{\coh{PAINTING}[][][][][][#1][#2][]}
\newcommandx{\cohttype}[2][1,2]{\cohtwo{FRAME}[][][][][][#1][#2][]}

\newcommandx{\deps}[2][1,2]{\depsmacro{deps}{restr}{russian-green}[#1][#2]}
\newcommandx{\fulldeps}[1][1]{\depsmacro{deps}{fullrestr}{russian-green}[#1]}
\newcommandx{\depscohs}[2][1,2]{\depsmacro{deps}{coh}{chestnut}[#1][#2]}
\newcommandx{\fulldepscohs}[1][1]{\depsmacro{deps}{fullcoh}{chestnut}[#1]}
\newcommandx{\depscoht}[2][1,2]{\depsmacro{deps}{coh2}{chestnut}[#1][#2]}
\newcommandx{\fulldepscoht}[1][1]{\depsmacro{deps}{fullcoh2}{chestnut}[#1]}
\newcommandx{\depstype}[2][1,2]{\depsmacro{DEPS}{restr}{russian-green}[#1][#2]}
\newcommandx{\fulldepstype}[1][1]{\depsmacro{DEPS}{fullrestr}{russian-green}[#1]}
\newcommandx{\depscohstype}[2][1,2]{\depsmacro{DEPS}{coh}{chestnut}[#1][#2]}
\newcommandx{\fulldepscohstype}[1][1]{\depsmacro{DEPS}{fullcoh}{chestnut}[#1]}
\newcommandx{\depscohttype}[2][1,2]{\depsmacro{DEPS}{coh2}{chestnut}[#1][#2]}
\newcommandx{\fulldepscohttype}[1][1]{\depsmacro{DEPS}{fullcoh2}{chestnut}[#1]}

\newcommandx{\coht}[7][1,2,3,4,5,6,7]{\cohtwo{frame}[#3][#4][#5][#6][#7][#1][#2][]}

% Kerning for star
\newcommand{\kstar}{{\star}}

% Squelch warnings when using faLink
\newcommand{\linkicon}{\texorpdfstring{\faLink}{🔗}}

% The eqntable environment
\newcolumntype{Y}{>{\centering\arraybackslash}X}
\NewEnviron{eqntable}[1]{
  \begin{table}[H]
  \smfontsize
  \begin{tabularx}{\linewidth}{
    @{}
    >{$}l<{$}
    >{$}c<{$}
    >{$}c<{$}
    >{$}Y<{$}
    @{}}
    \toprule
    \BODY
    \bottomrule
  \end{tabularx}
  \caption{#1}
  \end{table}
}

% A block in the eqntable environment
\NewDocumentCommand{\eqnline}{m m m m}{#1 & #2 & #3 & #4 \\}
\newcommandx*{\mc}[1]{\multicolumn{4}{c}{\emph{#1}} \\\\}

% \eqnarg is slightly more involved: output ( ... ) if not in mask,
% and { ... } if in mask
\newcommandx*{\eqnarg}[3]{\ifinmask[bonak]{#1}[#2]{\{#2:#3\}}{(#2:#3)}}

\title{WIP: Degeneracies and permuatations}
\author{Hugo Herbelin}{Université Paris Cité, Inria, CNRS, IRIF, Paris}{Hugo.Herbelin@inria.fr}{}{}
\author{Ramkumar Ramachandra}{Unaffiliated}{r@artagnon.com}{}{}
\authorrunning{H. Herbelin and R. Ramachandra}
\keywords{Parametricity, Semi-simplicial sets, Semi-cubical sets, Formalization, Rocq, Indexed-fibred correspondence}
\ccsdesc{10003752.10003790.10011740}
\Copyright{Hugo Herbelin and Ramkumar Ramachandra}
%\nolinenumbers

\begin{document}
\maketitle
\begin{abstract}
  TODO
\end{abstract}

\section{Introduction}
% Abbreviated tables in this section
%
% The rule for implicit arguments for each table:
% show 1 explicit in first two parts
% show E + 1 explicit in the third part
\appendmask[bonak]{layer}[D]
\appendmask[bonak]{painting}[D]
\appendmask[bonak]{restrframe}[D]
\appendmask[bonak]{restrlayer}[D, d]
\appendmask[bonak]{restrpainting}[D, d]
\appendmask[bonak]{cohframe}[D]
\appendmask[bonak]{cohlayer}[D, d]
\appendmask[bonak]{cohpainting}[D, d]

% Truncated sets, otherwise referred to as X
% Keep all arguments
\newcommandx{\Xp}[1]{\X[#1][][]}
\newcommandx{\Xto}[3][3=]{\X[#1][<#2][#3]}
\newcommandx{\Xcomp}[3][3=]{\X[#1][=#2][#3]}
\newcommandx{\Xfrom}[3][3=]{\X[#1][\geq#2][#3]}


% Refl truncated sets.
\newcommandx{\Xreflp}[1]{\Xrefl[#1][][]}
\newcommandx{\Xreflto}[3][3=]{\Xrefl[#1][<#2][#3]}
\newcommandx{\Xreflcomp}[3][3=]{\Xrefl[#1][=#2][#3]}
\newcommandx{\Xreflfrom}[3][3=]{\Xrefl[#1][\geq#2][#3]}

% Frame, layer, and painting
% Drop the universe letter
\renewcommandx{\framep}[5][1,2,3,4,5]{\prim{frame}[][#2][#3][#4][#5]}
\renewcommandx{\layer}[5][1,2,3,4,5]{\prim{layer}[][#2][#3][#4][#5]}
\renewcommandx{\painting}[5][1,2,3,4,5]{\prim{painting}[][#2][#3][#4][#5]}

% Restriction for frame, layer, and painting
% Drop the universe letter
\renewcommandx{\restrf}[7][1,2,3,4,5,6,7]{\restr{frame}[][#2][#3][#4][#5][#6][#7]
}
\renewcommandx{\restrl}[7][1,2,3,4,5,6,7]{\restr{layer}[][#2][#3][#4][#5][#6][#7]
}
\renewcommandx{\restrp}[7][1,2,3,4,5,6,7]{\restr{painting}[][#2][#3][#4][#5][#6][
    #7]}

% Coherence conditions for frame, layer, and painting
% Keep only ε, ω; n, p; proposition on n and p
\renewcommandx{\cohf}[9][1,2,3,4,5,6,7,8,9]{\coh{frame}[][#2][#3][][][][][#9]}
\renewcommandx{\cohl}[9][1,2,3,4,5,6,7,8,9]{\coh{layer}[][#2][#3][][][][][#9]}
\renewcommandx{\cohp}[9][1,2,3,4,5,6,7,8,9]{\coh{painting}[][#2][#3][][][][][#9]}

\def\lab{tab:coind}
\providecommand{\lab}{tab:coind}
\begin{eqntable}{Main definition\label{\lab}}
  \eqnline{\Xrefl}{\eqnarg{X}{D}{\Xto{n}}}{:}{\HSet[m+1]}
  \eqnline{\Xrefl}{X}{\defeq}{\Xreflfrom{0}[D=\unitpoint]}
  \midrule
  \eqnline{\Xreflfrom{n}}{\eqnarg{X}{D}{\Xreflto{n}}}{:}{\HSet[m+1]}
  \eqnline{\Xreflfrom{n}}{D}{\defeq}{\Sigma E:\Xreflcomp{n}[D=\D]. \Xreflfrom{n+1}[D=\pair{D}{E}]}
\end{eqntable}


% For just the second table, keep the universe letter in fullframe
\renewcommandx{\fullframe}[3][1,2,3]{\prim{fullframe}[#1][#2][][][#3]}

\def\lab{tab:core}
\providecommand{\lab}{tab:core}
\begin{eqntable}{Truncated $\nu$-sets, the core\label{\lab}}
  \eqnline{\Xreflto{n}}{\eqnarg{X}{D}{\Xto{n}}}{:}{\HSet[m+1]}
  \eqnline{\Xreflto{0}}{\unitpoint}{\defeq}{\unittype}
  \eqnline{\Xreflto{n+1}}{(D,E)}{\defeq}{\ensuremath{\Sigma}L:\Xreflto{n}[D=\D].\,\Xreflcomp{n}[L=\L]}
  \midrule
  \eqnline{\Xreflcomp{n}}{\makecell{\eqnarg{X}{D}{\Xto{n}} \\ \eqnarg{X}{L}{\Xreflto{n}[D=\D]}}}{:}{\HSet[m+1]}
  \eqnline{\Xreflcomp{n}}{((\D,\E,\E'),L)}{\defeq}{\Pi d:\fullframe[n](\D).\,\Pi c:\E(d).\,\E'(\reflfext[n][p][d=\d, L=\L],\lambda \epsilon.\,c)}
\end{eqntable}


\def\lab{tab:frames}
\providecommand{\lab}{tab:frames}
\begin{eqntable}{$q$-th degeneracies\label{\lab}}
  \eqnline{\reflfext[n][p][q]}{\makecell{\eqnarg{fullframe}{D}{\Xto{n+1}} \\ \eqnarg{fullframe}{L}{\Xreflto{n}(D)} \\ \eqnarg{fullframe}{d}{\framep[n][p]} \\ \eqnarg{fullframe}{c}{\painting[n][p](d)}}}{:}{\fullframe[n+1](\D)}
  \eqnline{\reflfext[n][0][q]}{DLdc}{\defeq}{(\reflf[n][p][q][D=\D](d),\lambda \epsilon.\,c)}
  \eqnline{\reflfext[n][p+1][q]}{DL(d,l)c}{\defeq}{\reflfext[n][p][q+1][D=\D](d)(l,c)}
  \midrule
  \eqnline{\reflf[n][p][q][p \leq n]}{\makecell{\eqnarg{frame}{D}{\Xreflto{n}} \\ \eqnarg{fullframe}{L}{\Xreflto{n}(D)} \\ \eqnarg{frame}{d}{\framep[n][p]}}}{:}{\framep[n+1][p]}
  \eqnline{\reflf[n][0][q]}{D L \unitpoint}{\defeq}{\unitpoint}
  \eqnline{\reflf[n][p+1][q]}{D L (d,l)}{\defeq}{(\reflf[n][p][q+1][][D=\D, d=\d],\refll[n-1][p][][D=\D, d=\d, l=\l])}
  \midrule
  \eqnline{\refll[n][p][p \leq n]}{\makecell{\eqnarg{refll}{D}{\Xreflto{n+1}} \\ \eqnarg{fullframe}{L}{\Xreflto{n}(D)} \\ \eqnarg{refll}{d}{\reflf[n+1][p][][D=\D]} \\ \eqnarg{refll}{l}{\layer[n][p][][D=\D, d=\d]}}}{:}{\layer[n+1][p][][D=\hdD, d={\reflf[n+1][p][][q+1][D=\D, d=\d]}]}
  \eqnline{\refll[n][p]}{D~d}{\defeq}{\lambda \omega.(\reflp[n][p][][q][D=\hdD, E=\tlD, d={\reflf[n][p][][0][\omega][D=\D, d=\d]}, c={\l_\omega}])}
  \midrule
  \eqnline{\reflp[n][p][p \leq n]}{\makecell{\eqnarg{filler}{D}{\Xreflto{n}} \\ \eqnarg{filler}{E}{\Xreflcomp{n}[D=\D]} \\ \eqnarg{fullframe}{L}{\Xreflto{n}(D)} \\ \eqnarg{fullframe}{R}{\Xreflcomp{n}(D)(L)} \\ \eqnarg{filler}{d}{\reflf[n][p][][D=\D]} \\ \eqnarg{fullframe}{c}{\painting[n][p](d)}}}{:}{\painting[n+1][p][][D=\hdD, E=\tlD, d={\reflf[n][p][][q][D=\D, d=\d]}]}
  \eqnline{\reflp[n][p][p=n]}{\D~\E~L~R~\d~c}{\defeq}{R(\d)(c)}
  \eqnline{\reflp[n][p][p<n]}{\D~\E~\d}{\defeq}{(\refll[n-1][p][][q][D=\D, d=\d, l=\l],\reflp[n][p+1][][q][D=\D, E=\E, d=\pair{d}{l}, c=\c])}
\end{eqntable}


% The third table mentions cohframe; keep ε, ω in this instance
\renewcommandx{\cohf}[9][1,2,3,4,5,6,7,8,9]{\coh{frame}[][#2][#3][][][#6,#7][#8][#9]}

\def\lab{tab:faces}
\providecommand{\lab}{fulltab:faces}
\begin{eqntable}{$q$-th projection of $\mathsf{restr}$, or faces\label{\lab}}
  \eqnline{\restrf[m][q][\epsilon][n][p][q \leq n-p]}{\makecell{\eqnarg{restrframe}{D}{\Xreflto{m}{n+1}} \\ \eqnarg{restrframe}{d}{\framep[m][n+1][p][][D=\D]}}}{:}{\framep[m][n][p][][D=\hdD]}
  \eqnline{\restrf[m][q][\epsilon][n][0]}{D~\unitpoint}{\defeq}{\unitpoint}
  \eqnline{\restrf[m][q][\epsilon][n][p+1]}{D~(\pair{d}{l})}{\defeq}{(\restrf[m][q+1][\epsilon][n][p][][D=\D, d=\d],\restrl[m][q][\epsilon][n-1][p][][D=\D, d=\d, l=\l])}
  \midrule
  \eqnline{\restrl[m][q][\epsilon][n][p][q \leq n-p]}{\makecell{\eqnarg{restrlayer}{D}{\Xreflto{m}{n+2}} \\ \eqnarg{restrlayer}{d}{\framep[m][n+2][p][][D=\D]} \\ \eqnarg{restrlayer}{l}{\layer[m][n+1][p][][D=\D, d=\d]}}}{:}{\layer[m][n][p][][D=\hdD, d={\restrf[m][q+1][\epsilon][n+1][p][][D=\D, d=\d]}]}
  \eqnline{\restrl[m][q][\epsilon][n][p]}{\D~\d~\l}{\defeq}{\lambda \omega.(\restrp[m][q][\epsilon][n][p][][D=\hdD, E=\tlD, d={\restrf[m][\omega][0][n][p][][D=\D, d=\d]}, c={\l_\omega}])}
  \midrule
  \eqnline{\restrp[m][q][\epsilon][n][p][q \leq n-p]}{\makecell{\eqnarg{restrpainting}{D}{\Xreflto{m}{n+1}} \\ \eqnarg{restrpainting}{E}{\Xreflcomp{m}{n+1}[D=\D]} \\\eqnarg{restrpainting}{d}{\framep[m][n+1][p][][D=\D]} \\ \eqnarg{restrpainting}{c}{\painting[m][n+1][p][][D=\D, E=\E, d=\d]}}}{:}{\painting[m][n][p][][D=\hdD, E=\tlD, d={\restrf[m][q][\epsilon][n][p][][D=\D, d=\d]}]}
  \eqnline{\restrp[m][0][\epsilon][n][p][]}{\D~\E~\d~(\pair{l}{\_})}{\defeq}{\l_\epsilon}
  \eqnline{\restrp[m][q+1][\epsilon][n][p][p<n]}{\D~\E~\d~(\pair{l}{c})}{\defeq}{(\restrl[m][q][\epsilon][n-1][p][][D=\D, d=\d, l=\l],\restrp[m][q][\epsilon][n][p+1][][D=\D, E=\E, d=\pair{d}{l}, c=\c])}
\end{eqntable}


% Restriction for frame, layer, and painting
% Drop just the universe letter for the coh table
\renewcommandx{\restrf}[7][1,2,3,4,5,6,7]{\restr{frame}[][#2][#3][#4][#5][#6][#7]}
\renewcommandx{\restrl}[7][1,2,3,4,5,6,7]{\restr{layer}[][#2][#3][#4][#5][#6][#7]}
\renewcommandx{\restrp}[7][1,2,3,4,5,6,7]{\restr{painting}[][#2][#3][#4][#5][#6][#7]}

% Coherence conditions for frame, layer, and painting
\renewcommandx{\cohf}[9][1,2,3,4,5,6,7,8,9]{\coh{frame}[][#2][#3][#4][#5][#6,#7][#8][#9]}
\renewcommandx{\cohl}[9][1,2,3,4,5,6,7,8,9]{\coh{layer}[][#2][#3][#4][#5][#6,#7][#8][#9]}
\renewcommandx{\cohp}[9][1,2,3,4,5,6,7,8,9]{\coh{painting}[][#2][#3][#4][#5][#6,#7][#8][#9]}

\def\lab{tab:coh}
\providecommand{\lab}{fulltab:coh}
\begin{eqntable}{Commutation of $q$-th projection and $r$-th projection, or coherence conditions\label{\lab}}
  \eqnline{\cohf[n][p][r \leq q \leq n-p][q][r][\epsilon][\omega]}{\makecell{\eqnarg{cohframe}{D}{\Xreflto{n+2}} \\ \eqnarg{cohframe}{d}{\framep[n+2][p][][D=\D]}}}{:}{\makecell{\restrf[q][\epsilon][n][p][][D=\hdD, d={\restrf[r][\omega][n+1][p][][D=\D, d=\d]}] \\ = \restrf[r][\omega][n][p][][D=\hdD, d={\restrf[q+1][\epsilon][n+1][p][][D=\D, d=\d]}]}}
  \eqnline{\cohf[n][0][][q][r][\epsilon][\omega]}{\D~\unitpoint}{\defeq}{\refl(\unitpoint)}
  \eqnline{\cohf[n][p+1][][q][r][\epsilon][\omega]}{\D~(\pair{\d}{\l})}{\defeq}{(\cohf[n][p][][q+1][r+1][\epsilon][\omega][D=\D, d=\d], \cohl[n-1][p][][q][r][\epsilon][\omega][D=\D, d=\d, l=\l])}
  \midrule
  \eqnline{\cohl[n][p][r \leq q \leq n-p][\epsilon][\omega][q][r]}{\makecell{\eqnarg{cohlayer}{D}{\Xreflto{n+3}} \\ \eqnarg{cohlayer}{d}{\framep[n+3][p][][D=\D]} \\ \eqnarg{cohlayer}{l}{\layer[n+2][p][][D=\D, d=\d]}}}{:}{\makecell{\restrl[q][\epsilon][n][p][][D=\hdD, d={\restrf[r][\omega][n+1][p][][D=\D, d=\d]}](\restrl[n+1][p][][r][\omega][D=\D, d=\d, l=\l]) \\ = \restrl[r][\omega][n][p][][D=\hdD, d={\restrf[n+1][p][][q+1][][\epsilon][D=\D, d=\d]}, l={\restrl[q+1][\epsilon][n+1][p][][D=\D, d=\d, l=\l]}]}}
  \eqnline{\cohl[n][p][][q][r][\epsilon][\omega]}{\D~\d~\l}{\defeq}{\makecell{\lambda \theta .\;\cohp[n][p][][q][r][\epsilon][\omega][E=\tlD, c=\l_\theta]}}
  \midrule
  \eqnline{\cohp[n][p][r \leq q \leq n-p][q][r][\epsilon][\omega]}{\makecell{\eqnarg{cohpainting}{D}{\Xreflto{n+2}} \\ \eqnarg{cohpainting}{E}{\Xreflcomp{n+2}[D=\D]} \\ \eqnarg{cohpainting}{d}{\framep[n+2][p][][D=\D]} \\ \eqnarg{cohpainting}{c}{\painting[n+2][p][][D=\D, E=\E, d=\d]}}}{:}{\makecell{\restrp[n][p][][q][\epsilon][D=\hdD, E=\tlD, d={\restrf[r][\omega][n+1][p][][D=\D, d=\d]}, c={\restrp[n+1][p][][r][\omega][D=\D, E=\E, d=\d, c=\c]}] \\ = \restrp[n][p][][r][\omega][D=\hdD, E=\tlD, D={\restrf[q+1][\epsilon][n+1][p][][D=\D, d=\d]}, c={\restrp[n+1][p][][q+1][\epsilon][D=\D, E=\E, d=\d, c=\c]}]}}
  \eqnline{\cohp[n][p][][q][0][\epsilon][\omega]}{\D~\E~\d~(\pair{\l}{\_})}{\defeq}{\refl(\restrp[n][p][][q][\epsilon][D=\hdD, E=\tlD, d={\restrf[\omega][0][n][p][][D=\D, d=\d]}, c={\l_\epsilon}])}
  \eqnline{\cohp[n][p][p<n][q+1][r+1][\epsilon][\omega]}{\D~\E~\d~(\pair{\l}{\c})}{\defeq}{(\pair{\cohl[n-1][p][][q][r][\epsilon][\omega][D=\D, d=\d, l=\l]}{\cohp[n][p+1][][q][r][\epsilon][\omega][D=\D, E=\E, d=\pair{d}{l}, c=\c]})}
\end{eqntable}


\renewcommandx{\cohf}{\coh{frame}[][][][][][][][]}
\renewcommandx{\cohl}{\coh{layer}[][][][][][][][]}
\renewcommandx{\cohp}{\coh{painting}[][][][][][][][]}

\section{Degeneracies}

\begin{equation*}
  \begin{array}{cllccllll}
    \multicolumn{3}{c}{\mbox{\emph{fibred form}}} & \quad\mathit{vs}\quad & \multicolumn{3}{l}{\mbox{\emph{indexed form}}}                                                                                                                                                                                                               \\
    \\
    Y_0                                           & :                     & \HSet[l]                                       &  & X_0                                                                                & :                & \HSet[l]                                                                     & \mbox{(points)}   \\
    \uparrow\uparrow\!\downarrow                                                                                                                                                                                                                                                                                                         \\
    Y_1                                           & :                     & \HSet[l]                                       &  & X_1                                                                                & :                & X_0 \times X_0 \rightarrow \HSet[l]                                          & \mbox{(segments)} \\
    \uparrow\uparrow\uparrow\uparrow\!\downarrow  &                       &                                                &  & r_0                                                                                & :                & \Pi x_0:X_0.\,X_1(x_0,x_0)                                                                       \\
    Y_2                                           & :                     & \HSet[l]                                       &  & X_2                                                                                & :                & \Pi (x^0_{LL},x^0_{LR}).\, \Pi x^1_{L*}:X_1 (x^0_{LL},x^0_{LR}).                                 \\
                                                  &                       &                                                &  &                                                                                    &                  & \Pi (x^0_{RL},x^0_{RR}).\, \Pi x^1_{R*}:X_1 (x^0_{RL},x^0_{RR}).\,                               \\
    \multicolumn{3}{c}{\mbox{~+ coherences}}      &                       &                                                &  & X_1 (x^0_{LL},x^0_{RL}) \times X_1 (x^0_{LR},x^0_{RR}) \rightarrow \HSet[l] \qquad & \mbox{(squares)}                                                                                                    \\
                                                  &                       &                                                &  & r_1                                                                                & :                & \Pi (x^0_{L},x^0_{R}):(X_0 \times X_0).\,\Pi x^1:X_1(x^0_L,x^0_R).\,                             \\
                                                  &                       &                                                &  &                                                                                    &                  & X_2((x^0_{L},x^0_{L}),r_0(x^0_{L}),(x^0_{R},x^0_{R}),r_0(x^0_{R}),(x^1,x^1))                     \\
    \vdots                                        &                       &                                                &  & \vdots                                                                                                                                                                                                   \\
  \end{array}
\end{equation*}

The added degeneracy is \emph{parametric}: in the binary case, it gives a standard cubical degeneracy; in the unary case, it gives a ParamTT-like degeneracy and \emph{not} a simplicial degeneracy.

First, our degeneracy implies a distinguished point $r_{-1}(a)$ for any $a:X_{-1}$. Then:

\begin{equation*}
  \begin{array}{lcccr}
    \begin{tabular}{l}source\\ (over some $a:X_{-1}$)\end{tabular}       & b~~                                                               & \begin{tikzcd}b \arrow[r, dash,"q"] & c\end{tikzcd} \\[1mm]
    \begin{tabular}{l}parametric\\ degeneracy\end{tabular}               & \begin{tikzcd}b \arrow[r, dash, "r_0(b)"] & r_{-1}(a)\end{tikzcd} &
    \begin{tikzcd}
      & |[alias=F]|r_{-1}(a) \arrow[ddr, dash, "r_0(c)"] & \\\\
      b \arrow[rr, dash, "q"{name=T, below}]\arrow[uur, dash, "r_0(b)"] && c \\
      \arrow[rightarrow, from=F, to=T, phantom, "r_1(q)" description]
    \end{tikzcd}                                                                                                                         \\[-8mm]
    \begin{tabular}{l}simplicial\\ degeneracy\end{tabular}               & \begin{tikzcd}b \arrow[r, dash, "s_0(b)"] & b\end{tikzcd}         &
    \begin{tikzcd}
      & |[alias=F]|c \arrow[ddr, dash, "s_0(c)"] & \\\\
      b \arrow[rr, dash, "q"{name=T, below}]\arrow[uur, dash, "q"] && c \\
      \arrow[rightarrow, from=F, to=T, phantom, "s_1(q)" description]
    \end{tikzcd} & \begin{tabular}{l}actually\\ a 1-connection!\end{tabular}
  \end{array}
\end{equation*}

\section*{Adding a degeneracy}

For any $(X_0,X_1,...): \X$, we define a stream of degeneracies:
\begin{equation*}
  \begin{array}{lll}
    \Xrefl(X_0,X_1,...) ~~ \defeq                                                                                                                                                                                                                            \\
    \quad \Sigma r_{0}:\Pi d:\mathsf{\color{indian-yellow}{fullframe}}^{0}.\,\Pi x:\textcolor{red}{X_0}(d).\,\textcolor{red}{X_1}(\mathsf{\color{russian-green}{refl}}_{\mathsf{\color{indian-yellow}{fullframe}}}^0(d),\lambda \epsilon.\,x).               \\
    \quad \Sigma r_1:\Pi d:\mathsf{\color{indian-yellow}{fullframe}}^1(X_0).\,\Pi x:\textcolor{red}{X_1}(d).\,\textcolor{red}{X_2}(\mathsf{\color{russian-green}{refl}}_{\mathsf{\color{indian-yellow}{fullframe}}}^1(r_{0})(d),\lambda \epsilon.\,x).       \\
    \quad \Sigma r_2:\Pi d:\mathsf{\color{indian-yellow}{fullframe}}^2(X_0,X_1).\,\Pi x:\textcolor{red}{X_2}(d).\,\textcolor{red}{X_3}(\mathsf{\color{russian-green}{refl}}_{\mathsf{\color{indian-yellow}{fullframe}}}^2(r_0,r_1)(d),\lambda \epsilon.\,x). \\
    \quad ...
  \end{array}
\end{equation*}
where
\begin{equation*}
  \mathsf{\color{russian-green}{refl}}_{\mathsf{\color{indian-yellow}{fullframe}}}^n(r_{-1},...,r_{n-1}):\mathsf{\color{indian-yellow}{fullframe}}^n(X_{-1},...,X_{n-1}) \rightarrow \mathsf{\color{indian-yellow}{frame}}^{n+1,n}(X_{-1},...,X_{n})
\end{equation*}
computes the $n$ first layers of the border of $r_{n}(d)(x)$, knowing that the last layer is made of $\nu$ times $x$ itself, so that

\begin{equation*}
  (\mathsf{\color{russian-green}{refl}}_{\mathsf{\color{indian-yellow}{fullframe}}}^n(r_{-1},...,r_{n-1})(d),\lambda \epsilon.\,x) : \mathsf{\color{indian-yellow}{frame}}^{n+1,n+1}(X_{-1},...,X_{n})
\end{equation*}
is a full frame.

On the way, we need two coherence conditions:
\begin{equation*}
  \begin{array}{l}
    \mathsf{\textcolor{chestnut}{idrestrrefl}}_{\textcolor{indian-yellow}{\mathsf{frame}},\epsilon}^n(r_{-1},...,r_{n-1})~(d:\mathsf{\color{indian-yellow}{fullframe}}^n(X_0,...,X_{n-1})):                          \\
    ~~ \mathsf{\color{russian-green}{restr}}_{\mathsf{\color{indian-yellow}{frame}},\epsilon,n}^{n,n}(\mathsf{\color{russian-green}{refl}}_{\mathsf{\color{indian-yellow}{fullframe}}}^n(r_{-1},...,r_{n-1})(d)) = d \\
    \\
    \mathsf{\textcolor{chestnut}{cohrestrrefl}}_{\textcolor{indian-yellow}{\mathsf{frame}},\epsilon,p<n}^n(r_{-1},...,r_{n-1})~(d:\mathsf{\color{indian-yellow}{frame}}^{n,p}(X_0,...,X_{n-1})):                     \\
    ~~\;\mathsf{\color{russian-green}{restr}}_{\mathsf{\color{indian-yellow}{frame}},\epsilon,p}^{n,p}(\mathsf{\color{russian-green}{refl}}^{n,p}_{\mathsf{\color{indian-yellow}{frame}}}(r_{-1},...,r_{n-1})(d)) = \mathsf{\color{russian-green}{refl}}^{n-1,p}_{\mathsf{\color{indian-yellow}{frame}}}(r_{-1},...,r_{n-2})(\mathsf{\color{russian-green}{restr}}_{\mathsf{\color{indian-yellow}{frame}},\epsilon,p}^{n-1,p}(d))
  \end{array}
\end{equation*}
where
$\mathsf{\color{russian-green}{refl}}^{n,p}_{\mathsf{\color{indian-yellow}{frame}}}$ generalises $\mathsf{\color{russian-green}{refl}}_{\mathsf{\color{indian-yellow}{fullframe}}}^{n}$
to prefixes of
$\mathsf{\color{indian-yellow}{fullframe}}^n$:

\begin{equation*}
  \mathsf{\color{russian-green}{refl}}^{n,p}_{\mathsf{\color{indian-yellow}{frame}}}(r_{-1},...,r_{n-1}):\mathsf{\color{indian-yellow}{frame}}^{n,p}(X_{-1},...,X_{n-1}) \rightarrow \mathsf{\color{indian-yellow}{frame}}^{n+1,p}(X_{-1},...,X_{n})
\end{equation*}

\bibliographystyle{plainurl}
\bibliography{paper}

\end{document}

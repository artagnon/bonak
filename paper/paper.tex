\documentclass[a4paper]{article}

\usepackage{hyperref, fullpage, amssymb, stmaryrd, url, color, amsmath}
\usepackage{bookmark, fontspec}
\usepackage[backend=biber, style=apa, natbib=true, sorting=nyt, url=false,isbn=false,doi=false]{biblatex}

% Add biblio-en.bib to the database
\addbibresource{biblio-en.bib}

\makeatletter
\newcommand{\Overrightarrow}[2]{\raisebox{#1}{$\ext@arrow 0359\Rightarrowfill@{\mbox{${#2}$}}{}$}}
\newcommand{\Overleftarrow}[2]{\raisebox{#1}{$\ext@arrow 0359\Leftarrowfill@{\mbox{${#2}$}}{}$}}
\makeatother

\newcommand{\highlight}[1]{\emph{#1}}

%%%%%%%%%%%%%% Environnement global %%%%%%%%%

\def\proofskip{\vskip 5pt}
\def\proofbox{\hfill\rule{6pt}{6pt}}
\newcommand{\TODO}[1]{}
\newcommand{\omitnow}[1]{}
\newcommand{\omitted}[1]{}
\newcommand{\replaced}[2]{#1}
\newcommand\postsoumission[1]{} %{#1}

\newenvironment{preuve}{{\noindent \sc Preuve:\/}~}{\proofbox\proofskip}
\newenvironment{defi}{\medskip}{\proofskip}

\newenvironment{petit}{\begin{footnotesize}}{\end{footnotesize}}

\def\remarque{\noindent{\bf Remarque}}
\def\remarques{\noindent{\bf Remarques}}
\def\remark{\noindent{\bf Remark}}
\def\remarks{\noindent{\bf Remarks}}
\def\fdefinition{{\bf D\'efinition}}
\def\fdefinitions{{\bf D\'efinitions}}

% Bilingue
\newtheorem{prop}{Proposition}
\newtheorem{conj}{Conjecture}

% Francais
\newtheorem{ppte}{Propri\'et\'e}
\newtheorem{theoreme}{Th\'eor\`eme}
\newtheorem{lemme}{Lemme}
\newtheorem{corollaire}{Corollaire}
\newtheorem{axiome}{Axiome}

% English
\newtheorem{ppty}{Property}
\newtheorem{theorem}{Theorem}
\newtheorem{lemma}{Lemma}
\newtheorem{corollary}{Corollary}
\newtheorem{axiom}{Axiom}

%%%%%%%%%%%%%% Les alias %%%%%%%%%%%%%%%

%%%% Connecteurs

\def\IMP{\rightarrow}
\def\arrow{\IMP}
\def\AND{\land}
\def\OR{\lor}
\def\BOT{\bot}
\def\DUAL{\bot}
\def\NOT{\neg}

%%%% Les reductions courantes

\newcommand{\red}{\rightarrow}
\newcommand{\invred}{\leftarrow}
\newcommand{\redh}{\stackrel{h}{\rightarrow}}
\newcommand{\redex}{\stackrel{r}{\rightarrow}}
\newcommand{\redun}{\stackrel{1}{\rightarrow}}
\newcommand{\redstar}{\stackrel{*}{\rightarrow}}
\newcommand{\invredstar}{\stackrel{*}{\leftarrow}}

%%%% Les traductions

\newcommand{\sem}[1]{[\![#1]\!]}
\newcommand{\trad}[1]{[\![#1]\!]}
\newcommand{\embedlr}[1]{#1^{\mbox{\tiny $>$}}}
\newcommand{\embedparlr}[1]{(#1)^{\mbox{\tiny $>$}}}
\newcommand{\embedrl}[1]{#1^{\mbox{\tiny $<$}}}
\newcommand{\embedparrl}[1]{(#1)^{\mbox{\tiny $<$}}}
\newcommand{\isocbn}[1]{#1^{\sf N}}   %{\cal N}
\newcommand{\isocbnpar}[1]{(#1)^{\sf N}}
\newcommand{\isocbncom}[2]{#2_{#1}^{\sf N}}
\newcommand{\isocbncompar}[2]{(#2)_{#1}^{\sf N}}
\newcommand{\isocbnrev}[1]{#1^{{\sf N}^{-1}}}
\newcommand{\isocbnrevpar}[1]{(#1)^{{\sf N}^{-1}}}
\newcommand{\isocbnrevcom}[2]{#1_{#2}^{{\sf N}^{-1}}}
\newcommand{\isocbnrevcompar}[2]{(#1)_{#2}^{{\sf N}^{-1}}}

\newcommand{\isocbv}[1]{#1^{\sf V}}   %{\cal V}
\newcommand{\isocbvpar}[1]{(#1)^{\sf V}}
\newcommand{\isocbvcom}[2]{{#2}_{#1}^{\sf V}}
\newcommand{\isocbvcompar}[2]{(#2)_{#1}^{\sf V}}
\newcommand{\isocbvrev}[1]{#1^{{\sf V}^{-1}}}
\newcommand{\isocbvrevpar}[1]{(#1)^{{\sf V}^{-1}}}
\newcommand{\isocbvrevcom}[2]{{#1}_{#2}^{{\sf V}^{-1}}}
\newcommand{\isocbvrevcompar}[2]{(#1)_{#2}^{{\sf V}^{-1}}}

\newcommand{\ccl}[1]{{\mathsf{ccl}}(#1)}

%%%%%%%% lambda-bar calcul %%%%%%

\def\LSP{\overline{\lambda}}
\def\LSPMU{\overline{\lambda}\mu}
\def\LSPC{\overline{\lambda}_C}
\def\LMT{\mbox{LMT}}
\def\LJT{\mbox{LJT}}
\def\LKT{\mbox{LKT}}
\def\LM{\mbox{LM}}
\def\LK{\mbox{LK}}
\def\LJ{\mbox{LJ}}

\def\FORALL{\forall}
\def\EXISTS{\exists}
\def\FORALLL{\FORALL_g}
\def\EXISTSL{\EXISTS_g}
\def\FORALLR{\FORALL_d}
\def\EXISTSR{\EXISTS_d}

\def\redh{\stackrel{h}{\rightarrow}}
\def\redplus{\stackrel{E}{\rightarrow}}
\def\ANYRED{\stackrel{R}{\rightarrow}}

\def\EQUIV{\!::=}
\def\DP{\!:\!}
\def\VEC{\overrightarrow}

\def\CONS{::}
\def\ARO{\;@\;}
\def\BEC{:=}
\def\LIFT{\!\uparrow}

\def\CASE{\mbox{\it case}}
\def\OF{\mbox{\it of}}
\def\FLECHE{\;-\!\!\!>}

\def\POINT{.\;}
\def\REDLJT{\stackrel{\LJT}{\longrightarrow}}
\def\REDLKT{\stackrel{\LKT}{\longrightarrow}}
\def\DERIVELJT#1#2{\DERI{\REDLJT}{#1}{#2}}
\def\DERIVELKT#1#2{\DERI{\REDLKT}{#1}{#2}}

%%%%%%%%%%%%%%%%%%%%%%%%%%%%%%%%%%%%%%%%%%%%%%%%%%%%%%%%%%%%%%%%%%%%%%
\newcommand{\Alt}{ \mid\!\!\mid  } % Alternative (lower and symmetric)
\newcommand \seql[3]
{\raisebox{3ex}{$\mbox{$#1$}\;\;$} \; \shortstack{$#2$ \\ \mbox{}\\
    \mbox{}\hrulefill\mbox{}\\ \mbox{}\\ $#3$}}
\newcommand \seqr[3]
{\shortstack{$#2$ \\ \mbox{}\\
    \mbox{}\hrulefill\mbox{}\\ \mbox{}\\ $#3$} \; \raisebox{3ex}{$\;\;\mbox{$#1$}$}}
\newcommand \seqadmissibler[3]
{\shortstack{$#2$ \\ \mbox{}\\
    \mbox{}\hrulefill\mbox{}\\\hrulefill\mbox{}\\ \mbox{}\\ $#3$} \; \raisebox{3ex}{$\;\;\mbox{$#1$}$}}
\newcommand \seq[2]{\shortstack{$#1$ \\ \mbox{}\\
    \mbox{}\hrulefill\mbox{}\\ \mbox{}\\ $#2$}}
\newcommand{\empht}[1]{{#1}}
\newcommand \seqdots[2]{\shortstack{$#1$ \\ \mbox{}\\
    \mbox{}\vdots\mbox{}\\ \mbox{}\\ $#2$}}
\newcommand \seqdotsl[3]{\shortstack{$#2$ \\ \mbox{}\\
    \mbox{}\vdots\mbox{$#1$}\\ \mbox{}\\ $#3$}}
\newcommand\definis[1]{\textbf{#1}}

\newcommand{\cad}{{c'est-à-dire}}

%%%%%%%%%%%%%%%%%%%%%%%%%%%%%%%%%%%%%%%%%%%%%%%%%%%%%%%%%%%%%%%%%%%%%%
%% Noms de calcul

\newcommand{\lkg}[3]{#1 \,|\, #2\vdash  #3}
\newcommand{\lkd}[3]{#1 \vdash #2 \,|\, #3}
\newcommand{\lk}[2]{#1 \vdash  #2}
\newcommand{\lkgl}[3]{#1 \,;\, #2\vdash  #3}
\newcommand{\lkdl}[3]{#1 \vdash #2 \,;\, #3}
\newcommand{\lksubst}[4]{(#1 \vdash  #2) \implies (#3 \vdash #4)}

\newcommand{\lktqg}[4]{#1 \,|\, #2\vdash  #4}
\newcommand{\lktqd}[4]{#1 \vdash #3 \,|\, #4}
\newcommand{\lktq}[4]{#1 \vdash  #4}
\newcommand{\lktqgi}[5]{#2 \,|\, #3\stackrel{#1}{\vdash} #5}
\newcommand{\lktqgli}[5]{#2 \,;\, #3\stackrel{#1}{\vdash} #5}
\newcommand{\lktqdi}[5]{#2 \stackrel{#1}{\vdash} #4 \,|\, #5}
\newcommand{\lktqdli}[5]{#2 \stackrel{#1}{\vdash} #4 \,;\, #5}
\newcommand{\lktqi}[5]{#2 \stackrel{#1}{\vdash} #5}
\newcommand{\lktqgl}[4]{#1 \,;\, #2\vdash  #4}
\newcommand{\lktqdl}[4]{#1 \vdash #3 \,;\, #4}
\newcommand{\lktqgll}[4]{#1 || #2\vdash  #4}
\newcommand{\lktqdll}[4]{#1 \vdash #3 || #4}

\newcommand{\lkch}{LK_{\mu\tilde\mu}}
\newcommand{\lkth}{LKT_{\mu}}
\newcommand{\lktch}{LKT_{\mu\tilde\mu}}
\newcommand{\lkqh}{LKQ_{\tilde\mu}}
\newcommand{\lkqch}{LKQ_{\mu\tilde\mu}}
\newcommand{\lkchset}{LK_{\mu\tilde\mu/}}

\newcommand{\lkcht}{LK_{\mu_n\tilde\mu}}
\newcommand{\lkchq}{LK_{\mu\tilde\mu_v}}

\newcommand{\lbdch}{\overline\lambda\mu\tilde\mu}
\newcommand{\lbdchmucbn}{\overline\lambda\mu_n}
\newcommand{\lbdchcbv}{\overline\lambda\mu\tmu_v}
\newcommand{\lbdchtmucbv}{\overline\lambda\tmu_v}
\newcommand{\lbdtch}{\lbdch_T}
\newcommand{\lbdqch}{\lbdch_Q}

\newcommand{\lbdsym}{\lbd^{Sym}}

%% Regles de reduction

\newcommand{\letlift}{\mathit{let_{\mathit lift}}}
\newcommand{\etalet}{\eta_{\mathit let}}
\newcommand{\muapp}{\mu_{\mathit{app}}}
\newcommand{\omuapp}{\omu_{\mathit{app}}}
\newcommand{\muvar}{\mu_{\mathit{var}}}
\newcommand{\mulet}{\mu_{\mathit{let}}}
\newcommand{\betav}{\beta_v}

\newcommand{\FV}{FV}

%%%%%%%%%%%%%%%%
%% Constructeurs

\definecolor{gray}{rgb}{0.18,0.18,0.18}

\newcommand{\black}[1]{\textcolor{black}{#1}}
\newcommand{\blue}[1]{\textcolor{black}{#1}}
\newcommand{\redcolor}[1]{\textcolor{black}{#1}}
\newcommand{\gray}[1]{\textcolor{gray}{#1}}

%\newcommand{\coupesep}{\Join}
\newcommand{\coupesep}{|\!|}
\newcount\coupelevel
\newcount\coupecolor
\coupelevel=0
\newcommand{\coupe}[2]{\black{
    \ifnum \coupelevel>0
      %    \overline{\langle \redcolor{#1} |\!| \blue{#2} \rangle}
      \coupecolor=\coupelevel
      \divide\coupecolor by 3\relax
      \output{\coupecolor}
      {\mbox{\textlangle}} \overline{\gray{#1} \gray{\coupesep} \gray{#2}} \mbox{\textrangle}
    \else
      \advance\coupelevel by 1\relax\langle \redcolor{#1} \coupesep \blue{#2} \rangle
      \output{\coupelevel}
    \fi}}
\newcommand{\cut}[2]{\coupe{#1}{#2}}
\newcommand{\cons}[2]{{#1}\cdot{#2}}
\newcommand{\consproj}[2]{{#1}[{#2}]}

\newcommand{\tmu}{{\tilde\mu}}
\newcommand{\muof}[2]{{\mu #1.#2}}
\newcommand{\lbar}{{\overline{\lambda}}}
\newcommand{\Sub}[3]{#1[{#3}/{#2}]}
\newcommand{\DSub}[3]{#1[#2\hookleftarrow #3]}
\newcommand{\CSub}[3]{#1[#2\leadsto #3]}
\newcommand{\omu}{\bar{\mu}}
\newcommand{\oalpha}{\star}
\newcommand{\omualpha}{\omu\!\star} % macro spéciale qui supprime un espace

\renewcommand{\l}{\lambda}
\newcommand{\lbd}{\lambda}
\newcommand{\letin}[3]{\mathbf{let}~{#1}=#2\;\mathbf{in}~#3}
\newcommand{\rawletin}{\mbox{\tt let-in}}

% Calculs avec liaison dynamique
\newcommand{\alphabf}{\hat{\alpha}}
\newcommand{\betabf}{\hat{\beta}}
\newcommand{\wmu}{\widehat\mu}
\newcommand{\lbdchdyn}{\overline\lambda\mu\tilde\mu\wmu}
\newcommand{\lbdchtpv}{\overline\lambda\mu\tilde\mu_v\#}
\newcommand{\lbdchtp}{\overline\lambda\mu\tilde\mu\#}

\newcommand{\darr}[2]{\;_{#1}\!\!\imp_{#2}}
\newcommand{\annot}[2]{{#1}_{#2}}

% Constructeur de la soustraction
\newcommand{\lbdminus}[2]{\tilde\lambda #1.#2}
\newcommand{\lbdminusl}[2]{\l_1 #1.#2}
\newcommand{\lbdminusr}[2]{\l_2 #1.#2}
\newcommand{\consminus}[2]{#2 - #1}

\newcommand{\tnu}{{\tilde\nu}}

\newcommand{\pair}[2]{({#1},{#2})}
\newcommand{\case}[2]{[{#1},{#2}]}
\newcommand{\unit}{\ltimes}
\newcommand{\tp}{\rtimes}
\newcommand{\notv}[1]{\neg(#1)}
\newcommand{\note}[1]{\neg[#1]}

\newcommand{\conthole}{\mbox{\tiny $\blacksquare$}}
\newcommand{\termhole}{\bullet}

%%%%%%%%%%%%%%%%%%%%%%%%%%%%%%%
% Control operators
\newcommand{\A}{\mathcal{A}}
\newcommand{\Shift}{\mathcal{S}}
\renewcommand{\P}{\#}
\newcommand{\F}{\mathcal{F}}
\newcommand{\K}{\mathcal{K}}
\newcommand{\abort}{\textit{abort}}
\newcommand{\callcc}{\textit{callcc}}
\newcommand{\shift}{\textit{shift}}
\newcommand{\reset}{\textit{reset}}
\newcommand{\prompt}{\textit{prompt}}

\newcommand{\Cm}{\mathcal{C^-}}
\newcommand{\tpcst}{\textsf{tp}}
\newcommand{\lCtop}{\lambda_{{\Cm\tpcst}}}
\newcommand{\botj}{\perp\!\!\!\!\perp}

\newcommand{\Type}{\mathsf{Type}}
\newcommand{\Set}{\mathsf{Set}}
\newcommand{\Prop}{\mathsf{Prop}}
\newcommand{\refl}[1]{\mathsf{refl}\,{#1}}
\newcommand{\mkmlidtype}[3]{{#1} =^{\mathrm{id}}_{#2} {#3}}
\newcommand{\mkmleqdep}[3]{{#1} =^{\mathrm{dep}}_{#2} {#3}}
\newcommand{\hrefl}[1]{\mathsf{refl^{\mathrm{dep}}}\,{#1}}
\newcommand{\mluniv}{\mathsf{U}}
\newcommand{\Jdep}[2]{\mathsf{J}\;{#1}\;{#2}}

\newcommand{\nth}[1]{{#1}^{\mbox{\scriptsize th}}}
\newcommand{\binomial}[2]{\left(\begin{array}{c}#1\\#2\end{array}\right)}

\newcommand{\wfctx}[1]{{#1}~\mathsf{ok}}

\newcommand{\mkover}[1]{\widetilde{#1}}
\newcommand{\mkoverdim}[2]{\widetilde{#2}^{#1}}
\newcommand{\mkprod}[3]{\Pi {#2}.\,{#3}}
\newcommand{\mkprodovereq}[3]{\mkover{\Pi} {#1}\!:\!{#2}.\,{#3}}
\newcommand{\mkprodovereqopp}[3]{\opp{\mkover{\Pi}} {#1}\!:\!{#2}.\,{#3}}
\newcommand{\mkboundedprod}[4]{\Pi {#3}\!:\!{#1}\leq{#2}.\,{#4}}
\newcommand{\mklam}[3]{\lambda^{#2}.\,{#3}}
\newcommand{\mkboundedlam}[4]{\lambda {#3}\!:\!{#1}\leq{#2}.\,{#4}}
\newcommand{\mkapp}[2]{{#1}\,{#2}}
\newcommand{\prodsort}[2]{\Pi ({#1},{#2})}

% selecting kind of treatment of sorts
\newcommand{\typeorsort}[2]{{#2}}

\newcommand{\mksigma}[3]{\Sigma {#2}.\,{#3}}
\newcommand{\mksigmaovereq}[3]{\mkover{\Sigma} {#1}\!:\!{#2}.\,{#3}}
\newcommand{\mksigmaovereqopp}[3]{\opp{\mkover{\Sigma}} {#1}\!:\!{#2}.\,{#3}}
\newcommand{\mkpair}[2]{\langle{#1},{#2}\rangle}
\newcommand{\mkfst}[1]{{#1}.1}
\newcommand{\mksnd}[1]{{#1}.2}
\newcommand{\sigsort}[2]{\Sigma ({#1},{#2})}

\newcommand{\sortsort}[1]{{\mathcal S}_{#1}}

\newcommand{\Sone}{\mathbb{S}^1}
\newcommand{\casesone}[5]
  {\mathsf{case}\;{#1}\;\mathsf{of}\;{#2}\;\Rightarrow\;{#3}\;|\;{#4}\;\Rightarrow\;{#5}\;\mathsf{end}}
\newcommand{\base}{\mathsf{base}}
\newcommand{\loopsone}{\mathsf{loop}}
\newcommand{\Sonesort}{l_{\mathbb{S}^1}}

\newcommand{\emptyctx}{\boxbox}

\newcommand{\sort}[1]{\mathsf{U}_{#1}}
\newcommand{\univ}{\boxempty}
\newcommand{\typeannot}{\mathsf{type}}
\newcommand{\istype}{~\typeannot}

\newcommand{\mkeq}[3]{{#1} =_{#2} {#3}}
\newcommand{\mkhomeq}[3]{{#1} =_{#2} {#3}}
\newcommand{\mkeqovereq}[3]{{#1} \,\mkover{=}_{#2}\, {#3}}
\newcommand{\mkeqovereqdim}[4]{{#2} \,\mkoverdim{#1}{=}_{#3}\, {#4}}
\newcommand{\mkeqtype}[3]{{#1} =_{\typeannot} {#3}}
\newcommand{\mkeqtypeovereq}[3]{{#1} \,\mkover{=}_{#2}\, {#3}} % case where we can write either s or refl{s}
\newcommand{\mkeqarray}[3]{\begin{array}{c}{#1}\\ =_{#2}\\ {#3}\end{array}}
\newcommand{\mkfaces}[3]{\mkeq{#1}{#2}{#3}}
\newcommand{\mkfacesarray}[3]{\mkeqarray{#1}{#2}{#3}}
\newcommand{\mkfacesover}[3]{\mkeqovereq{#1}{#2}{#3}}
\newcommand{\lameq}[2]{\lambda {#1}.{#2}}
\newcommand{\reflterm}[1]{\widehat{#1}}
\newcommand{\overreflterm}[1]{{\mkover{\reflterm{#1}}}}
\newcommand{\refltermdim}[2]{{\widehat{#2}}^{#1}}
\newcommand{\refltype}[1]{\reflterm{#1}}
\newcommand{\purerewlr}[1]{\overrightarrow{#1}}
\newcommand{\purerewrl}[1]{\overleftarrow{#1}}
\newcommand{\rewlr}[2]{\overrightarrow{#1}(#2)}
\newcommand{\rewrl}[2]{\overleftarrow{#1}(#2)}
\newcommand{\deprewlr}[1]{\raisebox{0em}{$\ulcorner$}\!#1}
\newcommand{\deprewrl}[1]{{#1}\raisebox{-0.2em}{$\!\lrcorner$}}
\newcommand{\rewlrspec}[2]{{#1}^{\rightarrow}(#2)}
\newcommand{\rewrlspec}[2]{{#1}^{\leftarrow}(#2)}
\newcommand{\rewlrrevspec}[2]{{#1}_-^{\rightarrow}(#2)}
\newcommand{\rewrlrevspec}[2]{{#1}_-^{\leftarrow}(#2)}
\newcommand{\rewlrrevspecin}[2]{{#1}_+^{\rightarrow}(#2)}
\newcommand{\rewrlrevspecin}[2]{{#1}_+^{\leftarrow}(#2)}
\newcommand{\doublerewlr}[2]{\Overrightarrow{1.4ex}{#1}({#2})}
\newcommand{\doublerewrl}[2]{\Overleftarrow{1.4ex}{#1}({#2})}
\newcommand{\mktuple}[6]{\{#3; #4; #5; #6\}_{#1,#2}}
\newcommand{\mktupleshort}[6]{\{#3; #4; #5; #6\}}
\newcommand{\opp}[1]{{#1}^{-1}}
\newcommand{\oppi}[2]{\appi{#1}{(\opp{#2}/{#2})}}

\newcommand{\weakensquarelr}[2]{{#1}_L(#2)}
\newcommand{\weakensquarerl}[2]{{#1}_R(#2)}

\newcommand{\rewlrmkeq}[4]{\mkeq{\rewlr{#1}{#2}}{#3}{#4}}

\newcommand{\appi}[2]{{#1}\;\!{#2}}
\newcommand{\appidep}[2]{{#1}_{#2}}
\newcommand{\bp}[1]{{#1}{\scriptstyle 0}}
\newcommand{\ep}[1]{{#1}{\scriptstyle 1}}
\newcommand{\bpoverdim}[2]{{#2}{\scriptstyle \mkoverdim{#1}{0}}}
\newcommand{\epoverdim}[2]{{#2}{\scriptstyle \mkoverdim{#1}{1}}}
\newcommand{\bpdep}[1]{{#1}_0}
\newcommand{\epdep}[1]{{#1}_1}

\newcommand{\bpsubstexpl}[2]{{#2}_{\{0/{#1}\}}}
\newcommand{\epsubstexpl}[2]{{#2}_{\{1/{#1}\}}}
\newcommand{\eqsubstexpl}[2]{{#2}_{\{\star/{#1}\}}}
\newcommand{\bpsubstexplcons}[2]{{#2}\}\{0/{#1}}
\newcommand{\epsubstexplcons}[2]{{#2}\}\{1/{#1}}
\newcommand{\eqsubstexplcons}[2]{{#2}\}\{\star/{#1}}
\newcommand{\bpsubst}[2]{{#2}{[0/{#1}]}}
\newcommand{\epsubst}[2]{{#2}{[1/{#1}]}}
\newcommand{\bpsubstctx}[3]{{#3}{[0/{#2}]}^{#1}}
\newcommand{\epsubstctx}[3]{{#3}{[1/{#2}]}^{#1}}
\newcommand{\takebpface}[2]{\partial^{#1}_0{#2}}
\newcommand{\takeepface}[2]{\partial^{#1}_1{#2}}

\newcommand{\dimvalid}[2]{{#1} \in {#2}}
\newcommand{\dimdeclare}[1]{{#1}}
\newcommand{\dimlength}[1]{|#1|_{\mathit{dim}}}
\newcommand{\dimindex}[2]{\#_{#1}{#2}}

\newcommand{\substminus}[2]{{#1}-{#2}}
\newcommand{\substminusbernardymoulin}[2]{{#1}/{#2}}

\newcommand{\N}{\mathbb{N}}
\newcommand{\expand}[2]{\mathsf{expand}(#1,#2)}

\newcommand{\defeq}{\triangleq}
\newcommand{\metaequiv}{\cong}
\newcommand{\metaletin}[3]{\mathit{let}\;{#1}\;\defeq\;{#2}\;\mathit{in}\;{#3}}
\newcommand{\map}[2]{\mathsf{ap}\,{#1}\,{#2}}
\newcommand{\depmap}[2]{\mathsf{apd}\,{#1}\,{#2}}
\newcommand{\transport}[3]{\mathsf{transport}\,{#1}\,{#2}\,{#3}}
\newcommand{\deptransport}[3]{\mathsf{transportd}\,{#1}\,{#2}\,{#3}}
\newcommand{\swap}[1]{{#1}^{\circ}}
\newcommand{\swapbracket}[1]{({#1})^{\circ}}
\newcommand{\swaptype}[1]{{#1}^{\circ}}
\newcommand{\swapoverdim}[2]{{#2}^{\mkoverdim{#1}{\circ}}}
\newcommand{\homsquare}[1]{\mathsf{homsquare}(#1)}
\newcommand{\permute}[2]{\mathsf{permute}_{#1}(#2)}
\newcommand{\swapprodeq}[2]{\mathsf{swap}_{\Pi=}^{#1}(#2)}
\newcommand{\swapeqprod}[2]{\mathsf{swap}_{=\Pi}^{#1}(#2)}
\newcommand{\swapeqprodlr}[2]{\mathsf{swap}_{=\Pi}^{\rightarrow}({#1})(#2)}
\newcommand{\swapeqprodrl}[2]{\mathsf{swap}_{=\Pi}^{\leftarrow}({#1})(#2)}
\newcommand{\swapsigmaeq}[1]{\mathsf{swap}_{\Sigma=}(#1)}
\newcommand{\swapeqsigma}[1]{\mathsf{swap}_{=\Sigma}(#1)}

\newcommand{\bool}{\mathsf{bool}}
\newcommand{\booldeux}{\bool^2}
\newcommand{\boolexp}{\bool^{\bool}}

\newcommand{\reduce}{\;\triangleright\;}

\newcommand{\emptysubst}{\emptyctx}

\newcommand{\sortrule}{\sort{}}
\newcommand{\axrule}{\mathsf{Ax}}
\newcommand{\ctxemptyrule}{\mathsf{Ctx}_{\emptyctx}}
\newcommand{\ctxconsrule}{\mathsf{Ctx}_{\mathsf{cons}}}
\newcommand{\convrule}{\mathsf{Conv}}
\newcommand{\convconvrule}{\mathsf{Conv}-\mathsf{Conv}}
\newcommand{\univrule}{\mathsf{\univ}}
\newcommand{\convruleredright}{\reduce_R}
\newcommand{\convruleredleft}{\reduce_L}
\newcommand{\convrulerefl}{\equiv_{\mathsf{refl}}}
\newcommand{\convruletrans}{\equiv_{\mathsf{trans}}}
\newcommand{\convruleredtyperight}{\reduce_R^{\typeannot}}
\newcommand{\convruleredtypeleft}{\reduce_L^{\typeannot}}
\newcommand{\convruletyperefl}{\equiv_{\mathsf{refl}}^{\typeannot}}
\newcommand{\convruletypetrans}{\equiv_{\mathsf{trans}}^{\typeannot}}

\newcommand{\circovereq}{\;\mkover{\circ}\;}
\newcommand{\circsigma}{\circ_{\Sigma}}

\newcommand{\comprule}{R_{\mathsf{\circ}}}
\newcommand{\opprule}{R_{\mathsf{\opp}}}

\newcommand{\idleftredrule}{\mathsf{Id_L}}
\newcommand{\idrightredrule}{\mathsf{Id_R}}
\newcommand{\idoppredrule}{\mathsf{\opp{Id}}}
\newcommand{\oppoppredrule}{\mathsf{\opp{(\opp{\_})}}}
\newcommand{\assoccompredrule}{\mathsf{Assoc}}
\newcommand{\distriboppcompredrule}{\mathsf{\opp{\circ}}}
\newcommand{\squaredownredrule}{\circ_{=_{\mkover{=}}}}
\newcommand{\exchangerule}{\mathsf{Exch}}
\newcommand{\compeqovereqredrule}{\circ_{\mkover{=}}}
\newcommand{\squareprodredrule}{\circ_{=_{\mkover{\Pi}}}}

\newcommand{\depmapdef}{\mathsf{ApD}}
\newcommand{\funextpuredef}{\mathsf{LiftDim}}
\newcommand{\funextdeppuredef}{\mathsf{LiftDepDim}}
\newcommand{\funextdepdeppuredef}{\mathsf{LiftDepDim}}
\newcommand{\funextdef}{\mathsf{FunExt}}
\newcommand{\funextdepdef}{\mathsf{FunDepExt}}
\newcommand{\funext}[1]{\mathsf{funext}(#1)}
\newcommand{\funextdep}[1]{\mathsf{fundepext}(#1)}
\newcommand{\funextpure}[1]{\mathsf{liftdim}(#1)}
\newcommand{\funextdeppure}[1]{\mathsf{liftdepdim}(#1)}

\newcommand{\OMEGATT}{\omega\mathsf{TT}}

\newcommand{\emptysigma}{\bullet}
\newcommand{\itsemterm}[4]{\llbracket #4 \rrbracket^{#2;#3}_{#1}}
\newcommand{\itsemtype}[4]{\llbracket #4 \rrbracket^{#2;#3}_{#1}}
\newcommand{\semterm}[3]{\llbracket #3 \rrbracket^{#1;#2}}
\newcommand{\semtype}[3]{\llbracket #3 \rrbracket^{#1;#2}}
\newcommand{\incrstep}[3]{\mathsf{split}_{#1}{#3}(#2)}
\newcommand{\distrstep}[3]{\mathsf{distr}_{#1}{#3}(#2)}
\newcommand{\applytype}[4]{\mathsf{apptype}_{#1}^{#2}(#3,#4)}
\newcommand{\shrink}[3]{\mathsf{shrink}_{#1}\,{#2}\,#3}
\newcommand{\under}[3]{\mathsf{under}_{#1}\,{#2}\,#3}
\newcommand{\undertau}[3]{\mathsf{under}_{#1}^{#2}\,#3}
\newcommand{\proj}[3]{\pi_{#1}^{#2}\,#3}
\newcommand{\elt}[2]{\mathsf{El}_{#1}\,#2}
\newcommand{\semtypenew}[2]{| #2 |_{#1}}
\newcommand{\semnew}[2]{\llbracket #2 \rrbracket_{#1}}
\newcommand{\semctx}[2]{\llbracket #2 \rrbracket_{#1}}
\newcommand{\semlam}[5]{\mathsf{lam}_{#1}^{#2}\,#4\,(#3)\,#5}
\newcommand{\semlambis}[5]{\mathsf{lam'}_{#1}^{#2}\,#4\,(#3)\,#5}
\newcommand{\semapp}[3]{\mathsf{app}_{#1}#2\,#3}
\newcommand{\semprod}[5]{\mathsf{prod}_{#1}^{#2}\,#4\,(#3)\,#5}
\newcommand{\access}[2]{\mathsf{get}_{#1}(#2)}
\newcommand{\accesstyped}[3]{\mathsf{get}_{#1}^{#2}(#3)}
\newcommand{\lift}[1]{\mathsf{lift}\, {#1}}
\newcommand{\liftlr}[1]{\mathsf{lift}^\rightarrow\, {#1}}
\newcommand{\liftrl}[1]{\mathsf{lift}^\leftarrow\, {#1}}
\newcommand{\diag}[2]{\mathsf{diag}_{#1}(#2)}
\newcommand{\semsort}[1]{\mathsf{sort}_{#1}}
\newcommand{\reflsort}[1]{\mathsf{refl}^{\sort{}}_{#1}}
\newcommand{\mktuplenew}[6]{\{#4\,|\begin{array}{l} #5 \\ #6 \end{array}\}_{#1,#2,#3}}
\newcommand{\mktuplein}[4]{{#1}; {#2}; {#3}; {#4}}
\newcommand{\nwconn}[2]{\deprewlr{#2}_{#1}}
\newcommand{\seconn}[2]{\deprewrl{#2}_{#1}}
\newcommand{\refltermn}[3]{\mathsf{reflterm}_{#1}(#3)}
\newcommand{\refltypen}[2]{\mathsf{refltype}_{#1}(#2)}

\newcommand{\restrict}{\mathsf{restrict}}
\newcommand{\MofL}{\mathsf{underlying}}
\newcommand{\defequiv}{\mathsf{\simeq}}
\newcommand{\isequivalence}{\mathsf{isequiv}}
\newcommand{\gpdlevel}{\mathsf{gpdlevel}}
\newcommand{\regrefl}{\mathsf{regrefl}}

\newcommand{\homoname}{}
\newcommand{\heteroname}{het}
\newcommand{\mycubsetcomp}[2]{\mathsf{cubset}_{#1}^{=#2}}

\newcommand{\informalcube}{\mathsf{cube}}
\newcommand{\informalcubeover}{\mathsf{\cubeovername}}

\newcommand{\cubeovername}{cube}
\newcommand{\Cubeovername}{Cube}
\newcommand{\pboxname}{box}
\newcommand{\pBoxname}{Box}

\newcommand{\mybox}[1]{\mathsf{{\homoname}box}_{#1}}
\newcommand{\mylayer}[1]{\mathsf{{\homoname}layer}_{#1}}
\newcommand{\mycube}[1]{\mathsf{{\homoname}\cubeovername}_{#1}}
\newcommand{\downbox}[2]{\mathsf{sub{\homoname}box}_{#1}^{#2}}
\newcommand{\downlayer}[2]{\mathsf{sub{\homoname}layer}_{#1}^{#2}}
\newcommand{\downcube}[2]{\mathsf{sub{\homoname}cube}_{#1}^{#2}}
\newcommand{\cohbox}[2]{\mathsf{coh{\homoname}box}_{#1}^{#2}}
\newcommand{\cohlayer}[2]{\mathsf{coh{\homoname}layer}_{#1}^{#2}}
\newcommand{\cohcube}[2]{\mathsf{coh{\homoname}\cubeovername}_{#1,#2}}
\newcommand{\myfullbox}[1]{\mathsf{full{\homoname}box}_{#1}}
\newcommand{\fulldownbox}[2]{\mathsf{subfull{\homoname}box}_{#1}^{#2}}
\newcommand{\fulldowncube}[2]{\mathsf{subfull{\homoname}cube}_{#1}^{#2}}

\newcommand{\myheterobox}[2]{\mathsf{{\heteroname}box}_{#1}^{#2}}
\newcommand{\myheterolayer}[2]{\mathsf{{\heteroname}layer}_{#1}^{#2}}
\newcommand{\myheterocube}[2]{\mathsf{{\heteroname}\cubeovername}_{#1}^{#2}}
\newcommand{\myheteroboxtype}[2]{\mathsf{Box}_{#1}^{#2}}
\newcommand{\myheterolayertype}[2]{\mathsf{Layer}_{#1}^{#2}}
\newcommand{\myheterocubetype}[2]{\mathsf{\Cubeovername}_{#1}^{#2}}

\newcommand{\myfulldepheterobox}[3]{\mathsf{fulldepBox}_{#1}^{#2,#3}}
\newcommand{\mydepheterobox}[3]{\mathsf{depBox}_{#1}^{#2,#3}}
\newcommand{\mydepheterolayer}[3]{\mathsf{depLayer}_{#1}^{#2,#3}}
\newcommand{\mydepheterocube}[3]{\mathsf{dep\Cubeovername}_{#1}^{#2,#3}}
\newcommand{\liftfulldepheterobox}[3]{\mathsf{liftfullBox}_{#1}^{#2,#3}}
\newcommand{\liftdepheterobox}[3]{\mathsf{liftBox}_{#1}^{#2,#3}}
\newcommand{\liftdepheterolayer}[3]{\mathsf{liftLayer}_{#1}^{#2,#3}}
\newcommand{\liftdepheterocube}[3]{\mathsf{lift\Cubeovername}_{#1}^{#2,#3}}

\newcommand{\expandheterocube}[2]{\mathsf{expand\Cubeovername}_{#1}^{#2}}
\newcommand{\expandheterolayer}[2]{\mathsf{expandLayer}_{#1}^{#2}}
\newcommand{\expanddownbox}[2]{\mathsf{expandsub{\homoname}box}_{#1}^{#2}}
\newcommand{\expanddownlayer}[2]{\mathsf{expandsub{\homoname}layer}_{#1}^{#2}}
\newcommand{\expanddowncube}[2]{\mathsf{expandsub{\homoname}cube}_{#1}^{#2}}

\newcommand{\depdownbox}[2]{\mathsf{depsub{\homoname}box}_{#1}^{#2}}
\newcommand{\depdownlayer}[2]{\mathsf{depsub{\homoname}layer}_{#1}^{#2}}
\newcommand{\depdowncube}[2]{\mathsf{depsub{\homoname}cube}_{#1}^{#2}}
\newcommand{\depdownboxbis}[2]{\mathsf{depsublift{\homoname}box}_{#1}^{#2}}
\newcommand{\depdownlayerbis}[2]{\mathsf{depsublift{\homoname}layer}_{#1}^{#2}}
\newcommand{\depdowncubebis}[2]{\mathsf{depsublift{\homoname}cube}_{#1}^{#2}}
\newcommand{\depdownboxter}[2]{\mathsf{depsublift{\homoname}box'}_{#1}^{#2}}
\newcommand{\depdownlayerter}[2]{\mathsf{depsublift{\homoname}layer'}_{#1}^{#2}}
\newcommand{\depdowncubeter}[2]{\mathsf{depsublift{\homoname}cube'}_{#1}^{#2}}

\newcommand{\depcohbox}[2]{\mathsf{depcoh{\homoname}box}_{#1}^{#2}}
\newcommand{\depcohlayer}[2]{\mathsf{depcoh{\homoname}layer}_{#1}^{#2}}
\newcommand{\depcohcube}[2]{\mathsf{depcoh{\homoname}cube}_{#1}^{#2}}
\newcommand{\depcohboxbis}[2]{\mathsf{depcohlift{\homoname}box}_{#1}^{#2}}
\newcommand{\depcohlayerbis}[2]{\mathsf{depcohlift{\homoname}layer}_{#1}^{#2}}
\newcommand{\depcohcubebis}[2]{\mathsf{depcohlift{\homoname}cube}_{#1}^{#2}}
\newcommand{\depcohboxter}[2]{\mathsf{depcohlift{\homoname}box'}_{#1}^{#2}}
\newcommand{\depcohlayerter}[2]{\mathsf{depcohlift{\homoname}layer'}_{#1}^{#2}}
\newcommand{\depcohcubeter}[2]{\mathsf{depcohlift{\homoname}cube'}_{#1}^{#2}}

\newcommand{\univfullcube}[2]{\mathsf{UnivfullCube}_{#1}^{#2}}
\newcommand{\univfullbox}[2]{\mathsf{UnivfullBox}_{#1}^{#2}}
\newcommand{\univbox}[2]{\mathsf{UnivBox}_{#1}^{#2}}
\newcommand{\univlayer}[2]{\mathsf{UnivLayer}_{#1}^{#2}}
\newcommand{\univcube}[2]{\mathsf{UnivCube}_{#1}^{#2}}
\newcommand{\univfiller}[2]{\mathsf{UnivFiller}_{#1}^{#2}}
\newcommand{\cubeofuniv}[2]{\mathsf{cubeofuniv}_{#1}^{#2}}
\newcommand{\extend}[2]{\mathsf{extendcube}_{#1}^{#2}}

\newcommand{\powersetfullcube}[2]{\mathsf{powersetfullCube}_{#1}^{#2}}

\newcommand{\heterodownbox}[2]{\mathsf{sub{\heteroname}box}_{#1}^{#2}}
\newcommand{\heterodownlayer}[2]{\mathsf{sub{\heteroname}layer}_{#1}^{#2}}
\newcommand{\heterodowncube}[2]{\mathsf{sub{\heteroname}\cubeovername}_{#1}^{#2}}
\newcommand{\depdownboxtype}[2]{\mathsf{Subdep{\homoname}box}_{#1}^{#2}}
\newcommand{\depdownlayertype}[2]{\mathsf{Subdep{\homoname}layer}_{#1}^{#2}}
\newcommand{\depdowncubetype}[2]{\mathsf{Subdep{\homoname}\cubeovername}_{#1}^{#2}}
\newcommand{\downboxtype}[2]{\mathsf{Subbox}_{#1}^{#2}}
\newcommand{\downlayertype}[2]{\mathsf{Sublayer}_{#1}^{#2}}
\newcommand{\downcubetype}[2]{\mathsf{Sub\cubeovername}_{#1}^{#2}}
\newcommand{\cohreflbox}[1]{\mathsf{cohreflbox}_{#1}}
\newcommand{\cohrefllayer}[1]{\mathsf{cohrefllayer}_{#1}}
\newcommand{\cohreflcube}[1]{\mathsf{cohreflcube}_{#1}}
\newcommand{\cohrefleqbox}[1]{\mathsf{cohrefleqbox}_{#1}}
\newcommand{\cohrefleqlayer}[1]{\mathsf{cohrefleqlayer}_{#1}}
\newcommand{\cohrefleqcube}[1]{\mathsf{cohrefleqcube}_{#1}}
\newcommand{\cohboxtype}[2]{\mathsf{Cohbox}_{#1}^{#2}}
\newcommand{\cohlayertype}[2]{\mathsf{Cohlayer}_{#1}^{#2}}
\newcommand{\cohcubetype}[2]{\mathsf{Coh\cubeovername}_{#1}^{#2}}
\newcommand{\heterocohbox}[2]{\mathsf{coh{\heteroname}box}_{#1}^{#2}}
\newcommand{\heterocohlayer}[2]{\mathsf{coh{\heteroname}layer}_{#1}^{#2}}
\newcommand{\heterocohcube}[2]{\mathsf{coh{\heteroname}\cubeovername}_{#1}^{#2}}
\newcommand{\myappdepbox}[1]{\mathsf{homoappbox}_{#1}}
\newcommand{\myappdeplayer}[1]{\mathsf{homoapplayer}_{#1}}
\newcommand{\myappdepcube}[1]{\mathsf{homoapp\cubeovername}_{#1}}
\newcommand{\myappdepfullcube}[1]{\mathsf{homoappfullcube}_{#1}}
\newcommand{\myappdepboxtype}[1]{\mathsf{homoappbox}_{#1}}
\newcommand{\myappdepfullboxtype}[1]{\mathsf{homoappfullbox}_{#1}}
\newcommand{\myappdeplayertype}[1]{\mathsf{homoapplayer}_{#1}}
\newcommand{\myappdepcubetype}[1]{\mathsf{homoapp\cubeovername}_{#1}}
\newcommand{\myappdepfullcubetype}[1]{\mathsf{homoappfullcube}_{#1}}
\newcommand{\myfullcube}[1]{\mathsf{full{\homoname}cube}_{#1}}
\newcommand{\myfullheteroboxtype}[2]{\mathsf{fullBox}_{#1}^{#2}}
\newcommand{\myfullheterobox}[2]{\mathsf{full{\heteroname}box}_{#1}^{#2}}
\newcommand{\reflfullcube}[1]{\mathsf{reflfullcube}_{#1}}
\newcommand{\unitpoint}{\star}
\newcommand{\unittype}{\mathsf{unit}}
\newcommand{\eqett}{\equiv_{\mathit{ETT}}}
\newcommand{\reflett}{\mathsf{refl}_{\mathit{ETT}}}
\newcommand{\mygpd}{\mathsf{gpd}}

\newcommand{\mycubset}[1]{\mathsf{cubset}_{#1}}
\newcommand{\mycubsetfrom}[2]{\mathsf{cubset}_{#1}^{\geq#2}}
\newcommand{\partialcubset}[2]{\mathsf{cubset}_{#1}^{<#2}}
\newcommand{\mydepcubset}[1]{\mathsf{dephomocubset}_{#1}}
\newcommand{\mydepcubsetfrom}[2]{\mathsf{dephomocubset}_{#1}^{\geq#2}}
\newcommand{\mydepcubsetcomp}[2]{\mathsf{dephomocubset}_{#1}^{=#2}}
\newcommand{\deppartialcubset}[2]{\mathsf{dephomocubset}_{#1}^{<#2}}
\newcommand{\properheterocubetype}[2]{\mathsf{Filler}_{#1}^{#2}}

\newcommand{\mydepheterocubset}[2]{\mathsf{depcubset}_{#1}^{#2}}
\newcommand{\mydepheterocubsetfrom}[3]{\mathsf{depcubset}_{#1}^{#2,\geq#3}}
\newcommand{\mydepheterocubsetcomp}[3]{\mathsf{depcubset}_{#1}^{#2,=#3}}
\newcommand{\deppartialheterocubset}[3]{\mathsf{depcubset}_{#1}^{#2,<#3}}
\newcommand{\myappheterodepbox}[1]{\mathsf{appbox}_{#1}}
\newcommand{\myappheterodeplayer}[1]{\mathsf{applayer}_{#1}}
\newcommand{\myappheterodepcube}[1]{\mathsf{app\cubeovername}_{#1}}
\newcommand{\myappheterodepfullcube}[1]{\mathsf{appfullcube}_{#1}}
\newcommand{\myappheterodepboxtype}[1]{\mathsf{appbox}_{#1}}
\newcommand{\myappheterodepfullboxtype}[3]{\mathsf{appfullbox}_{#1}^{#2,#3}}
\newcommand{\myappheterodeplayertype}[1]{\mathsf{applayer}_{#1}}
\newcommand{\myappheterodepcubetype}[1]{\mathsf{app\cubeovername}_{#1}}
\newcommand{\myappheterodepfullcubetype}[1]{\mathsf{appfullcube}_{#1}}

\newcommand{\typecubset}{\mathsf{typecubset}}
\newcommand{\mysymcubset}{\mathsf{symcubset}}
\newcommand{\issymcubset}{\mathsf{issymcubset}}
\newcommand{\partialsymcubset}{\mathsf{symcubset}}
\newcommand{\mypermute}{\mathsf{permute}}
\newcommand{\partialhascomp}{\mathsf{partialhascomp}}
\newcommand{\hascomp}{\mathsf{hascomp}}
\newcommand{\tube}[1]{\mathsf{{\homoname}tube}_{#1}}
\newcommand{\propercube}[1]{\mathsf{proper{\homoname}filler}_{#1}}
\newcommand{\properheterocube}[2]{\mathsf{{\heteroname}filler}_{#1}^{#2}}
\newcommand{\border}[1]{\mathsf{{\homoname}border}_{#1}}
\newcommand{\bordertype}[1]{\mathsf{Boxofcube}_{#1}}
\newcommand{\cohbordertype}[1]{\mathsf{Cohboxofcube}_{#1}}
\newcommand{\myheteroborder}[1]{\mathsf{{\heteroname}border}_{#1}}
\newcommand{\fullheterocubetype}[2]{\mathsf{fullCube}_{#1}^{#2}}
\newcommand{\fullheterocube}[1]{\mathsf{full{\heteroname}cube}_{#1}}
\newcommand{\heterotubetype}[1]{\mathsf{Tube}_{#1}}
\newcommand{\heterotube}[1]{\mathsf{{\heteroname}tube}_{#1}}
\newcommand{\mygroundedcube}[1]{\mathsf{grounded{\homoname}cube}_{#1}}
\newcommand{\mygroundedheterocubetype}[1]{\mathsf{groundedCube}_{#1}}
\newcommand{\mygroundedheterocube}[1]{\mathsf{grounded{\heteroname}cube}_{#1}}
\newcommand{\reflexive}[1]{\mathsf{reflexive}_{#1}}
\newcommand{\permutation}{\mathsf{permutation}}
\newcommand{\diagonal}{\mathsf{diagonal}}
\newcommand{\connection}{\mathsf{connection}}
\newcommand{\diagprop}{\mathsf{diag}}
\newcommand{\reglr}{\rightarrow-\mathsf{reg}}
\newcommand{\regrl}{\leftarrow-\mathsf{reg}}
\newcommand{\coercivelr}{\rightarrow-\mathsf{def}}
\newcommand{\coerciverl}{\leftarrow-\mathsf{def}}
\newcommand{\regularitylr}{\rightarrow-\mathsf{regularity}}
\newcommand{\regularityrl}{\leftarrow-\mathsf{regularity}}
\newcommand{\Reglr}{\Rightarrow-\mathsf{reg}}
\newcommand{\Regrl}{\Leftarrow-\mathsf{reg}}
\newcommand{\Coercivelr}{\Rightarrow-\mathsf{def}}
\newcommand{\Coerciverl}{\Leftarrow-\mathsf{def}}
\newcommand{\Regularitylr}{\Rightarrow-\mathsf{regularity}}
\newcommand{\Regularityrl}{\Leftarrow-\mathsf{regularity}}
\newcommand{\coebacklr}{\overrightarrow{\mathsf{coe}}}
\newcommand{\coebackrl}{\overleftarrow{\mathsf{coe}}}
\newcommand{\backcoercivelr}{\overrightarrow{\mathsf{coe}}-\mathsf{def}}
\newcommand{\backcoerciverl}{\overleftarrow{\mathsf{coe}}-\mathsf{def}}
\newcommand{\regbacklr}{\overrightarrow{\mathsf{coe}}-\mathsf{reg}}
\newcommand{\regbackrl}{\overleftarrow{\mathsf{coe}}-\mathsf{reg}}
\newcommand{\regularitybacklr}{\overrightarrow{\mathsf{coe}}-\mathsf{regularity}}
\newcommand{\regularitybackrl}{\overleftarrow{\mathsf{coe}}-\mathsf{regularity}}
\newcommand{\Coebacklr}{\Overrightarrow{1ex}{\mathsf{coe}}}
\newcommand{\Coebackrl}{\Overleftarrow{1ex}{\mathsf{coe}}}
\newcommand{\Coercivebacklr}{\Overrightarrow{1ex}{\mathsf{coe}}-\mathsf{def}}
\newcommand{\Coercivebackrl}{\Overleftarrow{1ex}{\mathsf{coe}}-\mathsf{def}}
\newcommand{\Regbacklr}{\Overrightarrow{1ex}{\mathsf{coe}}-\mathsf{reg}}
\newcommand{\Regbackrl}{\Overleftarrow{1ex}{\mathsf{coe}}-\mathsf{reg}}
\newcommand{\Regularitybacklr}{\Overrightarrow{1ex}{\mathsf{coe}}-\mathsf{regularity}}
\newcommand{\Regularitybackrl}{\Overleftarrow{1ex}{\mathsf{coe}}-\mathsf{regularity}}

\newcommand{\reflbox}[1]{\mathsf{reflbox}_{#1}}
\newcommand{\refllayer}[1]{\mathsf{refllayer}_{#1}}
\newcommand{\reflcube}[1]{\mathsf{reflcube}_{#1}}
\newcommand{\permutebox}{\mathsf{permutebox}}
\newcommand{\diagbox}{\mathsf{diagbox}}
\newcommand{\connbox}{\mathsf{connbox}}
\newcommand{\hd}{\mathsf{hd}}
\newcommand{\tl}{\mathsf{tl}}
\newcommand{\first}{\mathsf{first}}
\newcommand{\groupoid}{\mathsf{groupoid}}
\newcommand{\groupoidn}{\mathsf{groupoidn}}
\newcommand{\letinsplit}[3]{\!\!\begin{array}{l}\mathbf{let}~{#1}=#2\;\mathbf{in}~\\#3\end{array}}

\newcommand{\UIP}{\mathsf{UIP}}

\title{An indexed construction of semi-cubical types}
\author{Hugo Herbelin}
\author{Ramkumar Ramachandra}

\begin{document}
\maketitle

\tableofcontents

\section{Introduction}
\section{Semi-cubical sets}
\label{sec:truncated-cubical-sets}

In this section, we give the core of the definition of semi-cubical
sets as the coinductive limit (see Figure~\ref{fig:barecubicalset}) of a
construction of truncated semi-cubical sets.

\subsection{Truncated semi-cubical sets}

The definition is dispatched over Figures~\ref{fig:barecubicalsetstructurecore}, \ref{fig:barecubicalsetstructure}, \ref{fig:barecubicalsetfaces} and \ref{fig:barecubicalsetcoherences}. It describes the structure of the underlying higher-dimensional relations on which cubes and boxes are built, together with the definition of homogeneous $n$-boxes, homogeneous $n$-cubes, together with face operations on boxes and cubes, together with commutation properties of the face operations. All are mutually defined as types of the target language ETT. Note that such relational structure, cubes and boxes are relative to a universe. Note the presence of a coherence condition $\cohbox{l}{}$ to ensuring that both sides of the equality in $\downlayer{l}{}$ and $\downcube{l}{}$ are in the same type. The proof of $\cohbox{l}{}$ itself requires an higher-dimensional coherence condition which we obtain by working here in ETT where all proofs of an equality are identified (principle of Unicity of Identity Proofs). Note that if the proofs of the same equality were not equated, there would be a need for arbitrary many higher-dimensional coherences (see e.g.~\cite{Herbelin15} for a discussion on the de facto need for recursive higher-dimensional coherence conditions in formulating higher-dimensional structures in type theory). Note also that for a given $n$, the coherence conditions evaluate to a reflexivity proof, so that the construction evaluates to an effective sequence of types of iterated relations not mentioning $\downbox{l}{}$ nor $\cohbox{l}{}$ anymore.

When reflexivities are excepted, we call the structure thus defined \emph{bare truncated cubical sets}: \emph{bare} because it can be seen as defining a cubical equivalent to semi-simplicial sets with only faces as part of the structure (otherwise said, another terminology could have been ``semi-cubical'' sets); \emph{truncated} because we consider only such cubical sets up to some fixed dimension.

\begin{figure*}
  \centerline{ \framebox{$
        \begin{array}{llcl}
          \multicolumn{4}{c}{\mbox{\textit{Type of truncated cubical sets}}}                                                     \\
          \\
          \partialcubset{l}{n}    &                          & :      & \sort{l+1}                                               \\
          \partialcubset{l}{0}    &                          & \defeq & \unittype                                                \\
          \partialcubset{l}{n'+1} &                          & \defeq & \Sigma D:\partialcubset{l}{n'}.\,\mycubsetcomp{l}{n'}(D) \\
          \\
          \multicolumn{4}{c}{\mbox{\textit{Structure carried at each dimension}}}                                                \\
          \\
          \mycubsetcomp{l}{n}     & (D:\partialcubset{l}{n}) & :      & \sort{l}                                                 \\
          \mycubsetcomp{l}{n}     & D                        & \defeq & \myfullbox{l}^{n}(D)\imp \sort{l}                        \\
        \end{array}
      $
    }}
  \caption{Definition of a truncated cubical set (main part of the higher-dimensional relation structure)}
  \label{fig:barecubicalsetstructurecore}
\end{figure*}

\begin{figure*}
  \centerline{ \framebox{$
        \begin{array}{llcl}
          \multicolumn{4}{c}{\mbox{\textit{Full homogeneous $n$-boxes}}}                                                                            \\
          \\
          \myfullbox{l}^{n}           & (D:\partialcubset{l}{n})        & :      & \sort{l}                                                         \\
          \myfullbox{l}^{n}           & D                               & \defeq & \mybox{l}^{n,n}(D)                                               \\
          \\
          \multicolumn{4}{c}{\mbox{\textit{Homogeneous partial $n$-boxes and homogeneous partial $n$-cubes}}}                                       \\
          \\
          \mybox{l}^{n,p,[p \leq n]}  & (D:\partialcubset{l}{n})        & :      & \sort{l}                                                         \\
          \mybox{l}^{n,0}             & D                               & \defeq & \unittype                                                        \\
          \mybox{l}^{n,p'+1}          & D                               & \defeq & \Sigma d:\mybox{l}^{n,p'}(D).\,\mylayer{l}^{n,p'}(D)(d)          \\
          \\
          \mylayer{l}^{n,p,[p < n]}   & \!\!\!\begin{array}{l}(D:\partialcubset{l}{n})\\(d:\mybox{l}^{n,p}(D))\end{array} & :      & \sort{l}                                                         \\
          \mylayer{l}^{n,p}           & D~d                             & \defeq & \!\!\begin{array}{l}\mycube{l}^{n-1,p}(\hd(D))(\tl(D))(\downbox{l,L,p}{n,p}(D)(d)) \\\times\; \mycube{l}^{n-1,p}(\hd(D))(\tl(D))(\downbox{l,R,p}{n,p}(D)(d))\end{array}                                    \\
          \\
          \mycube{l}^{n,p,[p \leq n]} & \!\!\!\begin{array}{l}(D:\partialcubset{l}{n})\\(E:\mycubsetcomp{l}{n}(D))\\(d:\mybox{l}^{n,p}(D))\end{array} & :      & \sort{l}                                                         \\
          \mycube{l}^{n,p,[p = n]}    & D~E~d                           & \defeq & E.\mathsf{rel}(d)                                                \\
          \mycube{l}^{n,p,[p < n]}    & D~E~d                           & \defeq & \Sigma b:\mylayer{l}^{n,p}(D)(d).\,\mycube{l}^{n,p+1}(D)(E)(d,b) \\
        \end{array}
      $
    }}
  \caption{Definition of a truncated cubical set (boxes)}
  \label{fig:barecubicalsetstructure}
\end{figure*}

\begin{figure*}
  \centerline{ \framebox{$
        \begin{array}{llcl}
          \downbox{l,\epsilon,q}{n,p,[p \leq q < n]}  & \!\!\!\begin{array}{l}(D:\partialcubset{l}{n})\\(d:\mybox{l}^{n,p}(D))\end{array} & :      & \mybox{l}^{n-1,p}(\hd(D))                                                             \\
          \downbox{l,\epsilon,q}{n,0}                 & D~\unitpoint                    & \defeq & \unitpoint                                                                            \\
          \downbox{l,\epsilon,q}{n,p'+1}              & D~(d,b)                         & \defeq & (\downbox{l,\epsilon,q}{n,p'}(D)(d),\downlayer{l,\epsilon,q}{n,p'}(D)(d)(b))          \\
          \\
          \downlayer{l,\epsilon,q}{n,p,[p < q < n]}   & \!\!\!\begin{array}{l}(D:\partialcubset{l}{n})\\(d:\mybox{l}^{n,p}(D))\\(b:\mylayer{l}^{n,p}(D)(d))\end{array} & :      & \mylayer{l}^{n-1,p}(\hd(D))(\downbox{l,\epsilon,q}{n,p}(D)(d))                        \\
          \downlayer{l,\epsilon,q}{n,p}               & D~d~c                           & \defeq & \!\!\begin{array}{l}(\overrightarrow{\cohbox{l,\epsilon,L,q,p}{n,p}(D)(d)}(\downcube{l,\epsilon,q-1}{n-1,p}(\hd(D))(\tl(D))(\downbox{l,L,p}{n,p}(D)(d))(c_L)),\\\;\overrightarrow{\cohbox{l,\epsilon,R,q,p}{n,p}(D)(d)}(\downcube{l,\epsilon,q-1}{n-1,p}(\hd(D))(\tl(D))(\downbox{l,R,p}{n,p}(D)(d))(c_R)))\end{array}                                                         \\
          \\
          \downcube{l,\epsilon,q}{n,p,[p \leq q < n]} & \!\!\!\begin{array}{l}(D:\partialcubset{l}{n})\\(E:\mycubsetcomp{l}{n}(D))\\(d:\mybox{l}^{n,p}(D))\\(b:\mycube{l}^{n,p}(D)(E)(d))\end{array} & :      & \mycube{l}^{n-1,p}(\hd(D))(\tl(D))(\downbox{l,\epsilon,q}{n,p}(D)(d))                 \\
          \downcube{l,\epsilon,q}{n,p,[p=q]}          & D~E~d~(b,\_)                    & \defeq & b_{\epsilon}                                                                          \\
          \downcube{l,\epsilon,q}{n,p,[p<q]}          & D~E~d~(b,c)                     & \defeq & (\downlayer{l,\epsilon,q}{n,p}(D)(d)(b),\downcube{l,\epsilon,q}{n,p+1}(D)(E)(d,b)(c)) \\
          \\
        \end{array}
      $}}
  \caption{Definition of a homogeneous bare cubical set ($q$-th projection)}
  \label{fig:barecubicalsetfaces}
\end{figure*}

\begin{figure*}
  \centerline{ \framebox{$
        \begin{array}{llcl}
          \cohbox{l,\epsilon,\epsilon',q,r}{\!\!\!\begin{array}{l}n,p\\\mbox{}[p \leq r < q < n]\end{array}}    & \!\!\!\begin{array}{l}(D:\partialcubset{l}{n})\\(d:\mybox{l}^{n,p}(D))\end{array} & :      & \!\!\!\begin{array}{ll}\downbox{l,\epsilon,q-1}{n-1,p}(\hd(D))(\downbox{l,\epsilon',r}{n,p}(D)(d)) \\ \eqett \downbox{l,\epsilon',r-1}{n-1,p}(\hd(D))(\downbox{l,\epsilon,q}{n,p}(D)(d))\end{array}                                                                             \\
          \cohbox{l,\epsilon,\epsilon',q,r}{n,0}                                & D~\unitpoint                     & \defeq & \reflett(\unitpoint)                                                                                         \\
          \cohbox{l,\epsilon,\epsilon',q,r}{n,p'+1}                             & D~(d,b)                          & \defeq & (\cohbox{l,\epsilon,\epsilon',q,r}{n,p'}(D)(d),\cohlayer{l,\epsilon,\epsilon',q,r}{n,p'}(D)(d)(b))           \\
          \\
          \cohlayer{l,\epsilon,\epsilon',q,r}{\!\!\!\begin{array}{l}n,p\\\mbox{}[p < r < q < n]\end{array}} & \!\!\!\begin{array}{l}(D:\partialcubset{l}{n})\\(d:\mybox{l}^{n,p}(D))\\(b:\mylayer{l}^{n,p}(D)(d))\end{array} & :      & \!\!\!\begin{array}{ll}\downlayer{l,\epsilon,q}{n-1,p}(\hd(D))(\downbox{l,\epsilon',r}{n,p}(D)(d))(\downlayer{l,\epsilon',r}{n,p}(D)(d)(b)) \\ \eqett \downlayer{l,\epsilon',r}{n-1,p}(\hd(D))(\downbox{l,\epsilon,q}{n,p}(D)(d))(\downlayer{l,\epsilon,q}{n,p}(D)(d)(b))\end{array}                                                                             \\
          \cohlayer{l,\epsilon,\epsilon',q,r}{n,p}                              & D~d~c                            & \defeq & \!\!\begin{array}{l}(\cohcube{l,\epsilon,\epsilon',q-1,r-1}{n-1,p}(\hd(D))(\tl(D))(\downbox{l,L,p}{n,p}(D)(d))(c_L),\\\;\cohcube{l,\epsilon,\epsilon',q-1,r-1}{n-1,p}(\hd(D))(\tl(D))(\downbox{l,R,p}{n,p}(D)(d))(c_R))\end{array}                                                                               \\
          \\
          \cohcube{l,\epsilon,\epsilon',q,r}{\!\!\!\begin{array}{l}n,p\\\mbox{}[p \leq r < q < n]\end{array}}  & \!\!\!\begin{array}{l}(D:\partialcubset{l}{n})\\(E:\mycubsetcomp{l}{n}(D))\\(d:\mybox{l}^{n,p}(D))\\(b:\mycube{l}^{n,p}(D)(E)(d))\end{array} & :      & \!\!\!\begin{array}{ll}\downcube{l,\epsilon,q-1}{n-1,p}(\hd(D))(\tl(D))(\downbox{l,\epsilon',r}{n,p}(D)(d))(\downcube{l,\epsilon',r}{n,p}(D)(E)(d)(b)) \\ \eqett \downcube{l,\epsilon',r-1}{n-1,p}(\hd(D))(\tl(D))(\downbox{l,\epsilon,q}{n,p}(D)(d))(\downcube{l,\epsilon,q}{n,p}(D)(E)(d)(b))\end{array}                                                                             \\
          \cohcube{l,\epsilon,\epsilon',q,r}{n,p,[p=r]}                         & D~E~d~(b,\_)                     & \defeq & \reflett(\downcube{l,\epsilon,q-1}{n-1,p}(\hd(D))(\tl(D))(\downbox{l,\epsilon',p}{n,p}(D)(d))(b_{\epsilon})) \\
          \cohcube{l,\epsilon,\epsilon',q,r}{n,p,[p<r]}                         & D~E~d~(b,c)                      & \defeq & \!\!\begin{array}{l}(\cohlayer{l,\epsilon,\epsilon',q,r}{n,p}(D)(d)(b),\;\cohcube{l,\epsilon,\epsilon',q,r}{n,p+1}(D)(E)(d,b)(c))\end{array}                                                                               \\
        \end{array}
      $}}
  \caption{Definition of a homogeneous bare cubical set (commutation of $q$-th projection and $r$-th projection)}
  \label{fig:barecubicalsetcoherences}
\end{figure*}

\subsection{Derived notions}
\label{sec:derived-notions}

On top of the basic definitions on Figures~\ref{fig:barecubicalsetstructure}, \ref{fig:barecubicalsetfaces} and \ref{fig:barecubicalsetcoherences}, we can define on Figures~\ref{fig:macros} and~\ref{fig:macroscont} a few other concepts at a more standard level of abstraction. In particular, a homogeneous $n$-box in the above sense is defined to be an object of type $\myfullbox{l}^{n}(D_{n})$ for $D_{n} \defeq (A,=_A,...,=^n_A)$ the initial segment of $n$-th first iterated equalities over $A$ (by convention, a homogeneous $0$-box shall be the canonical singleton type). A $n$-tube is of type $\mybox{l}^{n+1,n}(D_{n+1})$. An homogeneous (proper) $n$-cube in some $n$-box $d$ is of type $\mycube{l}^{n+1,n}(\hd(D_{n+1}))(\tl(D_{n+1}))(d)$ and a full $n$-cube is of type $\mycube{l}^{n+1,0}(\hd(D_{n+1}))(\tl(D_{n+1})))(\star)$.

Note that the components of an $n$-box or $n$-tube are tuples associated to the left. For cubes, the components can be associated to the left, and we call this a full $n$-cube or to the right, which we call grounded $n$-cube, and which corresponds to ``complete'' partial cubes, i.e. to partial cubes over the empty box. Figure~\ref{fig:macros} show how to compute the $n$-box surrounding a full $n$-cube over a prefix $D_{n+1}$ of iterated equalities (see $\border{}$).

Figure~\ref{fig:macros} also shows how to compute the faces of a box or of a cube.

\begin{figure*}
  \centerline{ \framebox{$
        \begin{array}{llcl}
          \multicolumn{4}{c}{\mbox{Full (non-truncated) cubical sets with degeneracies}}                                         \\
          \\
          \mycubset{l}        &                          & :      & \sort{l+1}                                                   \\
          \mycubset{l}        &                          & \defeq & \mycubsetfrom{l}{0}(\star)                                   \\
          \\
          \mycubsetfrom{l}{n} & (D:\partialcubset{l}{n}) & :      & \sort{l+1}                                                   \\
          \mycubsetfrom{l}{n} & D                        & \defeq & \Sigma R:\mycubsetcomp{l}{n}(D).\,\mycubsetfrom{l}{n+1}(D,R)
        \end{array}
      $
    }}
  \caption{Definition of a bare cubical set (coinductive structure)}
  \label{fig:barecubicalset}
\end{figure*}

\begin{figure*}
  \centerline{ \framebox{$
        \begin{array}{llcl}
          \multicolumn{4}{c}{\mbox{\textit{Type of full $n$-cubes}}}                                                                                                                            \\
          \\
          \myfullcube{l}^{n}     & \!\!\!\begin{array}{l}(D:\partialcubset{l}{n+1})\end{array}                                                               & :      & \sort{l}                                           \\
          \myfullcube{l}^{n}     & (D,E)                                                                                          & \defeq & \Sigma d:\myfullbox{l}^{n}(D).\, E.\mathsf{rel}(d) \\
          \\
          \multicolumn{4}{c}{\mbox{\textit{Type of $n$-tubes}}}                                                                                                                                 \\
          \\
          \tube{l}^{n}           & (D:\partialcubset{l}{n+1})                                                                     & :      & \sort{l}                                           \\
          \tube{l}^{n}           & D                                                                                              & \defeq & \mybox{l}^{n+1,n}(D)                               \\
          \\
          \multicolumn{4}{c}{\mbox{\textit{Type of grounded $n$-cubes}}}                                                                                                                        \\
          \\
          \mygroundedcube{l}^{n} & (D:\partialcubset{l}{n+1})                                                                     & :      & \sort{l}                                           \\
          \mygroundedcube{l}^{n} & (D,E)                                                                                          & \defeq & \mycube{l}^{n,0}(D)(E)(\star)                      \\
          \\
          \multicolumn{4}{c}{\mbox{\textit{Type of fillers of some $n$-box}}}                                                                                                                   \\
          \\
          \propercube{l}^{n}     & (D:\partialcubset{l}{n+1})~(d:\myfullbox{l}^{n}(\hd(D)))                                       & :      & \sort{l}                                           \\
          \propercube{l}^{n}     & (D,E)~d                                                                                        & \defeq & E.\mathsf{rel}(d)                                  \\
          \\
          \multicolumn{4}{c}{\mbox{\textit{Border of a grounded $n$-cube}}}                                                                                                                     \\
          \\
          \border{l}^{n}         & (D:\partialcubset{l}{n+1})~(c:\mygroundedcube{l}^{n}(D))                                       & :      & \myfullbox{l}^{n}(\hd(D))                          \\
          \border{l}^{n}         & D~c                                                                                            & \defeq & \border{l}^{n,0}(D)(\star)(c)                      \\
          \\
          \multicolumn{4}{c}{\mbox{\textit{where the border of partial cubes is defined by}}}                                                                                                   \\
          \\
          \border{l}^{n,p}       & (D:\partialcubset{l}{n+1})~(d:\mybox{l}^{n,p}(\hd(D)))~(c:\mycube{l}^{n,p}(\hd(D))(\tl(D))(d)) & :      & \myfullbox{l}^{n}(\hd(D))                          \\
          \border{l}^{n,p,[p=n]} & D~d~c                                                                                          & \defeq & d                                                  \\
          \border{l}^{n,p,[p<n]} & D~d~(b,c)                                                                                      & \defeq & \border{l}^{n,p+1}(D)(d,b)(c)                      \\
        \end{array}
      $
    }}
  \caption{Standard derived notions}
  \label{fig:macros}
\end{figure*}

\begin{figure*}
  \centerline{ \framebox{$
        \begin{array}{llcl}
          \multicolumn{4}{c}{\mbox{$q$-th projection of an $n$-box}}                                                                                                                            \\
          \\
          \fulldownbox{l,\epsilon,q}{n,[q < n]}   & (D:\partialcubset{l}{n})~(d:\myfullbox{l}^{n}(D))    & :      & \myfullbox{l}^{n-1}(\hd(D))                                                 \\
          \fulldownbox{l,\epsilon,q}{n}           & D~(d,\_)                                             & \defeq & \downbox{l,\epsilon,q}{n,n-1}(D)(d)                                         \\
          \\
          \multicolumn{4}{c}{\mbox{$q$-th face of an $n$-cube}}                                                                                                                                 \\
          \\
          \fulldowncube{l,\epsilon,q}{n,[q < n]}  & (D:\partialcubset{l}{n+1})~(c:\myfullcube{l}^{n}(D)) & :      & \myfullcube{l}^{n-1}(\hd(D))                                                \\
          \fulldowncube{l,\epsilon,q}{n}          & (D,E)~c                                              & \defeq & \downcube{l,\epsilon,q}{n,0}(D)(E)(\star)(c)                                \\
          \\
          \multicolumn{4}{c}{\mbox{\textit{where the extension of $\downbox{l}{}$ to $p>q$ is}}}                                                                                                \\
          \\
          \downbox{l,\epsilon,q}{n,p,[q < p < n]} & (D:\partialcubset{l}{n})~(d:\mybox{l}^{n,p}(D))      & :      & \myfullbox{l}^{n-1}(\hd(D))                                                 \\
          \downbox{l,\epsilon,q}{n,q+1}           & D~(d,b)                                              & \defeq & \border{l}^{n-1,q}(\hd(D))(\downbox{l,\epsilon,q}{n,q}(D)(d))(b_{\epsilon}) \\
          \downbox{l,\epsilon,q}{n,p,[q < p-1]}   & D~(d,b)                                              & \defeq & \downbox{l,\epsilon,q}{n,p-1}(D)(d)
        \end{array}
      $
    }}
  \caption{Standard derived notions (continued)}
  \label{fig:macroscont}
\end{figure*}

\printbibliography

\end{document}

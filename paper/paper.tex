\documentclass[10pt, titlepage]{amsart}

\usepackage[allbordercolors={.192157 .76862 .28627}]{hyperref} % Emerald Green
\usepackage{bookmark, fontspec, xargs, multirow, polyglossia, luacode, amssymb, listings, color, soulutf8, graphicx, tcolorbox, tabularx, tikz, tikz-cd, booktabs, setspace, makecell, caption, floatrow, amsthm, mathrsfs, lmodern}
\usepackage[margin=0.3in]{geometry}

% Head and tail
\newcommand{\hd}{\ensuremath{\mathsf{hd}}}
\newcommand{\tl}{\ensuremath{\mathsf{tl}}}

% Universes
\newcommand{\sort}[1]{\ensuremath{\mathsf{U}_{#1}}}

% The unit type
\newcommand{\unittype}{\ensuremath{\mathsf{unit}}}
\newcommand{\unitpoint}{\ensuremath{\star}}

% Definitional and extensional equality
\newcommand{\defeq}{\ensuremath{\triangleq}}
\newcommand{\eqett}{\ensuremath{\equiv_{\mathit{ETT}}}}
\newcommand{\reflett}{\ensuremath{\mathsf{refl}_{\mathit{ETT}}}}

% Box, layer, and cube
\newcommand{\mycubset}[1]{\ensuremath{\mathsf{cubset}_{#1}}}
\newcommand{\mybox}[2]{\ensuremath{\mathsf{box}_{#1}^{#2}}}
\newcommand{\mylayer}[2]{\ensuremath{\mathsf{layer}_{#1}^{#2}}}
\newcommand{\mycube}[2]{\ensuremath{\mathsf{cube}_{#1}^{#2}}}

% Subbox, sublayer, and subcube
\newcommand{\downbox}[2]{\ensuremath{\mathsf{subbox}_{#1}^{#2}}}
\newcommand{\downlayer}[2]{\ensuremath{\mathsf{sublayer}_{#1}^{#2}}}
\newcommand{\downcube}[2]{\ensuremath{\mathsf{subcube}_{#1}^{#2}}}
\newcommand{\myfullbox}[2]{\ensuremath{\mathsf{fullbox}_{#1}^{#2}}}

% Coherence conditions
\newcommand{\cohbox}[2]{\ensuremath{\mathsf{cohbox}_{#1}^{#2}}}
\newcommand{\cohlayer}[2]{\ensuremath{\mathsf{cohlayer}_{#1}^{#2}}}
\newcommand{\cohcube}[2]{\ensuremath{\mathsf{cohcube}_{#1}^{#2}}}

% Partials and fulls related to box, cube, and tube
\newcommand{\border}[1]{\ensuremath{\mathsf{border}_{#1}}}
\newcommand{\partialcubset}[2]{\ensuremath{\mathsf{cubset}_{#1}^{<#2}}}
\newcommand{\mycubsetcomp}[2]{\ensuremath{\mathsf{cubset}_{#1}^{=#2}}}
\newcommand{\fulldownbox}[2]{\ensuremath{\mathsf{subfullbox}_{#1}^{#2}}}
\newcommand{\fulldowncube}[2]{\ensuremath{\mathsf{subfullcube}_{#1}^{#2}}}
\newcommand{\propercube}[1]{\ensuremath{\mathsf{properfiller}_{#1}}}
\newcommand{\mygroundedcube}[1]{\ensuremath{\mathsf{groundedcube}_{#1}}}
\newcommand{\tube}[1]{\ensuremath{\mathsf{tube}_{#1}}}
\newcommand{\mycubsetfrom}[2]{\ensuremath{\mathsf{cubset}_{#1}^{\geq#2}}}
\newcommand{\myfullcube}[1]{\ensuremath{\mathsf{fullcube}_{#1}}}

% Logical implication
\newcommand{\imp}{\ensuremath{\rightarrow}}
\newcommand{\overright}[1]{\ensuremath{\overrightarrow{#1}}}

% The eqntable environment, displaying the various
\NewDocumentEnvironment{eqntable}{m}{\table[!ht]
\tabularx{\textwidth}{@{} l c c >{\centering\arraybackslash}X @{}}\toprule}
{\endtabularx\hrule\caption{#1}\endtable}

% A block in the eqntable environment
\newcommand{\eqnline}[4]{$#1$ & $#2$ & $#3$ & $#4$ \\}
\newcommand{\mc}[1]{\multicolumn{4}{c}{\textit{#1}} \\\\}

% Table caption set up
\DeclareCaptionFormat{hfillstart}{\hfill#1#2#3\par}
\DeclareCaptionFont{mdit}{\mdseries\itshape}
\captionsetup[table]{
  justification=centering,
  font=bf,
  labelfont=mdit,
}
\DeclareFloatFont{small}{\small}
\floatsetup[table]{capposition=bottom, font=small}

% Lettered footnotes
\renewcommand{\thefootnote}{\alph{footnote}}

% Theorems
\newtheorem{notation}{Notation}
\newtheorem{definition}{Definition}
\newtheorem{lemma}{Lemma}
\newtheorem{proposition}{Proposition}
\newtheorem{theorem}{Theorem}
\newtheorem{remark}{Remark}
\newtheorem{example}{Example}

% Simplicial sets
\newcommand{\Set}{\ensuremath{\mathsf{Set}}}
\newcommand{\Simplex}[1]{\ensuremath{\boldsymbol{\Delta^{#1}}}}
\newcommand{\DeltaHat}{\ensuremath{\hat{\boldsymbol{\Delta}}}}
\newcommand{\SSet}{\ensuremath{\mathsf{Set}_{\boldsymbol{\Delta}}}}
\newcommand{\sq}[1]{\ensuremath{\mathsf{[#1]}}}
\newcommand{\sqsn}{\ensuremath{\mathsf{[1 \ldots n]}}}

% Cubical sets
\newcommand{\Cube}[1]{\ensuremath{\boldsymbol{\square^{#1}}}}
\newcommand{\I}[1]{\ensuremath{\mathsf{I}^{#1}}}
\newcommand{\CSet}{\ensuremath{\mathsf{Set}_{\boldsymbol{\square}}}}

% Type theory
\newcommand{\U}{\ensuremath{\mathscr{U}}}

\title{An iterated Grothendieck construction of semi-cubical types}
\author{Ramkumar Ramachandra}
\author{Hugo Herbelin}

\begin{document}
\begin{abstract}
  We present a dependently-typed construction of semi-cubical sets, along with an accompanying artifact in Coq.
\end{abstract}
\maketitle
\tableofcontents

\section{Introduction}

Cubical type theory is an extension of dependent type theory that provides a computational interpretation of Voevodsky's Univalence Axiom. Several strides in the field have been made since \cite{Bezem14} in constructing a model of type theory for cubical sets, and we present another approach to the construction, with a rigorous mathematical foundation. Though motivated by homotopy type theory, construction of the path type and transport are beyond the scope of our work; we merely formalize semi-cubical sets. Our secondary contribution is a mechanically checked proof of our construction in vanilla Coq~\footnote{\href{https://github.com/artagnon/bonak}{github.com/artagnon/bonak}}.

\section{The intutution from \texorpdfstring{\SSet}{simplicial sets}}
\subsection{Standard definitions}

Let us first review a few standard definitions appearing in mathematical literature. We will refer the reader to \cite{Friedman08} and \cite{Riehl11} for more on the subject.

\begin{notation}[\sq{n}]
  The notation \sq{n} is used to denote the set of finite ordinals, with preorder structure.

  \begin{equation*}
    \sq{n} := 0 \rightarrow 1 \rightarrow \ldots \rightarrow n
  \end{equation*}
\end{notation}

\begin{definition}[Informal definition of \Simplex{n}]
  The standard $n$-simplex is the prototypical example of a simplicial set,  can be informally defined as all ordered subsets of \sq{n}, also denoted \sqsn.
\end{definition}

\begin{example}[\Simplex{0}, \Simplex{1}, and \Simplex{2}]
  $\Simplex{0}$ can be drawn as:

  $$
    \begin{tikzcd}
      0
    \end{tikzcd}
  $$

  $\Simplex{1}$ can be drawn as:

  $$
    \begin{tikzcd}
      0 \arrow[r] & 1
    \end{tikzcd}
  $$

  and $\Simplex{2}$ can be drawn as:

  $$
    \begin{tikzcd}
      & 1 \arrow[dr] & \\
      0 \arrow[ur] \arrow[rr] & & 2
    \end{tikzcd}
  $$

  Note that we only show non-degerate simplices in the figure.
\end{example}

We will now define the category of simplicial sets, and redefine $\Simplex{n}$ formally.

\begin{definition}[\Simplex{}]
  The $\Simplex{}$ category can be defined in terms of its objects and morphisms:

  \begin{align*}
    obj(\Simplex{}) & := \sq{n}                                             \\
    mor(\Simplex{}) & := \sq{m} \rightarrow \sq{n}, \text{order-preserving}
  \end{align*}
\end{definition}

\begin{definition}[\SSet]
  The category of simplicial sets is then defined as the functor category:

  \begin{equation*}
    \SSet := \Set^{\Simplex{}^{op}}
  \end{equation*}
\end{definition}

\begin{definition}[\Simplex{n}]
  $\Simplex{n}$ is the Yoneda embedding of \sq{n}.

  \begin{equation*}
    \Simplex{n} := y(\sq{n})
  \end{equation*}
\end{definition}

To concretely define simplicial sets in terms of objects and morphisms, they can be viewed from two different equivalent viewpoints.

\begin{definition}[\SSet, from the first viewpoint]
  \begin{align*}
    obj(\SSet) & := \text{functors}, \Simplex{} \rightarrow \Set   \\
    mor(\SSet) & := \text{natural transforms between the functors}
  \end{align*}
\end{definition}

\begin{definition}[\SSet, from the second viewpoint]
  \begin{align*}
    obj(\SSet) & := \text{ordered n-simplices}, X_n \\
    mor(\SSet) & :=
    \begin{cases}
      d_i : X_{n + 1} \rightarrow X_n & \text{face maps}       \\
      s_i : X_n \rightarrow X_{n + 1} & \text{degeneracy maps}
    \end{cases}
  \end{align*}

  where $X_n$ is defined as:
  \begin{equation*}
    X_n := X(\sq{n}) = \text{Hom}(\Simplex{n}, X), \text{where X is the simplicial set}
  \end{equation*}

  and $d_i$, $s_i$ are constrained by the following ``gluing conditions'':

  \begin{align*}
    d_i d_j & = d_{j - 1} d_i, i < j \\
    s_i s_j & = s_{j + 1} s_i, i < j \\
    d_i s_j & =
    \begin{cases}
      s_{j - 1} d_i & i < j \\
      id            & i = j \\
      s_j d_{i - 1} & i > j
    \end{cases}
  \end{align*}
\end{definition}

\subsection{The motivation for \texorpdfstring{\DeltaHat}{delta sets}}

The issue with mechanisation of \SSet, and the corresponding cubical sets, is the degeneracy maps - there are an infinite number of degenerate simplices in any simplicial set, and these degeneracies are not interesting from the point of view of cubical type theory. To remedy this problem, let us instead use \DeltaHat, and proceed towards building, what is called semi-cubical sets in literature. $\DeltaHat$ is identical to \Simplex{}, but for the fact that the maps $\sq{m} \rightarrow \sq{n}$ are strictly order-preserving. The degeneracies then vanish, and we're left with the task of defining gluing conditions solely on the basis of face maps. In mathematical literature, there is little interest in studying \DeltaHat, since delta maps are not well-defined, and constructing a $\Set^{\DeltaHat^{op}}$ is inelegant. The objective of this paper is to first generalize delta-sets to the cubical setting, and provide a precise mechanically-checked formalization of the same.

\subsection{Towards semi-cubical sets}

Let us first motivate the definition of semi-cubical sets, as defined in \cite{Antolini00}. On the outset, we might naively attempt to define $\Cube{}$ identically to \DeltaHat, but let us briefly explain why this wouldn't work. If there were morphisms from every $\sq{m}$ to $\sq{n}$, we would end up with:

$$
  \begin{tikzcd}
    \bullet \arrow[r] \arrow[dr] \arrow[d] & \bullet \arrow[d] \\
    \bullet \arrow[r] \arrow[ur] & \bullet
  \end{tikzcd}
$$

where the filling conditions are conflated with the cube itself. $\Simplex{n}$ can be defined quite simply as the convex hull of $n$ points, but even defining the standard $n$-cube becomes a problem if we start with $\sq{n}$, but the situation becomes much more amenable if we define:

\begin{definition}[\Cube{n}]
  \begin{equation*}
    \Cube{n} := \I{n} = \sq{0, 1}^n
  \end{equation*}
\end{definition}

\begin{example}[$\Cube{0}$, $\Cube{1}$ and $\Cube{2}$]
  $\Cube{0}$ can be drawn as:

  $$
    \begin{tikzcd}
      0 \equiv 1
    \end{tikzcd}
  $$

  $\Cube{1}$ can be drawn as:

  $$
    \begin{tikzcd}
      0 \arrow[r, dash] & 1
    \end{tikzcd}
  $$

  and $\Cube{2}$ can be drawn as:

  $$
    \begin{tikzcd}
      (0, 1) \arrow[r, dash] & (1, 1) \arrow[d, dash] \\
      (0, 0) \arrow[u, dash] & (1, 0) \arrow[l, dash]
    \end{tikzcd}
  $$
\end{example}

Here, $\I{n}$ serves the purpose of $\sq{n}$, but this change will cascade into other definitions. In view of defining a category \CSet, we will restrict the morphisms in \Cube{}.

\begin{definition}[\Cube{}]
  \begin{align*}
    obj(\Cube{}) & := \I{n}                                           \\
    mor(\Cube{}) & := \delta^\epsilon_i : \I{n + 1} \rightarrow \I{n}
  \end{align*}

  where $\delta^\epsilon_i$ must satisfy the corresponding face map condition:

  \begin{equation*}
    \delta^\epsilon_i \delta^\omega_j = \delta^\omega_{j - 1} \delta^\epsilon_i
  \end{equation*}

  where $\epsilon$ and $\omega$ correspond to opposite faces.
\end{definition}

\begin{definition}[\CSet]
  Just as in \SSet, we define semi-cubical sets as the functor category:

  \begin{equation*}
    \CSet := \Set^{\Cube{}^{op}}
  \end{equation*}
\end{definition}

Or, in terms of objects and morphisms:

\begin{definition}[$\CSet$ in terms of objects and morphisms]
  \begin{align*}
    obj(\CSet) & := X_n                                                   \\
    mor(\CSet) & := X_\lambda, \text{where $\lambda$ is \Cube{}-morphism}
  \end{align*}

  where we term $X_n$ as the $n$-cubex, and $X_\lambda$ as the ``face map'', defined similarly:

  \begin{align*}
    X_n       & := X(\I{n}), \text{where X is the semi-cubical set} \\
    X_\lambda & := X(\lambda)
  \end{align*}
\end{definition}

\begin{theorem}[$\CSet$ does not admit degeneracies]
  This is proved in \cite{Antolini00}.
\end{theorem}

\section{Iterated Grothendieck construction}

The Grothendieck construction provides a correspondence between a fibered representation, and an indexed representation.

$$
  \begin{tikzcd}
    X_0 : \U & X_1 : \U \arrow[l, "\delta^\epsilon_0" description, shift left=2] \arrow[l, "\delta^\omega_0" description, shift right=2] & X_2 : \U \arrow[l, "\delta^\epsilon_1" description, shift left=6] \arrow[l, "\delta^\epsilon_0" description, shift left=2] \arrow[l, "\delta^\omega_0" description, shift right=2] \arrow[l, "\delta^\omega_1" description, shift right=6] & \ldots
  \end{tikzcd}
$$

\begin{align*}
  X_0 & : \U                                                                                                      \\
  X_1 & : X_0 \times X_0 \rightarrow \U                                                                           \\
  X_2 & : \forall a b c d, X_1 : ab \rightarrow X_1 : bc \rightarrow X_1 : cd \rightarrow X_1 : da \rightarrow \U \\
  \ldots
\end{align*}

\section{Semi-cubical sets}

In this section, we give the core of the definition of semi-cubical
sets as the coinductive limit (see table \ref{tab:barecubicalset}) of a
construction of truncated semi-cubical sets.

\subsection{Truncated semi-cubical sets}

The definition is dispatched over tables \ref{tab:barecubicalsetstructurecore}, \ref{tab:barecubicalsetstructure}, \ref{tab:barecubicalsetfaces} and \ref{tab:barecubicalsetcoherences}. It describes the structure of the underlying higher-dimensional relations on which cubes and boxes are built, together with the definition of homogeneous $n$-boxes, homogeneous $n$-cubes, together with face operations on boxes and cubes, together with commutation properties of the face operations. All are mutually defined as types of the target language ETT. Note that such relational structure, cubes and boxes are relative to a universe. Note the presence of a coherence condition $\cohbox{l}{}$ to ensuring that both sides of the equality in $\downlayer{l}{}$ and $\downcube{l}{}$ are in the same type. The proof of $\cohbox{l}{}$ itself requires an higher-dimensional coherence condition which we obtain by working here in ETT where all proofs of an equality are identified (principle of Unicity of Identity Proofs). Note that if the proofs of the same equality were not equated, there would be a need for arbitrary many higher-dimensional coherences (see e.g.~\cite{Herbelin15} for a discussion on the de facto need for recursive higher-dimensional coherence conditions in formulating higher-dimensional structures in type theory). Note also that for a given $n$, the coherence conditions evaluate to a reflexivity proof, so that the construction evaluates to an effective sequence of types of iterated relations not mentioning $\downbox{l}{}$ nor $\cohbox{l}{}$ anymore.

When reflexivities are excepted, we call the structure thus defined \emph{bare truncated cubical sets}: \emph{bare} because it can be seen as defining semi-cubical sets corresponding to semi-simplicial sets with only faces as part of the structure; \emph{truncated} because we consider only such cubical sets up to some fixed dimension.

\begin{eqntable}{\label{tab:barecubicalsetstructurecore} Definition of a truncated cubical set (main part of the higher-dimensional relation structure)}
  \mc{Type of truncated cubical sets}
  \eqnline{\partialcubset{l}{n}}{}{:}{\sort{l+1}}
  \eqnline{\partialcubset{l}{0}}{}{\defeq}{\unittype}
  \eqnline{\partialcubset{l}{n'+1}}{}{\defeq}{\ensuremath{\Sigma}D:\partialcubset{l}{n'}.\,\mycubsetcomp{l}{n'}(D)}

  \\

  \mc{Structure carried at each dimension}
  \eqnline{\mycubsetcomp{l}{n}}{(D:\partialcubset{l}{n})}{:}{\sort{l}}
  \eqnline{\mycubsetcomp{l}{n}}{D}{\defeq}{\myfullbox{l}{n}(D)\imp \sort{l}}
\end{eqntable}

\begin{eqntable}{\label{tab:barecubicalsetstructure} Definition of a truncated cubical set (boxes)}
  \mc{Full homogeneous $n$-boxes}
  \eqnline{\myfullbox{l}{n}}{(D:\partialcubset{l}{n})}{:}{\sort{l}}
  \eqnline{\myfullbox{l}{n}}{D}{\defeq}{\mybox{l}{n,n}(D)}

  \\

  \mc{Homogeneous partial $n$-boxes and homogeneous partial $n$-cubes}
  \eqnline{\mybox{l}{n,p,[p \leq n]}}{(D:\partialcubset{l}{n})}{:}{\sort{l}}
  \eqnline{\mybox{l}{n,0}}{D}{\defeq}{\unittype}
  \eqnline{\mybox{l}{n,p'+1}}{D}{\defeq}{\Sigma d:\mybox{l}{n,p'}(D).\,\mylayer{l}{n,p'}(D)(d)}

  \\

  \eqnline{\mylayer{l}{n,p,[p < n]}}{\makecell{(D:\partialcubset{l}{n}) \\ (d:\mybox{l}{n,p}(D))}}{:}{\sort{l}}

  \eqnline{\mylayer{l}{n,p}}{D~d}{\defeq}{\makecell{\mycube{l}{n-1,p}(\hd(D))(\tl(D))(\downbox{l,L,p}{n,p}(D)(d)) \\ \times\;\mycube{l}{n-1,p}(\hd(D))(\tl(D))(\downbox{l,R,p}{n,p}(D)(d))}}

  \\

  \eqnline{\mycube{l}{n,p,[p \leq n]}}{\makecell{(D:\partialcubset{l}{n}) \\(E:\mycubsetcomp{l}{n}(D)) \\ (d:\mybox{l}{n,p}(D))}}{:}{\sort{l}}

  \eqnline{\mycube{l}{n,p,[p = n]}}{D~E~d}{\defeq}{E.\mathsf{rel}(d)}

  \eqnline{\mycube{l}{n,p,[p < n]}}{D~E~d}{\defeq}{\Sigma b:\mylayer{l}{n,p}(D)(d).\,\mycube{l}{n,p+1}(D)(E)(d,b)}
\end{eqntable}

\begin{eqntable}{\label{tab:barecubicalsetfaces} Definition of a homogeneous bare cubical set ($q$-th projection)}

  \eqnline{\downbox{l,\epsilon,q}{n,p,[p \leq q \leq n - 1]}}{\makecell{(D:\partialcubset{l}{n}) \\ (d:\mybox{l}{n,p}(D))}}{:}{\mybox{l}{n-1,p}(\hd(D))}

  \eqnline{\downbox{l,\epsilon,q}{n,p'+1}}{D~(d,b)}{\defeq}{(\downbox{l,\epsilon,q}{n,p'}(D)(d),\downlayer{l,\epsilon,q-1}{n,p'}(D)(d)(b))}

  \\

  \eqnline{\downlayer{l,\epsilon,q}{n,p,[p \leq q \leq n - 2]}}{\makecell{(D:\partialcubset{l}{n}) \\ (d:\mybox{l}{n,p}(D)) \\ (b:\mylayer{l}{n,p}(D)(d))}}{:}{\mylayer{l}{n-1,p}(\hd(D))(\downbox{l,\epsilon,q+1}{n,p}(D)(d))}

  \eqnline{\downlayer{l,\epsilon,q}{n,p}}{D~d~c}{\defeq}{\makecell{(\overright{\cohbox{l,\epsilon,L,q,p}{n,p}(D)(d)}(\downcube{l,\epsilon,q}{n-1,p}(\hd(D))(\tl(D))(\downbox{l,L,p}{n,p}(D)(d))(c_L)), \\ \;\overright{\cohbox{l,\epsilon,R,q,p}{n,p}(D)(d)}(\downcube{l,\epsilon,q}{n-1,p}(\hd(D))(\tl(D))(\downbox{l,R,p}{n,p}(D)(d))(c_R)))}}

  \\

  \eqnline{\downcube{l,\epsilon,q}{n,p,[p \leq q \leq n - 1]}}{\makecell{(D:\partialcubset{l}{n}) \\ (E:\mycubsetcomp{l}{n}(D)) \\(d:\mybox{l}{n,p}(D)) \\ (b:\mycube{l}{n,p}(D)(E)(d))}}{:}{\mycube{l}{n-1,p}(\hd(D))(\tl(D))(\downbox{l,\epsilon,q+1}{n,p}(D)(d))}

  \eqnline{\downcube{l,\epsilon,q}{n,p,[p=q]}}{D~E~d~(b,\_)}{\defeq}{b_\epsilon}

  \eqnline{\downcube{l,\epsilon,q}{n,p,[p<q]}}{D~E~d~(b,c)}{\defeq}{(\downlayer{l,\epsilon,q}{n,p}(D)(d)(b),\downcube{l,\epsilon,q}{n,p+1}(D)(E)(d,b)(c))}
\end{eqntable}

\begin{eqntable}{\label{tab:barecubicalsetcoherences} Definition of a homogeneous bare cubical set (commutation of $q$-th projection and $r$-th projection)}

  \eqnline{\cohbox{l,\epsilon,\omega,q,r}{n,p,[p \leq r \leq q \leq n - 2]}}{\makecell{(D:\partialcubset{l}{n}) \\ (d:\mybox{l}{n,p}(D))}}{:}{\makecell{\downbox{l,\epsilon,q}{n-1,p}(\hd(D))(\downbox{l,\omega,r}{n,p}(D)(d)) \\ \eqett \downbox{l,\omega,r}{n-1,p}(\hd(D))(\downbox{l,\epsilon,q+1}{n,p}(D)(d))}}

  \eqnline{\cohbox{l,\epsilon,\omega,q,r}{n,0}}{D~\unitpoint}{\defeq}{\reflett}{(\unitpoint)}

  \eqnline{\cohbox{l,\epsilon,\omega,q,r}{n,p'+1}}{D~(d,b)}{\defeq}{(\cohbox{l,\epsilon,\omega,q,r}{n,p'}(D)(d),\cohlayer{l,\epsilon,\omega,q,r}{n,p'}(D)(d)(b))}

  \\

  \eqnline{\cohlayer{l,\epsilon,\omega,q,r}{n,p,[p < r \leq q \leq n - 2]}}{\makecell{(D:\partialcubset{l}{n}) \\ (d:\mybox{l}{n,p}(D)) \\(b:\mylayer{l}{n,p}(D)(d))}}{:}{\makecell{\downlayer{l,\epsilon,q}{n-1,p}(\hd(D))(\downbox{l,\omega,r}{n,p}(D)(d))(\downlayer{l,\omega,r}{n,p}(D)(d)(b)) \\ \eqett \downlayer{l,\omega,r}{n-1,p}(\hd(D))(\downbox{l,\epsilon,q+1}{n,p}(D)(d))(\downlayer{l,\epsilon,q+1}{n,p}(D)(d)(b))}}

  \eqnline{\cohlayer{l,\epsilon,\omega,q,r}{n,p}}{D~d~c}{\defeq}{\makecell{(\cohcube{l,\epsilon,\omega,q-1,r-1}{n-1,p}(\hd(D))(\tl(D))(\downbox{l,L,p}{n,p}(D)(d))(c_L), \\ \;\cohcube{l,\epsilon,\omega,q-1,r-1}{n-1,p}(\hd(D))(\tl(D))(\downbox{l,R,p}{n,p}(D)(d))(c_R))}}

  \\

  \eqnline{\cohcube{l,\epsilon,\omega,q,r}{n,p,[p \leq r \leq q \leq n - 2]}}{\makecell{(D:\partialcubset{l}{n}) \\ (E:\mycubsetcomp{l}{n}(D)) \\ (d:\mybox{l}{n,p}(D)) \\ (b:\mycube{l}{n,p}(D)(E)(d))}}{:}{\makecell{\downcube{l,\epsilon,q}{n-1,p}(\hd(D))(\tl(D))(\downbox{l,\omega,r}{n,p}(D)(d))(\downcube{l,\omega,r}{n,p}(D)(E)(d)(b)) \\ \eqett \downcube{l,\omega,r}{n-1,p}(\hd(D))(\tl(D))(\downbox{l,\epsilon,q+1}{n,p}(D)(d))(\downcube{l,\epsilon,q+1}{n,p}(D)(E)(d)(b))}}

  \eqnline{\cohcube{l,\epsilon,\omega,q,r}{n,p,[p=r]}}{D~E~d~(b,\_)}{\defeq}{\reflett(\downcube{l,\epsilon,q-1}{n-1,p}(\hd(D))(\tl(D))(\downbox{l,\omega,p}{n,p}(D)(d))(b_{\epsilon}))}

  \eqnline{\cohcube{l,\epsilon,\omega,q,r}{n,p,[p<r]}}{D~E~d~(b,c)}{\defeq}{(\cohlayer{l,\epsilon,\omega,q,r}{n,p}(D)(d)(b),\;\cohcube{l,\epsilon,\omega,q,r}{n,p+1}(D)(E)(d,b)(c))}
\end{eqntable}

\subsection{Derived notions}
\label{sec:derived-notions}

On top of the basic definitions on tables \ref{tab:barecubicalsetstructure}, \ref{tab:barecubicalsetfaces} and \ref{tab:barecubicalsetcoherences}, we can define on tables \ref{tab:macros} and \ref{tab:macroscont} a few other concepts at a more standard level of abstraction. In particular, a homogeneous $n$-box in the above sense is defined to be an object of type $\myfullbox{l}{n}(D_{n})$ for $D_{n} \defeq (A, =_A, \ldots, =^n_A)$ the initial segment of $n$-th first iterated equalities over $A$ (by convention, a homogeneous $0$-box shall be the canonical singleton type). A $n$-tube is of type $\mybox{l}{n+1,n}(D_{n+1})$. An homogeneous (proper) $n$-cube in some $n$-box $d$ is of type $\mycube{l}{n+1,n}(\hd(D_{n+1}))(\tl(D_{n+1}))(d)$ and a full $n$-cube is of type $\mycube{l}{n+1,0}(\hd(D_{n+1}))(\tl(D_{n+1})))(\star)$.

Note that the components of an $n$-box or $n$-tube are tuples associated to the left. For cubes, the components can be associated to the left, and we call this a full $n$-cube or to the right, which we call grounded $n$-cube, and which corresponds to ``complete'' partial cubes, i.e. to partial cubes over the empty box. Table \ref{tab:macros} show how to compute the $n$-box surrounding a full $n$-cube over a prefix $D_{n+1}$ of iterated equalities (see $\border{}$).

Table \ref{tab:macros} also shows how to compute the faces of a box or of a cube.

\begin{eqntable}{\label{tab:barecubicalset} Definition of a bare cubical set (coinductive structure)}
  \mc{Full (non-truncated) cubical sets with degeneracies}
  \eqnline{\mycubset{l}}{}{:}{\sort{l+1}}
  \eqnline{\mycubset{l}}{}{\defeq}{\mycubsetfrom{l}{0}(\star)}
  \\
  \eqnline{\mycubsetfrom{l}{n}}{(D:\partialcubset{l}{n})}{:}{\sort{l+1}}
  \eqnline{\mycubsetfrom{l}{n}}{D}{\defeq}{\Sigma R:\mycubsetcomp{l}{n}(D).\,\mycubsetfrom{l}{n+1}(D,R)}
\end{eqntable}

\begin{eqntable}{\label{tab:macros} Standard derived notions}
  \mc{Type of full $n$-cubes}
  \eqnline{\myfullcube{l}^{n}}{(D:\partialcubset{l}{n+1})}{:}{\sort{l}}

  \eqnline{\myfullcube{l}^{n}}{(D,E)}{\defeq}{\Sigma d:\myfullbox{l}{n}(D).\, E.\mathsf{rel}(d)}

  \\ \mc{Type of $n$-tubes}

  \eqnline{\tube{l}^{n}}{(D:\partialcubset{l}{n+1})}{:}{\sort{l}}
  \eqnline{\tube{l}^{n}}{D}{\defeq}{\mybox{l}{n+1,n}(D)}

  \\ \mc{Type of grounded $n$-cubes}

  \eqnline{\mygroundedcube{l}^{n}}{(D:\partialcubset{l}{n+1})}{:}{\sort{l}}
  \eqnline{\mygroundedcube{l}^{n}}{(D,E)}{\defeq}{\mycube{l}{n,0}(D)(E)(\star)}

  \\ \mc{Type of fillers of some $n$-box}

  \eqnline{\propercube{l}^{n}}{(D:\partialcubset{l}{n+1})~(d:\myfullbox{l}{n}(\hd(D)))}{:}{\sort{l}}

  \eqnline{\propercube{l}^{n}}{(D,E)~d}{\defeq}{E.\mathsf{rel}(d)}

  \\ \mc{Border of a grounded $n$-cube}

  \eqnline{\border{l}^{n}}{(D:\partialcubset{l}{n+1})~(c:\mygroundedcube{l}^{n}(D))}{:}{\myfullbox{l}{n}(\hd(D))}

  \eqnline{\border{l}^{n}}{D~c}{\defeq}{\border{l}^{n,0}(D)(\star)(c)}

  \\ \mc{where the border of partial cubes is defined by}

  \eqnline{\border{l}^{n,p}}{(D:\partialcubset{l}{n+1})~(d:\mybox{l}{n,p}(\hd(D)))~(c:\mycube{l}{n,p}(\hd(D))(\tl(D))(d))}{:}{\myfullbox{l}{n}(\hd(D))}

  \eqnline{\border{l}^{n,p,[p=n]}}{D~d~c}{\defeq}{d}

  \eqnline{\border{l}^{n,p,[p<n]}}{D~d~(b,c)}{\defeq}{\border{l}^{n,p+1}(D)(d,b)(c)}
\end{eqntable}

\begin{eqntable}{\label{tab:macroscont} Standard derived notions (continued)}
  \mc{$q$-th projection of an $n$-box}

  \eqnline{\fulldownbox{l,\epsilon,q}{n,[q < n]}}{(D:\partialcubset{l}{n})~(d:\myfullbox{l}{n}(D))}{:}{\myfullbox{l}{n-1}(\hd(D))}

  \eqnline{\fulldownbox{l,\epsilon,q}{n}}{D~(d,\_)}{\defeq}{\downbox{l,\epsilon,q}{n,n-1}(D)(d)}

  \\ \mc{$q$-th face of an $n$-cube}

  \eqnline{\fulldowncube{l,\epsilon,q}{n,[q < n]}}{(D:\partialcubset{l}{n+1})~(c:\myfullcube{l}^{n}(D))}{:}{\myfullcube{l}^{n-1}(\hd(D))}

  \eqnline{\fulldowncube{l,\epsilon,q}{n}}{(D,E)~c}{\defeq}{\downcube{l,\epsilon,q}{n,0}(D)(E)(\star)(c)}

  \\ \mc{where the extension of $\downbox{l}{}$ to $p>q$ is}

  \eqnline{\downbox{l,\epsilon,q}{n,p,[q < p < n]}}{(D:\partialcubset{l}{n})~(d:\mybox{l}{n,p}(D))}{:}{\myfullbox{l}{n-1}(\hd(D))}

  \eqnline{\downbox{l,\epsilon,q}{n,q+1}}{D~(d,b)}{\defeq}{\border{l}^{n-1,q}(\hd(D))(\downbox{l,\epsilon,q}{n,q}(D)(d))(b_{\epsilon})}

  \eqnline{\downbox{l,\epsilon,q}{n,p,[q < p-1]}}{D~(d,b)}{\defeq}{\downbox{l,\epsilon,q}{n,p-1}(D)(d)}
\end{eqntable}

\begin{thebibliography}{10}
  \bibitem[Bez]{Bezem14}
  Bezem, M., Coquand, T., \& Huber, S. (2014, July). A model of type theory in cubical sets. \textit{In 19th International Conference on Types for Proofs and Programs (TYPES 2013)} (Vol. 26, pp. 107-128). Wadern, Germany: Schloss Dagstuhl–Leibniz Zentrum fuer Informatik.

  \bibitem[Fri]{Friedman08}
  Friedman, G. (2008). An elementary illustrated introduction to simplicial sets. \textit{arXiv preprint arXiv:0809.4221}.

  \bibitem[Rie]{Riehl11}
  Riehl, E. (2011). A leisurely introduction to simplicial sets. \textit{Unpublished expository article available online at \href{http://www.math.harvard.edu/~eriehl}{http://www.math.harvard.edu/~eriehl}}.

  \bibitem[Ant]{Antolini00}
  Antolini, R. (2000). Cubical structures, homotopy theory. \textit{Annali di Matematica pura ed applicata, 178}(1), 317-324.

  \bibitem[Her]{Herbelin15}
  Herbelin, H. (2015). A dependently-typed construction of semi-simplicial types. \textit{Mathematical Structures in Computer Science, 25}(5), 1116-1131.
\end{thebibliography}

\end{document}
